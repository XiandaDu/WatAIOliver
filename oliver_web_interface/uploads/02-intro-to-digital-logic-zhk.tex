\setlength{\columnseprule}{1pt}
\def\columnseprulecolor{\color{blue}}


\begin{frame}[fragile]
\STitle{A Brief Look at Electricity}
\begin{itemize}
	\item Computers work with current/voltage that are related as $V = IR$,   where V is voltage, I is current and R is resistance
\begin{tcolorbox}[enhanced,attach boxed title to top center={yshift=-3mm,yshifttext=-1mm},
  colback=blue!5!white,colframe=blue!75!black,colbacktitle=blue!80!black,
  title=Think About It,fonttitle=\bfseries,
  boxed title style={size=small,colframe=red!50!black} ]
    Assuming voltage (V) is \textbf{fixed}
    \begin{itemize}
        \item a low resistance (R) in the circuit implies high current(I), and 
        \item a high resistance (R) in the circuit implies low current(I)
    \end{itemize}
    % What would happen if we connect power to a circuit with low resistance capacitor:
				% we get high current flowing, and can burn
				% out the transistor
\end{tcolorbox}
    
 \item These quantities are continuous, e.g. plot of voltage (V) vs time (t).

	\Figure{!}{1in}{0.95in}{Figs/contsig}

	\item We can "digitize" the analog signals.
\end{itemize}
\end{frame}
\BNotes\ifnum\Notes=1
\begin{frame}[fragile]
Instructor's Notes:
\begin{itemize}
	\item A brief review (?) of electricity would be appropriate.
		In particular, mention:
		\begin{itemize}
			\item Current ($I$) flows from power to ground.  There 
				is a lot of confusion around which way current
				flows and which way electrons flow; this is
				sufficient for what we need in the course, but
				feel free to confuse them with the details
				if you like.  We don't actually care which
				way the current flows or which way the
				electrons flow in this course.
			\item The $V=IR$ formula tells us that the smaller
				the resistance, the higher the current
				(assuming voltage is fixed, which it is
				in what we're considering).  This will be
				important later for two reasons (you don't
				need to give these reasons now):

				This is important when we look at transistors:
				when a transistor has high resistance, no 
				current flows, and when there is low resistance,
				current flows.

				$V=IR$ is also imporant if we connect power
				to ground through a low resistance capacitor:
				we get high current flowing, and can burn
				out the transistor.
		\end{itemize}
\end{itemize}
\end{frame}
\fi\ENotes

\begin{frame}[fragile]
\Title{Digitizing}
\begin{itemize}
	\item Digital Signal

	\Figure{!}{1.75in}{1in}{Figs/discsig}

	\item Signal is either high (1) or low (0)
	\item Transformation could lead to intermediate values but these can be ``designed out''
\begin{multicols}{2}
		\Figure{!}{1.75in}{1in}{Figs/inbetwsig}
  \columnbreak
  \includegraphics[scale=0.25]{02-circuit/02-combinational/figures/digitize_signal.png}
\end{multicols}


		
\end{itemize}
\end{frame}

\begin{frame}[fragile]
\Title{Why Binary?}
\begin{itemize}
	\item Could have more levels...

		\Figure{!}{2.25in}{1.5in}{Figs/trisig2}

	\item	Plus, Zero, Minus (+1, 0, $-1$ --- ternary)
	\item   Two levels are simpler and just as expressive
	\item Nearly all computers today use binary
	\item Nearly all computers have similar underlying structure
\end{itemize}
\end{frame}
\BNotes\ifnum\Notes=1
ZHK Notes:
\begin{itemize}
\item the example we are seeing of trenary is balanced, there are other interpretations on  \href{https://en.wikipedia.org/wiki/Ternary_computer#:~:text=A%20ternary%20computer%2C%20also%20called,trits%2C%20instead%20of%20binary%20bits.}{wikipedia}
	\item example of trenary systems in  \href{https://eng.libretexts.org/Bookshelves/Chemical_Engineering/Phase_Relations_in_Reservoir_Engineering_(Adewumi)/05%3A_Phase_Diagrams_IV/5.04%3A_Ternary_Systems}{chemistry}
 \item examples of trenary systems in computing
 \begin{itemize}
     \item Ternary Computing to Strengthen Cybersecurity
B. Cambou, D. TelescaAdvances in Intelligent Systems and Computing2018Corpus ID: 57321934
\item Towards quantum reversible ternary coded decimal adder
M. Haghparast, R. Wille, A. T. MonfaredQuantum Information Processing2017Corpus ID: 43495632
\item also there was a desire to design MOS for trenary computing in 1984
Low power dissipation MOS ternary logic family
P. Balla, A. Antoniou1984Corpus ID: 12594482
 \end{itemize}
 \item trenary digits are called trits
 \item binary digits are called bits
 \item other important and more relevant information on \href{https://ternaryresearch.com/}{trenary research group}
 \item first modern ternary computer setun (by russia) -citation on wikipedia and trenary research group ---   
\end{itemize}
\fi\ENotes

\begin{frame}[fragile]
\STitle{Logic Blocks}
\begin{itemize}
	%\item Readings: Appendix A, sections A.1--A-3, A5, A.7-10.
    \item Inputs and outputs are 1/0

		(High/low voltage, true/false)
	\item Combinational: without memory
		\Figure{!}{1.25in}{0.75in}{Figs/block}
	\item Sequential: with memory
		\Figure{!}{1.25in}{0.75in}{Figs/sequential}
	
\end{itemize}
\end{frame}
\BNotes\ifnum\Notes=1
%\BNotes
%\begin{SMNotes}
\begin{frame}[fragile]
Instructor's Notes:
\begin{itemize}
\item At many institutions, this course is preceded by a course in
	digital logic design; at other institutions, the logic course comes
	after, or is omitted (as it is at UW, currently). Appendix A provides a
	brief overview of digital logic design, enough to understand the designs
	presented in this book. Students should read the given sections
	carefully. We will go into a little more detail, but not much more.
\item	Conceptually, there is a one-way flow of information through a
	combinational circuit (from inputs to outputs), and the output is
	computed relatively rapidly, given the inputs.
\item   A sequential circuit is named after the ``sequence'' of states
	in the storage. Remind students of the earlier statement that most
	computers have clocks and that state changes occur at clock ticks. The
	clock must tick slowly enough for the combinational circuit to be able
	to compute the next state given the current state and the inputs.
\item Next we need to discuss the building blocks of both
	combinational circuits and storage elements.
\end{itemize}
\end{frame}
\fi\ENotes
%\ENotes
%\end{SMNotes}


\begin{frame}[fragile]
\STitle{Defining a Circuit as a Boolean Function}
\begin{itemize}
	\item Truth table: specifies outputs for each possible input
combination 
\begin{multicols}{2}
 {\footnotesize
		\begin{center}\ifnum\slides=1\huge\fi
		\begin{tabular}{ccc|cc}
		X&Y&Z & F & G \\\hline
		0&0&0 & 0 & 1 \\
		0&0&1 & 1 & 1 \\
		0&1&0 & 0 & 1 \\
		0&1&1 & 0 & 1 \\
		1&0&0 & 0 & 1 \\
		1&0&1 & 1 & 1 \\
		1&1&0 & 1 & 1 \\
		1&1&1 & 1 & 0 \\
		\end{tabular}

		\end{center}	
  
  }	

  \columnbreak

  \begin{tcolorbox}[enhanced,attach boxed title to top center={yshift=-3mm,yshifttext=-1mm},
  colback=green!5!white,colframe=green!75!black,colbacktitle=green!80!black,
  title=Remember It,fonttitle=\bfseries,
  boxed title style={size=small,colframe=green!50!black} ]
   {\footnotesize  \begin{itemize}
        \item Inputs (X, Y, Z) and Outputs (F, G) are alphabetized
        \item For sub-scripted inputs or outputs, use numerical ordering based on index in scripts ($A_2, A_1, A_0$)  
        \item Permutation of input values are in increasing order (000, 001, 010, 011...)
    \end{itemize}}
\end{tcolorbox}

\end{multicols}

	\item Complete description: but can grow quickly \& gets hard to understand

\end{itemize}
\end{frame}
\BNotes\ifnum\Notes=1
\fi
\begin{frame}[fragile]
\STitle{Compact Alternative: Boolean Expressions}
\begin{itemize}
\item Define Boolean functions using Boolean expressions 
    \item With Boolean inputs and operators
    \item Some \textit{simple} Boolean operators and their truth tables:
\begin{itemize}
	% \item Signals are identified as variables (usually $A, B, C$ or $X, Y, Z$) and have binary values (0 or 1)
	\item OR ($+$) operator has result 1 \textit{if} \textbf{either} input has value 1
	\item AND ($\cdot$) operator has result 1 \textit{iff} \textbf{both} inputs have value 1
	
		{\footnotesize $A\cdot B$ often written $AB$}
	\item NOT ($\lnot$) operator has result 1 \textit{if} the input has
		value 0

		{\footnotesize $\lnot A$ usually written $\bar A$}

	\begin{center}
	\begin{tabular}{|c|c||c|c|c|c||c|c|c||c|}\cline{1-3}\cline{5-7}\cline{9-10} 
	\multicolumn{3}{|c|}{OR}& &\multicolumn{3}{|c|}{AND}& &\multicolumn{2}{|c|}{NOT}\\\cline{1-3}\cline{5-7}\cline{9-10}
% {\footnotesize \multicolumn{2}{|c|}{Boolean Variables} & Boolean Expression & &  \multicolumn{2}{|c|}{Boolean Variables} & Boolean Expression  & & Boolean Variable & Boolean Expression \\\cline{1-3}\cline{5-7}\cline{9-10}}
	A & B & A+B & & A & B & AB & & A & $\lnot$A \\\cline{1-3}\cline{5-7}\cline{9-10}
	0 & 0 & 0   & & 0 & 0 & 0  & & 0 & 1 \\\cline{1-3}\cline{5-7}\cline{9-10}
	0 & 1 & 1   & & 0 & 1 & 0  & & 1 & 0 \\\cline{1-3}\cline{5-7}\cline{9-10}
	1 & 0 & 1   & & 1 & 0 & 0 \\\cline{1-3}\cline{5-7}
	1 & 1 & 1   & & 1 & 1 & 1 \\\cline{1-3}\cline{5-7}
	\end{tabular}
	\end{center}
	% \item For truth table on previous slide, clearly $G = \Bar{XYZ}$
	% \item $F = \Bar{X}\Bar{Y}Z + X\Bar{Y}Z + XY\Bar{Z} + XYZ$ (not obvious)
\end{itemize}
\end{itemize}
\end{frame}
\BNotes\ifnum\Notes=1
\begin{frame}[fragile]
Instructor's Notes:
Much of this will be review from high-school (at least until the
new curriculum comes in, and Boolean algebra is lost)
\begin{itemize}
\item For N inputs and M outputs, there are $2^N$ lines
\item Note that it is acceptable to break the table in two to use
horizontal space more effectively.
\item Tell the class that details are important:
\begin{itemize}
	\item Inputs should be labeled and alphabetized along the top
	\item Inputs should be in increasing order (000,001,010, etc)
	\item Outputs should be labeled alphabetized
	\item For sub scripted inputs, use the order $A_2$, $A_1$, $A_0$
\end{itemize}
\end{itemize}
\end{frame}
\fi


\begin{frame}[fragile]
\STitle{Boolean Expressions Example 1}

\begin{tcolorbox}[enhanced,attach boxed title to top center={yshift=-3mm,yshifttext=-1mm},
  colback=blue!5!white,colframe=blue!75!black,colbacktitle=blue!80!black,
  title=Think About It,fonttitle=\bfseries,
  boxed title style={size=small,colframe=red!50!black} ]
To define the Boolean functions $F$ and $G$ using Boolean expressions is not always straightforward. 
{\footnotesize
  \begin{center}
		\begin{tabular}{ccc|cc}
		X&Y&Z & F & G \\\hline
		0&0&0 & 0 & 1 \\
		0&0&1 & 1 & 1 \\
		0&1&0 & 0 & 1 \\
		0&1&1 & 0 & 1 \\
		1&0&0 & 0 & 1 \\
		1&0&1 & 1 & 1 \\
		1&1&0 & 1 & 1 \\
		1&1&1 & 1 & 0 \\
		\end{tabular}
		\end{center}	
}  
    \begin{itemize}
	\item  It maybe easy to observe that $G = \Bar{X}+\Bar{Y}+\Bar{Z} \equiv \overline{XYZ}$
	\item  However, defining $F$ is not straightforward. \ifnum\Ans=1{\color{red}$F = \Bar{X}\Bar{Y}Z + X\Bar{Y}Z + XY\Bar{Z} + XYZ$}\fi
\end{itemize}

\end{tcolorbox}

\end{frame}



% \begin{frame}[fragile]
% \STitle{Boolean Expressions - Example}
% \begin{itemize}
% 	\item Truth table: specifies outputs for each possible input
% combination 
% \begin{multicols}{2}
%  {\footnotesize
% 		\begin{center}\ifnum\slides=1\huge\fi
% 		\begin{tabular}{ccc|cc}
% 		X&Y&Z & F & G \\\hline
% 		0&0&0 & 0 & 1 \\
% 		0&0&1 & 1 & 1 \\
% 		0&1&0 & 0 & 1 \\
% 		0&1&1 & 0 & 1 \\
% 		1&0&0 & 0 & 1 \\
% 		1&0&1 & 1 & 1 \\
% 		1&1&0 & 1 & 1 \\
% 		1&1&1 & 1 & 0 \\
% 		\end{tabular}

% 		\end{center}	}	

%   \columnbreak

%   \begin{tcolorbox}[enhanced,attach boxed title to top center={yshift=-3mm,yshifttext=-1mm},
%   colback=green!5!white,colframe=green!75!black,colbacktitle=green!80!black,
%   title=Remember It,fonttitle=\bfseries,
%   boxed title style={size=small,colframe=green!50!black} ]
%    {\footnotesize  \begin{itemize}
%         \item Inputs (X, Y, Z) and Outputs (F, G) are alphabetized
%         \item For sub-scripted inputs or outputs, use numerical ordering based on index in scripts ($A_2, A_1, A_0$)  
%         \item Permutation of input values are in increasing order (000, 001, 010, 011...)
%     \end{itemize}}
% \end{tcolorbox}

% \end{multicols}

% 	\item Complete description, but big and hard to understand

% \end{itemize}
% \end{frame}


\begin{frame}[fragile]
\STitle{Two-Level Representation for Boolean Expressions}
\begin{itemize}
	\item \hl{Any Boolean expression can be represented as a sum of
		products}\footnote{An alternative is product of sums that can also be useful} 
 {\footnotesize \item level 1: AND logic followed by level 2: OR logic 
 \item think about it as \textbf{OR} of \textbf{AND}s
  }
	\item Each product in the sum corresponds to a single
		line in the truth table with output value 1
  \item \textbf{If} each product contains \textbf{all the input literals}, it is called a \hl{minterm} 
    \item The sum of products (SOP) is also the sum of the minterms 
% \item Example: consider the following truth table
 \end{itemize}

% {\footnotesize
% \begin{center}

% 		\begin{tabular}{cc|c}
% 		X&Y & F \\\hline
% 		0&0 & 1  \\
% 		0&1 & 0  \\
% 		1&0 & 1 \\
% 		1&1 & 1 \\
% 		\end{tabular}
% 		\end{center}	
%   }
 


% \begin{multicols}{2}
% \begin{tcolorbox}[enhanced,attach boxed title to top center={yshift=-3mm,yshifttext=-1mm},
%   colback=red!5!white,colframe=red!75!black,colbacktitle=red!80!black,
%   title=Try this,fonttitle=\bfseries,
%   boxed title style={size=small,colframe=red!50!black} ]
%   Write the function $F$ in Boolean algebra using\\ product of sums.  
  
% \end{tcolorbox}   
% \columnbreak
\begin{tcolorbox}[enhanced,attach boxed title to top center={yshift=-3mm,yshifttext=-1mm},
  colback=red!5!white,colframe=red!75!black,colbacktitle=red!80!black,
  title=Try this,fonttitle=\bfseries,
  boxed title style={size=small,colframe=red!50!black} ]
  Define the Boolean function $F$ using sum of products form.  
{\footnotesize
\begin{center}

		\begin{tabular}{cc|c}
		X&Y & F \\\hline
		0&0 & 1  \\
		0&1 & 0  \\
		1&0 & 1 \\
		1&1 & 1 \\
		\end{tabular}
		\end{center}	
  }
 \end{tcolorbox}   
% \end{multicols}

\BNotes\ifnum\Notes=1
~% notes text
\fi\ENotes
\end{frame}

% \HideAns{
\ifnum\Ans=1
\begin{frame}[label=handout, fragile]
\STitle{Solution: Two-Level Representations}

% \begin{multicols}{2}
% \begin{tcolorbox}[enhanced,attach boxed title to top center={yshift=-3mm,yshifttext=-1mm},
%   colback=red!5!white,colframe=red!75!black,colbacktitle=red!80!black,
%   title=Try this,fonttitle=\bfseries,
%   boxed title style={size=small,colframe=red!50!black} ]
%   Write the function $F$ in Boolean algebra using\\ product of sums.  \\
%   {\color{red} $F = X+\Bar{Y}$}
% \end{tcolorbox}   
% \columnbreak
\begin{tcolorbox}[enhanced,attach boxed title to top center={yshift=-3mm,yshifttext=-1mm},
  colback=red!5!white,colframe=red!75!black,colbacktitle=red!80!black,
  title=Try this,fonttitle=\bfseries,
  boxed title style={size=small,colframe=red!50!black} ]
 Consider the following truth table:
\begin{center}
		\begin{tabular}{cc|c}
		X&Y & F \\\hline
		0&0 & 1  \\
		0&1 & 0  \\
		1&0 & 1 \\
		1&1 & 1 \\
		\end{tabular}
		\end{center}	
   Define the Boolean function $F$ using sum of products form.   \\
    {\color{red} $F = \Bar{X}\Bar{Y}+X\Bar{Y}+XY$}
  
\end{tcolorbox}   
% \end{multicols}
% \begin{itemize}
%     \item This can be simplified  \underline{by hand} or by computers.
% \end{itemize}

\BNotes\ifnum\Notes=1
\begin{itemize}
    \item For those with digital logic background they may observe that 
    \item this can be simplified  \underline{by hand} or by computers.
    \item we will see how to simplify shortly 
\end{itemize}
\fi\ENotes
\end{frame}
% }
\fi

% \begin{frame}[fragile]
% \STitle{Boolean Expressions Example 2}

% \begin{tcolorbox}[enhanced,attach boxed title to top center={yshift=-3mm,yshifttext=-1mm},
%   colback=red!5!white,colframe=red!75!black,colbacktitle=red!80!black,
%   title=Try this,fonttitle=\bfseries,
%   boxed title style={size=small,colframe=red!50!black} ]
%   Write the Boolean expression for $F$ and $G$ using Boolean algebra.  
%   \begin{center}
% 		\begin{tabular}{ccc|cc}
% 		X&Y&Z & F & G \\\hline
% 		0&0&0 & 0 & 1 \\
% 		0&0&1 & 1 & 1 \\
% 		0&1&0 & 0 & 1 \\
% 		0&1&1 & 0 & 1 \\
% 		1&0&0 & 0 & 1 \\
% 		1&0&1 & 1 & 1 \\
% 		1&1&0 & 1 & 1 \\
% 		1&1&1 & 1 & 0 \\
% 		\end{tabular}
% 		\end{center}	
% \end{tcolorbox}

% %   \begin{itemize}
% % 	\item For this truth table, it is obvious $G = \Bar{XYZ}$
% % 	\item However, it is not clear that $F = \Bar{X}\Bar{Y}Z + X\Bar{Y}Z + XY\Bar{Z} + XYZ$
% % \end{itemize}
% \end{frame}
% \BNotes\ifnum\Notes=1
% \begin{frame}[fragile]
% Instructor's Notes:
% \begin{itemize}
% 	\item For truth table on previous slide, clearly $G = \Bar{XYZ}$
% 	\item $F = \Bar{X}\Bar{Y}Z + X\Bar{Y}Z + XY\Bar{Z} + XYZ$ (not obvious)
% \end{itemize}
% Much of this will be review from high-school (at least until the
% new curriculum comes in, and Boolean algebra is lost)
% \end{frame}
% \fi\ENotes


% \begin{frame}[fragile]
% \STitle{Solution: Boolean Expressions Example 2}

% \begin{tcolorbox}[enhanced,attach boxed title to top center={yshift=-3mm,yshifttext=-1mm},
%   colback=red!5!white,colframe=red!75!black,colbacktitle=red!80!black,
%   title=Try this,fonttitle=\bfseries,
%   boxed title style={size=small,colframe=red!50!black} ]
%   Write the Boolean expressions for $F$ and $G$ using Boolean algebra.  
%   \begin{center}
% 		\begin{tabular}{ccc|cc}
% 		X&Y&Z & F & G \\\hline
% 		0&0&0 & 0 & 1 \\
% 		0&0&1 & 1 & 1 \\
% 		0&1&0 & 0 & 1 \\
% 		0&1&1 & 0 & 1 \\
% 		1&0&0 & 0 & 1 \\
% 		1&0&1 & 1 & 1 \\
% 		1&1&0 & 1 & 1 \\
% 		1&1&1 & 1 & 0 \\
% 		\end{tabular}
% 		\end{center}	
%    \begin{itemize}
% 	\item  {\color{red}$G = \Bar{X}+\Bar{Y}+\Bar{Z} \equiv \overline{XYZ}$}
% 	\item  {\color{red}$F = \Bar{X}\Bar{Y}Z + X\Bar{Y}Z + XY\Bar{Z} + XYZ$}
% \end{itemize}
% \end{tcolorbox}

 

% \end{frame}


\begin{frame}[fragile]
\Title{From Truth Table To Boolean Expressions}
\begin{itemize}
%     \item  We will use Minimal Terms (minterms)
%     \item \hl{A minterm is a product of all the input literals}
%     \item Every row in the truth table with the output value 1 can be represented as a minterm
    \item Realize that two-level representation is a systematic approach to define a Boolean function $F$
    \item Simply:
    \begin{itemize}
        \item combine the inputs in a logical AND operation for when output $F=1$, 
        \item these are minterms that partially define $F$,
        \item combine the minterms in a logical OR operation to completely define $F$.
    \end{itemize}
    % \item can be defined as SOP using two-level representation 
\end{itemize}
% \begin{tcolorbox}[enhanced,attach boxed title to top center={yshift=-3mm,yshifttext=-1mm},
%   colback=blue!5!white,colframe=blue!75!black,colbacktitle=blue!80!black,
%   title=Think About It,fonttitle=\bfseries,
%   boxed title style={size=small,colframe=red!50!black} ]
\begin{center}
\begin{tabular}{ccc|c||ccc|c}
$A$&$B$&$C$ & $F$ & $\bar A \bar B C $& $A\bar B C$&$ A B C $& $\bar A \bar B C + A\bar B C+ A B C$ \\\hline
0&0&0 & 0 & 0 & 0 & 0 & 0 \\
0&0&1 & 1 & 1 & 0 & 0 & 1 \\
0&1&0 & 0 & 0 & 0 & 0 & 0 \\
0&1&1 & 0 & 0 & 0 & 0 & 0 \\
1&0&0 & 0 & 0 & 0 & 0 & 0 \\
1&0&1 & 1 & 0 & 1 & 0 & 1 \\
1&1&0 & 0 & 0 & 0 & 0 & 0 \\
1&1&1 & 1 & 0 & 0 & 1 & 1 \\
\end{tabular}
\end{center}
% \end{tcolorbox}

\BNotes\ifnum\Notes=1
Instructor's Notes:
\begin{itemize}
\item Show how to read minimal terms (minterms) off from the 1's in the table.
\end{itemize}
\fi\ENotes
\end{frame}

% \begin{frame}[fragile]
% \Title{From Truth Table To Boolean Expression}

% \begin{tcolorbox}[enhanced,attach boxed title to top center={yshift=-3mm,yshifttext=-1mm},
%   colback=red!5!white,colframe=red!75!black,colbacktitle=red!80!black,
%   title=Try this,fonttitle=\bfseries,
%   boxed title style={size=small,colframe=red!50!black} ]
%   Write the function $F$ in Boolean algebra.  
%   \begin{center}
% 		\begin{tabular}{ccc|c}
% 		X&Y&Z & F  \\\hline
% 		0&0&0 & 1  \\
% 		0&0&1 & 0  \\
% 		0&1&0 & 0  \\
% 		0&1&1 & 0  \\
% 		1&0&0 & 0  \\
% 		1&0&1 & 0  \\
% 		1&1&0 & 0 \\
% 		1&1&1 & 1 \\
% 		\end{tabular}
% 		\end{center}	
% \end{tcolorbox}

% \BNotes\ifnum\Notes=1

% \fi\ENotes
% \end{frame}

% \begin{frame}[fragile]
% \Title{Solution: From Truth Table To Boolean Expression}

% \begin{tcolorbox}[enhanced,attach boxed title to top center={yshift=-3mm,yshifttext=-1mm},
%   colback=red!5!white,colframe=red!75!black,colbacktitle=red!80!black,
%   title=Try this,fonttitle=\bfseries,
%   boxed title style={size=small,colframe=red!50!black} ]
%   Write the function $F$ in Boolean algebra.  
%   \begin{center}
% 		\begin{tabular}{ccc|c}
% 		X&Y&Z & F  \\\hline
% 		0&0&0 & 1  \\
% 		0&0&1 & 0  \\
% 		0&1&0 & 0  \\
% 		0&1&1 & 0  \\
% 		1&0&0 & 0  \\
% 		1&0&1 & 0  \\
% 		1&1&0 & 0 \\
% 		1&1&1 & 1 \\
% 		\end{tabular}
% 		\end{center}	
%   {\color{red} $F = \Bar{X}\Bar{Y}\Bar{Z} + XYZ$}
% \end{tcolorbox}

% \BNotes\ifnum\Notes=1

% \fi\ENotes
% \end{frame}


% \begin{frame}[fragile]
% \STitle{Don't Cares in Truth Tables}
% \begin{itemize}
% 	\item Represented as X instead of 0 or 1
% 	\item When used in output, indicates that we don't care what
% 	output is for that input
% 	\item When used in input, indicates outputs are valid for all
% 	inputs created by replacing X by 0 or 1 (useful in compressing
% 		truth tables)  
% 	\item Example:

% 		\begin{center}
% 		\begin{tabular}{ccc|c}
% 		A&B&C & F\\
% 		\hline
% 		0&0&X & 0\\
% 		0&1&X & 1\\
% 		1&X&X & X\\
% 		\end{tabular}
% 		\end{center}

% \end{itemize}
% \end{frame}



%%%%%%%%XBL notes

\begin{frame}\frametitle{Don't Cares in Truth Tables}
\begin{itemize}
	\item A useful technique for compressing
		truth tables  
 \item Represent value of a literal as X instead of 0 or 1
	\item When used in output: we \hl{don't care} what
	the output is for that input combination
	\item When used in input: outputs are \hl{valid} for all
	inputs created by replacing X by 0 or 1 
 \end{itemize}
 \begin{multicols}{2}
Consider these two rows in the truth table below that give the same output value for F
\vspace{4mm}
 {\footnotesize
\begin{center}
    \begin{tabular}{ ccc|c}
    \hline 
    A & B & C & F  \\ \hline
    0 & 0 & 0 & 0  \\
    0 & 0 & 1 & 0  \\
\end{tabular}
\end{center}
}
\columnbreak
Using don't cares for \textbf{input} C simplifies the truth table to:
\vspace{3mm}
{\footnotesize
\begin{center}
    \begin{tabular}{ ccc|c}  
    \hline 
    A & B & C & F  \\ \hline
    0 & 0 & X & 0
\end{tabular}
\end{center}
}
 \end{multicols}


% We will see don't cares when talking about control.
\end{frame}

%%%%%%%% examples
\begin{frame}\frametitle{Examples - Don't Cares in Truth Tables}

 % \begin{multicols}{2}

\begin{tcolorbox}[enhanced,attach boxed title to top center={yshift=-3mm,yshifttext=-1mm},
  colback=red!5!white,colframe=red!75!black,colbacktitle=red!80!black,
  title=Try this,fonttitle=\bfseries,
  boxed title style={size=small,colframe=red!50!black} ]
  Compress this truth table using don't cares.  
 {\footnotesize
\begin{center}
    \begin{tabular}{ ccc|c}
    \hline 
    A & B & C & F  \\ \hline
    0 & 0 & 0 & 0  \\
    0 & 0 & 1 & 0  \\
    0 & 1 & 0 & 1\\
    0 & 1 & 1 & 1 \\
    \hline
     1 & 0 & 0 & 0  \\
    1 & 0 & 1 & 1  \\
    1 & 1 & 0 &0 \\
    1 & 1 & 1 & 1\\
    
\end{tabular}
\end{center}}
\end{tcolorbox}

% \columnbreak
% \begin{tcolorbox}[enhanced,attach boxed title to top center={yshift=-3mm,yshifttext=-1mm},
%   colback=red!5!white,colframe=red!75!black,colbacktitle=red!80!black,
%   title=Try this,fonttitle=\bfseries,
%   boxed title style={size=small,colframe=red!50!black} ]
%   Compress this truth table \\using don't cares in the output 

% {\footnotesize
% \begin{center}
%     \begin{tabular}{ ccc|c}  
%     \hline 
%     A & B & C & F  \\ \hline
%     0 & 0 & X & 0
% \end{tabular}
% \end{center}
% }
% \end{tcolorbox}
%  \end{multicols}

\end{frame}

\ifnum\Ans=1
\begin{frame}\frametitle{Solution - Don't Cares in Truth Tables}

 % \begin{multicols}{2}
\begin{tcolorbox}[enhanced,attach boxed title to top center={yshift=-3mm,yshifttext=-1mm},
  colback=red!5!white,colframe=red!75!black,colbacktitle=red!80!black,
  title=Try this,fonttitle=\bfseries,
  boxed title style={size=small,colframe=red!50!black} ]
  Compress this truth table using don't cares.  
 {\footnotesize
\begin{center}
    \begin{tabular}{ ccc|c}
    \hline 
    A & B & C & F  \\ \hline
    0 & 0 & X & 0  \\
    0 & 1 & X & 1\\
    \hline
     1 & X & 0 & 0  \\
    1 & X & 1 & 1  \\
    
\end{tabular}
\end{center}}
\end{tcolorbox}

% \columnbreak
% \begin{tcolorbox}[enhanced,attach boxed title to top center={yshift=-3mm,yshifttext=-1mm},
%   colback=red!5!white,colframe=red!75!black,colbacktitle=red!80!black,
%   title=Try this,fonttitle=\bfseries,
%   boxed title style={size=small,colframe=red!50!black} ]
%   Compress this truth table \\using don't cares in the output 

% {\footnotesize
% \begin{center}
%     \begin{tabular}{ ccc|c}  
%     \hline 
%     A & B & C & F  \\ \hline
%     0 & 0 & X & 0
% \end{tabular}
% \end{center}
% }
% \end{tcolorbox}
%  \end{multicols}

\end{frame}
\fi

% for comprehensive
\newpage
%-----------------------------------------------------
\begin{frame}\frametitle{Be Careful with Don't Cares}
\begin{tcolorbox}[enhanced,attach boxed title to top center={yshift=-3mm,yshifttext=-1mm},
  colback=blue!5!white,colframe=blue!75!black,colbacktitle=blue!80!black,
  title=Think About It,fonttitle=\bfseries,
  boxed title style={size=small,colframe=red!50!black} ]
Is the following truth table valid?
% \begin{table}[H]
\begin{center}
    \begin{tabular}{ ccc|c}
    \hline 
    A & B & C & F  \\ \hline
    0 & 0 & X & 0  \\
    0 & X & 1 & 1  \\
    1 & X & X & X  \\
\end{tabular}
\end{center}
% \end{table}
{\color{red} No.} It is easy to write an incorrect specification.

Here are the problems with the above truth table:
\begin{itemize}
\item 001 appears in two rows with different outputs. A truth table must have exactly one output for each input.
\item 010 is missing in the input. Each possible input must be specified exactly once.
\end{itemize}
% Also note that input A can be eliminated is unnecessary.
\end{tcolorbox}
% Students don't need to worry because the truth tables are given to you.
\end{frame}

\begin{frame}[fragile]
\STitle{Example: Boolean Expression for Truth Tables with Don't Cares}
\begin{tcolorbox}[enhanced,attach boxed title to top center={yshift=-3mm,yshifttext=-1mm},
  colback=red!5!white,colframe=red!75!black,colbacktitle=red!80!black,
  title=Try this,fonttitle=\bfseries,
  boxed title style={size=small,colframe=red!50!black} ]
  Write the Boolean expression for $F$  
		\begin{center}
		\begin{tabular}{ccc|c}
		A&B&C & F\\
		\hline
		0&0&X & 0\\
		0&1&X & 1\\
		1&X&X & X\\
		\end{tabular}
		\end{center}
\end{tcolorbox}

\end{frame}

\BNotes\ifnum\Notes=1
\begin{frame}[fragile]
Instructor's Notes:
\begin{itemize}
\item This is simply a bit of convenient shorthand we use later in specifying
	control functionality.

\item Note: It is easy, when using don't cares, to write an incorrect
  specification of a function. Consider what would happen if the
  second line read ``0,X,1,1''. Here are three problems with it: (1)
  input 001 is specified twice; (2) input 010 is not specified; (3)
  input A can be eliminated.
\end{itemize}
\end{frame}
\fi\ENotes

\ifnum\Ans=1
\begin{frame}[fragile]
\STitle{Solution: Boolean Expression for Truth Tables with Don't Cares Example}
\begin{tcolorbox}[enhanced,attach boxed title to top center={yshift=-3mm,yshifttext=-1mm},
  colback=red!5!white,colframe=red!75!black,colbacktitle=red!80!black,
  title=Try this,fonttitle=\bfseries,
  boxed title style={size=small,colframe=red!50!black} ]
  Write the Boolean expression for $F$  
		\begin{center}
		\begin{tabular}{ccc|c}
		A&B&C & F\\
		\hline
		0&0&X & 0\\
		0&1&X & 1\\
		1&X&X & X\\
		\end{tabular}
		\end{center}
  {\color{red}$F = \Bar{A}B$}
\end{tcolorbox}
\begin{itemize}
    \item Notice that boolean function $F$ is defined using a term that is not a minterm.
    \item A \hl{non-minimal} term is a product term that \textbf{does not} contain all input literals 
\end{itemize}
\end{frame}
\fi

% for comprehensive
\newpage
\begin{frame}[fragile]
\Title{Compressed Truth Tables and Non-Minimal Terms}
% \vspace*{.5truein}


 \begin{tcolorbox}[enhanced,attach boxed title to top center={yshift=-3mm,yshifttext=-1mm},
  colback=green!5!white,colframe=green!75!black,colbacktitle=green!80!black,
  title=Remember It,fonttitle=\bfseries,
  boxed title style={size=small,colframe=green!50!black} ]
  To write the Boolean function $F$  using compressed truth table, we can use minterms and, or non-minimal terms.
		\begin{center}
\begin{tabular}{ccc|c||cc|c}
$A$&$B$&$C$ & $F$ & $\bar A \bar B C $& $A C$ & $\bar A \bar B C + A C$ \\\hline
0&0&0 & 0 & 0 & 0 & 0 \\
0&0&1 & 1 & 1 & 0 & 1 \\
0&1&X & 0 & 0 & 0 & 0 \\
1&X&0 & 0 & 0 & 0 & 0 \\
1&X&1 & 1 & 0 & 1 & 1 \\
\end{tabular}
\end{center}
\end{tcolorbox}


\BNotes\ifnum\Notes=1
Instructor's Notes:
\begin{itemize}
\item This is the same F as before. 
\item We could have gotten this formula by
simplifying the one from the previous slide, or by noticing that the
term $AC$ covers two 1's, but here we just read off one term per 1 in
the function value. This leads into the next slide.
\end{itemize}
\fi\ENotes
\end{frame}

% \begin{frame}[fragile]
% \Title{Solution - Compressed Truth Tables and Non-Minimal Terms}
% \vspace*{.5truein}


% \begin{tcolorbox}[enhanced,attach boxed title to top center={yshift=-3mm,yshifttext=-1mm},
%   colback=red!5!white,colframe=red!75!black,colbacktitle=red!80!black,
%   title=Try this,fonttitle=\bfseries,
%   boxed title style={size=small,colframe=red!50!black} ]
%   Write the Boolean expression for $F$  
% 		\begin{center}
% \begin{tabular}{ccc|c|cc|c}
% $A$&$B$&$C$ & $F$ & $\bar A \bar B C $& $A C$ & $\bar A \bar B C + A C$ \\\hline
% 0&0&0 & 0 & 0 & 0 & 0 \\
% 0&0&1 & 1 & 1 & 0 & 1 \\
% 0&1&X & 0 & 0 & 0 & 0 \\
% 1&X&0 & 0 & 0 & 0 & 0 \\
% 1&X&1 & 1 & 0 & 1 & 1 \\
% \end{tabular}
% \end{center}

% \end{tcolorbox}


% \BNotes\ifnum\Notes=1
% Instructor's Notes:
% \begin{itemize}
% \item This is the same F as before. 
% \item We could have gotten this formula by
% simplifying the one from the previous slide, or by noticing that the
% term $AC$ covers two 1's, but here we just read off one term per 1 in
% the function value. This leads into the next slide.
% \end{itemize}
% \fi\ENotes
% \end{frame}

\begin{frame}[fragile]
\Title{Using Overlapping Non-Minimal Terms}

\begin{tcolorbox}[enhanced,attach boxed title to top center={yshift=-3mm,yshifttext=-1mm},
  colback=blue!5!white,colframe=blue!75!black,colbacktitle=blue!80!black,
  title=Think About It,fonttitle=\bfseries,
  boxed title style={size=small,colframe=red!50!black} ]
  To write the Boolean function $F$, we can use non-minimal terms that overlap. For example in the last row in the truth table.
	\begin{center}
\begin{tabular}{ccc|c|cc|c}
$A$&$B$&$C$ & $F$ & $A B$ & $A C$& $A B + A C$ \\\hline
0&0&0 & 0 & 0 & 0 & 0 \\
0&0&1 & 0 & 0 & 0 & 0 \\
0&1&0 & 0 & 0 & 0 & 0 \\
0&1&1 & 0 & 0 & 0 & 0 \\
1&0&0 & 0 & 0 & 0 & 0 \\
1&0&1 & 1 & 0 & 1 & 1 \\
1&1&0 & 1 & 1 & 0 & 1 \\
1&1&1 & 1 & 1 & 1 & 1 \\
\end{tabular}
\end{center}
\textbf{Can minterms overlap?} \ifnum\Ans=1{\color{red}No. There are no duplicate minterms in a truth table}\fi
\end{tcolorbox}

% \begin{center}
% \begin{tabular}{ccc|c|cc|c}
% $A$&$B$&$C$ & $F$ & $A B$ & $A C$& $A B + A C$ \\\hline
% 0&0&0 & 0 & 0 & 0 & 0 \\
% 0&0&1 & 0 & 0 & 0 & 0 \\
% 0&1&0 & 0 & 0 & 0 & 0 \\
% 0&1&1 & 0 & 0 & 0 & 0 \\
% 1&0&0 & 0 & 0 & 0 & 0 \\
% 1&0&1 & 1 & 0 & 1 & 1 \\
% 1&1&0 & 1 & 1 & 0 & 1 \\
% 1&1&1 & 1 & 1 & 1 & 1 \\
% \end{tabular}
% \end{center}

\BNotes\ifnum\Notes=1
Instructor's Notes:
\begin{itemize}
\item This is a different $F$ than we saw earlier.
\item The previous two methods would have given us $A\bar B C + A B \bar C +
A B C$ and (after compressing the truth table) $A \bar B C + AB$.
\end{itemize}
\ENotes\fi
\end{frame}




%%%%%%%%%%rearrangement

\begin{frame}[fragile]
\Title{Boolean Expression Simplification}
\begin{itemize}
    \item Boolean expressions can often be simplified either manually or by computers.
    \item Take for example the following truth table 
\begin{multicols}{2}
 \begin{center}
	{\footnotesize
 \begin{tabular}{ccc|c}
		X&Y&Z & F \\\hline
		0&0&0 & 0  \\
		0&0&1 & 1 \\
		0&1&0 & 0  \\
		0&1&1 & 0 \\
		1&0&0 & 0  \\
		1&0&1 & 1 \\
		1&1&0 & 1  \\
		1&1&1 & 1 \\
		\end{tabular}
  }
		\end{center}	
  \columnbreak

\small
% \begin{eqnarray*}
% F &=& \Bar{X}\Bar{Y}Z + X\Bar{Y}Z + XY\Bar{Z} + XYZ \\
%   &=& \Bar{Y}Z (\Bar{X}+X) + XY(\Bar{Z}+Z)\\
%   &=& \Bar{Y}Z + XY
% \end{eqnarray*}
\begin{itemize}
    \item Recall, {\footnotesize $F=\Bar{X}\Bar{Y}Z + X\Bar{Y}Z + XY\Bar{Z} + XYZ$}
    \item This is equivalent to $F = \Bar{Y}Z + XY$
\end{itemize}
\end{multicols}
 \item Consider using don't cares and observe that:
    \item in last two rows, $Z$ can be replaced with don't care, which implies, that $F$ can be partially defined by $XY$.
    \item in row 2 and row 6, when $F=1$, $X$ can be replaced with a don't care and the non-minimal term for those two rows is $\Bar{Y}Z$
    \item Therefore, $F = \Bar{Y}Z + XY$
% \item Difficult even for humans, tricky to automate
% \item Seems inherently hard to get ``simplest'' formula
% \item Is simplest formula the best for implementation?
\end{itemize}
\BNotes\ifnum\Notes=1
Instructor's Notes:
Show how the distributive law is used in reverse here, as well as the
inverse and identity laws.
\fi\ENotes
\end{frame}

\begin{frame}[fragile]
\Title{Laws of Boolean Algebra}
\begin{itemize}
\item We can use algebraic manipulation based on laws of Boolean Algebra to simplify formulas.
\end{itemize}

		\begin{center}
		\begin{tabular}{ccl}
		$\underline{\mbox{~~~Rule~~~}}$	& $\underline{\mbox{~~~Dual Rule~~~}}$ \\
		$\overline{\Bar{X}} = X$ & \\
		$X+0=X$	& $X\cdot 1 = X$& (identity)\\
		$X+1=1$	& $X\cdot 0 = 0$& (zero/one)\\
		$X+X=X$	& $XX = X$ & (absorption)\\
		$X+\Bar{X}=1$ &$X\Bar{X} = 0$ & (inverse)\\
		$X+Y=Y+X$ & $XY = YX$&(commutative)\\
		$X+(Y+Z)=\qquad$ & $X(YZ) = (XY)Z$ & (associative)\\
		$\qquad (X+Y)+Z$ &\\
		$X(Y+Z) = XY+XZ$ & $X+YZ = \qquad$ & (distributive)\\
				& $\qquad (X+Y)(X+Z)$\\
		$\overline{X+Y}=\overline{X}\cdot\overline{Y}$ & $\overline{XY} = \overline{X}+\overline{Y}$
							& (DeMorgan)
		\end{tabular}
		\end{center}

\end{frame}
\BNotes\ifnum\Notes=1
\begin{frame}[fragile]
Instructor's Notes:
\begin{itemize}
\item Go over these briefly; we're not requiring deep knowledge.
\item Exception: DeMorgan's law is useful; example: there are two ways
	to draw a NAND gate: an AND with its output complemented, and an OR
	with its inputs complemented.  Wait to tell them about two ways to
	draw NAND until NAND is introduced later, but do a truth table proof
	of DeMorgan's law.
\end{itemize}
\end{frame}
\fi\ENotes

\begin{frame}[fragile]
\Title{Boolean Expression Simplification Using Laws of Boolean Algebra}
% \begin{itemize}
% \item 
Recall, the truth table

\begin{multicols}{2}
\begin{center}
     {\footnotesize	
 \begin{tabular}{ccc|c}
		X&Y&Z & F  \\\hline
		0&0&0 & 0  \\
		0&0&1 & 1  \\
		0&1&0 & 0  \\
		0&1&1 & 0  \\
		1&0&0 & 0  \\
		1&0&1 & 1  \\
		1&1&0 & 1 \\
		1&1&1 & 1  \\
		\end{tabular}
  }
  \end{center}	
  \columnbreak  
{\footnotesize
\begin{eqnarray*}
F &=& \Bar{X}\Bar{Y}Z + X\Bar{Y}Z + XY\Bar{Z} + XYZ \\
& & \text{use distributive law}\\
  &=& \Bar{Y}Z (\Bar{X}+X) + XY(\Bar{Z}+Z)\\
  & & \text{use inverse law}\\
   &=& \Bar{Y}Z (1) + XY(1)\\
    & & \text{use identity law}\\
  &=& \Bar{Y}Z + XY
\end{eqnarray*}
}
% \columnbreak
% \begin{itemize}
% \item Difficult even for humans, tricky to automate
% \item Seems inherently hard to get ``simplest'' formula
% \item Is simplest formula the best for implementation? {\color{red}No. Depends on optimization criteria and implementation and design of system.}
% \end{itemize}
\end{multicols}
% \begin{eqnarray*}
% F &=& \Bar{X}\Bar{Y}Z + X\Bar{Y}Z + XY\Bar{Z} + XYZ \\
% & & \text{use distributive law}\\
%   &=& \Bar{Y}Z (\Bar{X}+X) + XY(\Bar{Z}+Z)\\
%   & & \text{use inverse law}\\
%    &=& \Bar{Y}Z (1) + XY(1)\\
%     & & \text{use identity law}\\
%   &=& \Bar{Y}Z + XY
% \end{eqnarray*}
% \end{multicols}
% \item Difficult even for humans, tricky to automate
% \item Seems inherently hard to get ``simplest'' formula
% \item Is simplest formula the best for implementation? {\color{red}No. Depends on fabric of implementation and system design.}
% \end{itemize}
\begin{itemize}
\item Difficult even for humans, tricky to automate
\item Seems inherently hard to get ``simplest'' Boolean expression
\item Is simplest Boolean expression the best for implementation? \ifnum\Ans=1{\color{red}No. Depends on optimization criteria and design and implementation of the system.}\fi
\end{itemize}
\BNotes\ifnum\Notes=1
Instructor's Notes:
\begin{itemize}
    \item Sometimes the goal is to optimize the number of inputs, in which case the simplified function is preferred
    \item other times, our goal is SOP, since the minterms might be shared as parts of other systems. 
    % \item other times, our implementation may not be based on the digital components we are expecting to use, e.g. our implementation fabric maybe mux
\end{itemize}
\fi\ENotes
\end{frame}


%%%%%%edits by zhk


%-----------------------------------------------------
\begin{frame}\frametitle{Example: Simplifying Boolean Expressions}
\begin{itemize}
    \item Use Don't Care or Laws of Boolean Algebra to simplify Boolean expressions. 
    \item Until the expression cannot be simplified any further.
\end{itemize}
\begin{tcolorbox}[enhanced,attach boxed title to top center={yshift=-3mm,yshifttext=-1mm},
  colback=red!5!white,colframe=red!75!black,colbacktitle=red!80!black,
  title=Try this,fonttitle=\bfseries,
  boxed title style={size=small,colframe=red!50!black} ]
  Write a simplified Boolean function for $F$. Given that the truth table for $F$ is as follows:  
\begin{figure}[H]
\centering
	{\includegraphics[width=0.6\textwidth]{Figs/min-term-ex-no-box}}
%\caption{}
\end{figure}
\end{tcolorbox}
% % \begin{figure}[H]
% % \centering
	% {\includegraphics[width=0.6\textwidth]{Figs/min-term-ex-no-box}}
% % %\caption{}
% % \end{figure}
% \vspace{-4mm}

% \begin{figure}[H]
% \centering
	% {\includegraphics[width=0.4\textwidth]{Figs/dont-care-ex}}
% %\caption{}
% \end{figure}
\end{frame}

\ifnum\Ans=1
\begin{frame}\frametitle{Solution: Simplifying Boolean Expressions}

\begin{multicols}{2}
Un-simplified Boolean expression for function $F$ using minterms
{\includegraphics[width=0.5\textwidth]{Figs/min-term-ex-no-box}}

\columnbreak

Simplified Boolean expression for function $F$ using non-minimal terms and minterms
% \centering
\begin{figure}[H]

 {\includegraphics[width=0.8\textwidth]{Figs/dont-care-ex}}
\end{figure}

\end{multicols}

\end{frame}
\fi
%-----------------------------------------------------

\ifnum\Ans=1
\begin{frame}\frametitle{Solution: Further Simplification}
\begin{itemize}
    \item The solution can be further simplified using Laws of Boolean Algebra
\item The original sum of products expression
$$F=\bar{A} \bar{B} C + A\bar{B} C + ABC$$
\item Simplified using Don't Care 
$$F = \bar{A} \bar{B} C + AC$$
%$F = \bar{A}\cdot \overline{B} \cdot C + AC$ 
\item This expression looks like it can be simplified further since $A$ and its inverse $\bar{A}$ is in both minterms


% \begin{figure}[H]
% \centering
% 	{\includegraphics[width=0.4\textwidth]{Figs/dont-care-ex-further-simplify}}
% %\caption{}
% \end{figure}
{\footnotesize
\begin{align*}
F&=\bar{A} \bar{B} C + AC & &     \text{use distributive law}\\
&=C (\bar{A} \bar{B} + A)        & &     \text{use dual rule in distributive law}\\
&=C ((\bar{A} + A)(\bar{B} + A)) & &\text{use inverse law}\\
&=C (1 (\bar{B} + A)) & &\text{use identity law} \\
&=C (\bar{B} + A) \equiv \bar{B} C + AC\\
\end{align*}}
\item In CS251, $C (\bar{B} + A)$ is \textit{considered} equivalent to $\bar{B} C + AC$
\end{itemize}
\end{frame}
\fi

%-----------------------------------------------------
% \begin{frame}\frametitle{Solution: Using Laws of Boolean Algebra}

% Indeed,
% \begin{align*}
% F&=\bar{A} \bar{B} C + A\bar{B} C + ABC\\
%  &=\bar{A} \bar{B} C + A\bar{B} C + A\bar{B} C + ABC\\
%  &=(\bar{A}+A) \bar{B} C + A(\bar{B} + B)C\\
%  &=\bar{B} C + AC\\
% \end{align*}
% Thus, if $\bar{X}$ appears in one minterm, and $X$ appears in another, try to simplify further. 
% \end{frame}



\begin{frame}[fragile]
\STitle{Conclusion}
 \underline{\textbf{Lecture Summary}}
 \begin{itemize}
 % \item Combinational Logic Gates
 \item Defining digital functions 
 \item Truth Tables
 \item Boolean expressions
 \item Sum of Product (SOP) 
 \item Simplifying Boolean expressions using
 \begin{itemize}
     \item Laws of Boolean Algebra 
     \item Don't Care
     \item Nonminimal terms
 \end{itemize} 
 \end{itemize}
 \underline{\textbf{Assigned Textbook Readings}}
\begin{itemize}
     \item Sections A.1 and A.2 in Appendix A. 
     \end{itemize}
    \underline{\textbf{Next Steps}}
    \begin{itemize}
    \item \textbf{Ask} questions in the next tutorial or office hours.
 \end{itemize}

 \end{frame}





