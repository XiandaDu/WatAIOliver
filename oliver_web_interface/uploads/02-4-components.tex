\setlength{\columnseprule}{1pt}
\def\columnseprulecolor{\color{blue}}


\begin{frame}[fragile]
\STitle{Implementing Boolean Functions: PLAs}
\begin{itemize}
	\item A PLA (Programmable Logic Array) implements a two-level
	function
	
%	\Figure{!}{4in}{3in}{PHFigs/B05}
	\Figure{!}{4in}{2in}{PHALL/B05}
	\item Typically a PLA has fixed number of product terms and outputs available
 	% \item Typically a PLA has fixed number of inputs and outputs available
\end{itemize}
\end{frame}
\BNotes\ifnum\Notes=1
\begin{frame}[fragile]
Instructor's Notes:
\begin{itemize}
	\item Mention that both AND and OR arrays are programmable.
	\item Internally, complemented inputs are available
	\item Draw simple example (show AND and OR gates, but not fuses,
		just draw connections) on blackboard
\end{itemize}
\end{frame}
\fi\ENotes

\begin{frame}{Example: 3 input 2 output PLA}
\resizebox{8cm}{8cm}{
    \input 02-circuit/02-combinational/pla1_image
    }
\end{frame}

\begin{frame}{Example: \hl{Programmed} 3 input 2 output PLA}
    \resizebox{12cm}{8cm}{
    % \input 02-circuit/02-combinational/pla2_image
    \includegraphics[width=\textwidth]{02-circuit/02-combinational/pla2_image.png}
    }
\end{frame}

% \begin{frame}[fragile]
% \STitle{Implementing Boolean Functions: ROMs}
% \begin{itemize}
% 	\Figure{!}{1.25in}{.75in}{Figs/ROM}
% 	\item Can think of ROM as table of $2^n$ $m$-bit words
% 	\item Can think of ROM as implementing $m$ one-bit functions
% 	of $n$ variables
% 	\item Internally, consists of a decoder plus an OR gate for
% 	each output
% 	\item Types of ROM: PROM, EPROM, EEPROM
% %	\item ROMs and PLAs closely related
% 	\item PLAs - simplified ROM

% 		Less hardware, but less flexible
% \end{itemize}
% \end{frame}
% \BNotes\ifnum\Notes=1
% \begin{frame}[fragile]
% Instructor's Notes:
% \begin{itemize}
% \item We'll be talking about using ROMs or PLAs in our discussion of
% implementing finite-state machines, but otherwise we do not use them
% much in this course. You can talk about their more general uses if you
% wish. 
% \item Note that the fixed number of product terms in a PLA means that
% it cannot implement all possible collections of $m$ one-bit functions;
% some would require too many product terms. In contrast, a ROM can implement
% any collection.
% \bigskip
% \item {\bf The real point}: there is non-volatile memory in the computer (ie, that doesn't forget when power is turned off).  This memory is read many times and written rarely.  In older technologies, this memory was written once, when it was created, and then installed in the computer.  But newer ones can be written after having been installed in the computer.  Flashing the bios is an example of this.
% \end{itemize}
% \end{frame}
% \fi\ENotes



% \begin{frame}[fragile]
% \STitle{Conclusion}
%  \underline{\textbf{Lecture Summary}}
%  \begin{itemize}
%  \item PLA
%  \item ROM
%  \end{itemize}
%  \underline{\textbf{Assigned Textbook Readings}}
% \begin{itemize}
%      \item \textbf{Read} Sections A.3
%      %--A.12 is not in the details relevant to the scope of the course
%      \end{itemize}
%     \underline{\textbf{Next Steps}}
%     \begin{itemize}
%     \item \textbf{Ask} questions in the next tutorial or office hours.
%  \end{itemize}

% \end{frame}


% \begin{frame}{Additional Slides}
%      Remaining slides are additional notes for your information.
%  \end{frame}

 % \begin{frame}{Example: Using ROM for Logic}
 %     \includegraphics[scale=0.2]{02-circuit/02-combinational/rom-logic.png}
 % \end{frame}