\setlength{\columnseprule}{1pt}
\def\columnseprulecolor{\color{blue}}


% \begin{frame}[fragile]
% \STitle{Implementing Boolean Functions: PLAs}
% \begin{itemize}
% 	\item A PLA (Programmable Logic Array) implements a two-level
% 	function
	
% %	\Figure{!}{4in}{3in}{PHFigs/B05}
% 	\Figure{!}{4in}{2in}{PHALL/B05}
% 	\item Typically a PLA has fixed number of product terms and outputs available
% \end{itemize}
% \end{frame}
% \BNotes\ifnum\Notes=1
% \begin{frame}[fragile]
% Instructor's Notes:
% \begin{itemize}
% 	\item Mention that both AND and OR arrays are programmable.
% 	\item Internally, complemented inputs are available
% 	\item Draw simple example (show AND and OR gates, but not fuses,
% 		just draw connections) on blackboard
% \end{itemize}
% \end{frame}
% \fi\ENotes


% \begin{frame}[fragile]
% \STitle{Implementing Boolean Functions: ROMs}
% \begin{itemize}
% 	\Figure{!}{1.25in}{.75in}{Figs/ROM}
% 	\item Can think of ROM as table of $2^n$ $m$-bit words
% 	\item Can think of ROM as implementing $m$ one-bit functions
% 	of $n$ variables
% 	\item Internally, consists of a decoder plus an OR gate for
% 	each output
% 	\item Types of ROM: PROM, EPROM, EEPROM
% %	\item ROMs and PLAs closely related
% 	\item PLAs - simplified ROM

% 		Less hardware, but less flexible
% \end{itemize}
% \end{frame}
% \BNotes\ifnum\Notes=1
% \begin{frame}[fragile]
% Instructor's Notes:
% \begin{itemize}
% \item We'll be talking about using ROMs or PLAs in our discussion of
% implementing finite-state machines, but otherwise we do not use them
% much in this course. You can talk about their more general uses if you
% wish. 
% \item Note that the fixed number of product terms in a PLA means that
% it cannot implement all possible collections of $m$ one-bit functions;
% some would require too many product terms. In contrast, a ROM can implement
% any collection.
% \bigskip
% \item {\bf The real point}: there is non-volatile memory in the computer (ie, that doesn't forget when power is turned off).  This memory is read many times and written rarely.  In older techologies, this memory was written once, when it was created, and then installed in the computer.  But newer ones can be written after having been installed in the computer.  Flashing the bios is an example of this.
% \end{itemize}
% \end{frame}
% \fi\ENotes

\begin{frame}[fragile]
\STitle{Clocks and Sequential Circuits}
\begin{itemize}
\item A sequential circuit has a storage (state) element 
\Figure{!}{1in}{.5in}{02-circuit/02-sequential/figures/seq.png}
% \includegraphics[width=1.5in]{figures/seq.png}
	% \item Two types of sequential circuits
	\item Synchronous: has a clock and storage (memory) changes only at discrete points in time
		\Figure{!}{1in}{.5in}{Figs/sequential2}
  
		Clock pulse:
		\Figure{!}{1in}{.5in}{Figs/clock2}
  	Easier to analyze, tends to be more stable
	\item Asynchronous: no clock, potentially faster and less power-hungry, but harder to design and analyze

\end{itemize}
\end{frame}

\BNotes\ifnum\Notes=1
\begin{frame}[fragile]
Instructor's Notes:
\begin{itemize}
\item Point out rising edge, falling edge, clock period on diagram
\item Emphasize that the circuits we use will all be synchronous
	except for the latch (next slide).
\end{itemize}
\end{frame}
\fi\ENotes

\begin{frame}[fragile]
\Title{SR Latch with NOR gates}
\begin{itemize}

	\item SR Latch with NOR gates
\begin{multicols}{2}
		\Figure{!}{2in}{1in}{Figs/sr}
  (bad circuit drawing style)
\columnbreak
    \Figure{!}{2in}{1in}{Figs/SRNOR}
    (good circuit drawing style)
    
\end{multicols}
 \end{itemize}
 \begin{tcolorbox}[enhanced,attach boxed title to top center={yshift=-3mm,yshifttext=-1mm},
  colback=blue!5!white,colframe=blue!75!black,colbacktitle=blue!80!black,
  title=Think About It,fonttitle=\bfseries,
  boxed title style={size=small,colframe=red!50!black} ]         		
	Can you derive a truth table for outputs $Q$ and $\bar{Q}$?
 % Problem: Behavior depends on {\em previous} values of
		% $Q$ and $\Bar{Q}$ when $S=R=0$
	% \item Need to add time, talk about transitions
	% 	\begin{center}
	% 	\begin{tabular}{c|c}
	% 	$S,R$ transition & $Q$, $\Bar{Q}$ transition\\\hline
	% 	1,0 $\rightarrow$ 0,0 &  ~~~~~ $\rightarrow$ ~~~~~  \\
	% 	0,1 $\rightarrow$ 0,0 &  ~~~~~ $\rightarrow$ ~~~~~  \\
	% 	1,1 $\rightarrow$ 0,0 &  ~~~~~ $\rightarrow$ ~~~~~  \\
	% 	\end{tabular}
	% 	\end{center}
 \end{tcolorbox}

\end{frame}

\BNotes\ifnum\Notes=1
\begin{frame}[fragile]
Instructor's Notes:
\begin{itemize}
\item Make sure you point out that this is not drawn according to our
style guidelines; we use it because it's straight from the
book. Here's how it should be drawn.

		\Figure{!}{2in}{1in}{Figs/SRNOR}


\item Analyzing this and subsequent circuits can be very tricky.  Be
sure to read the book and practice this material before giving the
lecture. Key to this slide is that when $S=R=0$, we don't have enough
information to know what the outputs $Q$ and $\Bar{Q}$ are, but when
one of $S$ or $R$ is one, we do. Work this out on the board.
\end{itemize}
\end{frame}

\begin{frame}[fragile]
Instructor's Notes Continued:
\begin{itemize}
\item The last transition is a problem; it's unstable (the
outputs oscillate 00,11,00, and eventually one will change first and
there will be a transition through 01 or 10 to 00). This behaviour
has to be ``designed out'' (eliminate the possibility through
additional circuitry).

\item The names $Q$ and $\bar{Q}$ aren't right for inputs 1,1.
	However, we will design this input out, so for all ``real''
	inputs, the names work.

\item Also note that this latch design is typical of a certain class
	of memory element, and thus we often have both $Q$ and $\bar{Q}$
	available as inputs.
\end{itemize}
\end{frame}
\fi\ENotes

\begin{frame}[fragile]
\Title{Functional Description of SR Latch}
\begin{itemize}
\item Truth table for SR latch
		\begin{center}
		\begin{tabular}{cc|ccl}
		$S$ & $R$ & $Q$ & $\Bar{Q}$ &\\\cline{1-4}
		0 & 0 & $Q$ & $\Bar{Q}$ & Latch state (no change) \\
		0 & 1 & 0 & 1 & Reset state\\
		1 & 0 & 1 & 0 & Set state\\
		1 & 1 & ? & ? & Undefined
		\end{tabular}
		\end{center}


	\item Advantages:
	\begin{itemize}
		\item Can ``remember'' value
		\item Natural ``reset'' and ``set'' signals
	
			(SR=01 is ``reset'' Q to 0, SR=10 is ``set'' Q to
			1)
	\end{itemize}
	\item Disadvantages:
	\begin{itemize}
		\item SR=11 input has to be avoided
		\item No notion of a clock or change at discrete
			points in time yet
	\end{itemize}
\end{itemize}
\end{frame}

\BNotes\ifnum\Notes=1
\begin{frame}[fragile]
Instructor's Notes:
We fix both the cons by adding a pair of AND gates together with the
clock signal; when the clock is off, the AND gates transmit 00, the
latch state of the SR latch. The other input to the AND gates is D or
not-D respectively, and that provides the value that the latch remembers.
\end{frame}
\fi\ENotes

%%%%%%%from XBL slides, added by ZHK W23

\begin{frame}\frametitle{Improve the SR Latch}
Idea: Use AND gates with clock. When clock is off (signal is 0), the AND gate outputs 0. The other input to this AND gate is called $D$, which is the ``data'' we want to remember. This circuit is called the \textbf{D-Latch}.

\begin{figure}[H]
\centering
	{\includegraphics[scale=0.4]{Figs/d-latch}}
\caption{D Latch}
\end{figure}

\end{frame}

\begin{frame}[fragile]
\STitle{The D Latch}
\begin{tcolorbox}[enhanced,attach boxed title to top center={yshift=-3mm,yshifttext=-1mm},
  colback=blue!5!white,colframe=blue!75!black,colbacktitle=blue!80!black,
  title=Think About It,fonttitle=\bfseries,
  boxed title style={size=small,colframe=red!50!black} ]

		\Figure{!}{1.5in}{1in}{Figs/d-latch}

Complete the truth table for the D latch:
		\begin{center}
		\begin{tabular}{cc|l}
		$C$ & $D$ & Next state of $Q$\\\hline
	0 & 0 & \\	
  0 & 1 & \\
		1 & 0 &\\
		1 & 1 &\\
		\end{tabular}
		\end{center}
  \end{tcolorbox}
\end{frame}

\ifnum\Ans=1
\begin{frame}[fragile]
\STitle{Solution: The D Latch Truth Table}

		\Figure{!}{1in}{1in}{Figs/d-latch}

Complete the truth table for the D latch:
		\begin{center}
		\begin{tabular}{cc|l}
		$C$ & $D$ & Next state of $Q$\\\hline
		0 & X & No change\\
		1 & 0 & $Q=0$ (Reset)\\
		1 & 1 & $Q=1$ (Set)
		\end{tabular}
		\end{center}

\end{frame}
\fi

% for comprehensive
\newpage

\begin{frame}[fragile]
\STitle{The D Latch Signal}
\begin{tcolorbox}[enhanced,attach boxed title to top center={yshift=-3mm,yshifttext=-1mm},
  colback=blue!5!white,colframe=blue!75!black,colbacktitle=blue!80!black,
  title=Think About It,fonttitle=\bfseries,
  boxed title style={size=small,colframe=red!50!black} ]
\Figure{!}{1in}{1in}{Figs/d-latch}
Trace the output signal $Q$, given the input $C$ and $D$.
		\Figure{!}{1.5in}{1in}{Figs/dlatch}
  \end{tcolorbox}
\end{frame}

\ifnum\Ans=1
\begin{frame}\frametitle{Solution: D Latch Signal}

While $C=1$, then $Q=D$. While $C=0$, $Q$ is whatever it was before.
\Figure{!}{1in}{1in}{Figs/d-latch}
\begin{figure}[H]
\centering
	{\includegraphics[scale=0.2]{Figs/d-latch-signal-drawn}}
\caption{D latch signal.}
\end{figure}
\end{frame}
\fi

% for comprehensive
\newpage

%-----------------------------------------------------
\begin{frame}\frametitle{D Latch Problem}

\begin{tcolorbox}[enhanced,attach boxed title to top center={yshift=-3mm,yshifttext=-1mm},
  colback=blue!5!white,colframe=blue!75!black,colbacktitle=blue!80!black,
  title=Think About It,fonttitle=\bfseries,
  boxed title style={size=small,colframe=red!50!black} ]
While $C=1$, $D$ can change $Q$. 
\begin{multicols}{2}
{\includegraphics[scale=0.19]{Figs/d-latch-signal-drawn}}

\columnbreak

{\includegraphics[scale=0.19]{Figs/d-latch-signal-change-drawn}}
Is this what we want? \textbf{No}.
\end{multicols}
\end{tcolorbox}
\end{frame}





\BNotes\ifnum\Notes=1
\begin{frame}[fragile]
Instructor's Notes:
\begin{itemize}
\item Notice the {\it gate propagation delay time} (from C to Q).
\item Point out that the "1,1" input to the SR-latch can't occur.
\item 
The D latch is transparent; that is, when the clock is on, the output
reflects any change in the input. 

Demonstrate this by drawing on the
slide, showing a variation in the D input and the corresponding
variation in the Q output. 

Thus we still don't have the situation we
want, where change only occurs at discrete times.
\item The "clock" (C input) is really acting as a control, not a clock
\end{itemize}
\end{frame}
\fi\ENotes

\begin{frame}[fragile]
\STitle{The D Flip-Flop}
\begin{itemize}
\item We want state to be changed only at discrete points in time; a
master-slave design achieves this.

		%\PHFigure{!}{1.5in}{1in}{PHALL/B15}{Simlar to Figure A.8.6}
		\Figure{!}{1.5in}{1in}{Figs/d-flip-flop}
% \item Graphical example:
% \vspace{-4mm}
% 		\Figure{!}{3.25in}{1.5in}{Figs/dff}
\end{itemize}
\end{frame}

% for comprehensive
\newpage

\begin{frame}{Example: D Flip-Flop}
\begin{tcolorbox}[enhanced,attach boxed title to top center={yshift=-3mm,yshifttext=-1mm},
  colback=red!5!white,colframe=red!75!black,colbacktitle=red!80!black,
  title=Try this,fonttitle=\bfseries,
  boxed title style={size=small,colframe=red!50!black} ]
  For the circuit illsutrated:
  \Figure{!}{0.5in}{0.5in}{Figs/d-flip-flop}
Draw the resulting traces for $Q_I$ and $Q_E$, given the trace for inputs $C$ and $D$.
\vspace{-4mm}
		\Figure{!}{3in}{1.5in}{Figs/dff}
\end{tcolorbox}    
\end{frame}
\BNotes\ifnum\Notes=1
\begin{frame}[fragile]
Instructor's Notes:\\
Explain in words how 
\begin{itemize}
\item the master latch (leftmost) reflects the D input
	while the clock signal is high, but the slave latch (receiving an
	inverted clock signal) does not; 
\item at the falling edge of the clock, the
	master is turned off, but the slave is turned on and reflects the
	value stored in the master (the D input just before the falling
	edge). 
\item Since the master is off, this value will not change through the
	period where the clock is high. 
\item At the rising edge of the clock, the
	master turns on, but the slave is off again, and so the slave
	``remembers'' the value it received at the falling edge.
\end{itemize}
\end{frame}


\begin{frame}[fragile]
Instructor's Notes Continued:
\begin{itemize}
\item $Q_I$ is the output of the ``internal'' latch; 
	$Q_E$ is the output of the ``external'' latch (ie, the output
	of the flip-flop).  You need to label the subscripts in the
	figure.
\item Illustrate the behaviour by drawing lines from the D input to
	the $Q_I$ line while the clock is high to indicate that the
	input is copied to $Q_I$ during this time, and lines from $Q_I$
	to $Q_E$ to while the clock is high to indicate the copying
	the occurs during this time.
\item Draw up-down spikes on the D-input and show how it changes $Q_I$ and
	$Q_E$.
\end{itemize}
\end{frame}
\fi\ENotes

\ifnum\Ans=1
\begin{frame}\frametitle{Solution: D Flip-Flop}
\begin{itemize}
\item While $C=1$, $\overline{C}=0$, $Q_I$ can change with $D$, but $Q_E$ ignores $Q_I$.

\item While $C=0$, $\overline{C}=1$, $Q_I$ remembers its previous state and sends that to $Q_E$. \begin{itemize}
\item Since $C=0$, $Q_I$ cannot change, $ \Rightarrow Q_E$ \textbf{cannot} change.
\end{itemize}
\end{itemize}
Therefore, $Q_E$ \textbf{only} changes when $C$ changes from $C=1$ to $C=0$.

\begin{figure}[H]
\centering
	{\includegraphics[scale=0.2]{Figs/d-flipflop-signal-draw}}
\caption{D flip-flop signal.}
\end{figure}
\end{frame}
\fi
%-----------------------------------------------------
\begin{frame}\frametitle{Edge-triggered D Flip-Flop}
\begin{tcolorbox}[enhanced,attach boxed title to top center={yshift=-3mm,yshifttext=-1mm},
  colback=blue!5!white,colframe=blue!75!black,colbacktitle=blue!80!black,
  title=Think About It,fonttitle=\bfseries,
  boxed title style={size=small,colframe=red!50!black} ]
\begin{multicols}{2}
\begin{center}
    \includegraphics[width=2in]{Figs/d-flip-flop}
 \end{center}   
\small This D flip-flop \textbf{only} passes the value of $D$ to $Q_E$ on the \textbf{falling-edge} of the \textbf{C} (clock).
\begin{center}
    {\includegraphics[width=1.3in]{Figs/d-flipflop-signal-draw}}
\end{center}

\columnbreak

\begin{center}    
      \includegraphics[width=2.3in]{02-circuit/02-sequential/figures/dff_rising.png}
      \end{center}
\small When inverted clock is input to the D flip-flop, it \textbf{only} passes the value of $D$ to $Q_E$ on the \textbf{rising-edge} of the \textbf{C} (clock).
\begin{center}
    {\includegraphics[width=1.3in]{02-circuit/02-sequential/figures/rising-dff-trace.png}}
\end{center}
\end{multicols}
\end{tcolorbox}
\end{frame}
%----------
% \begin{frame}\frametitle{D Flip-Flop Signal}
% \begin{itemize}
% \item While $C=1$, $\overline{C}=0$, $Q_I$ can change with $D$, but $Q_E$ ignores $Q_I$.

% \item While $C=0$, $\overline{C}=1$, $Q_I$ remembers its previous state and sends that to $Q_E$.
% \begin{itemize}
% \item $Q_I$ cannot change, therefore, $Q_E$ also cannot change.
% \end{itemize}
% \end{itemize}
% Therefore, $Q_E$ \textbf{only} changes when $C$ changes from $C=1$ to $C=0$.

% \begin{figure}[H]
% \centering
% 	{\includegraphics[scale=0.3]{Figs/d-flipflop}}

% \caption{D flip-flop.}
% \end{figure}
% \end{frame}

% %-----------------------------------------------------
% \begin{frame}\frametitle{D Flip-Flop Signal}
% \begin{itemize}
% \item While $C=1$, $\overline{C}=0$, $Q_I$ can change with $D$, but $Q_E$ ignores $Q_I$.

% \item While $C=0$, $\overline{C}=1$, $Q_I$ remembers its previous state and sends that to $Q_E$. \begin{itemize}
% \item $Q_I$ cannot change, therefore, $Q_E$ also cannot change.
% \end{itemize}
% \end{itemize}
% Therefore, $Q_E$ \textbf{only} changes when $C$ changes from $C=1$ to $C=0$.

% \begin{figure}[H]
% \centering
% 	{\includegraphics[scale=0.2]{Figs/d-flipflop-signal-draw}}
% \caption{D flip-flop signal.}
% \end{figure}
% \end{frame}

%-----------------------------------------------------
% \begin{frame}\frametitle{Another D Flip-Flop}
% Repeat the above analysis for a D flip flop where the NOT gate applied to the first clock.
% \end{frame}
%-----------------------------------------------------

\begin{frame}[fragile]
\STitle{Conclusion}
 \underline{\textbf{Lecture Summary}}
 \begin{itemize}
 \item Sequential Logic and Clocks
 \item SR Latch
 \item D Latch
 \item D Flip Flop (DFF) and Edge Triggered DFFs
 \end{itemize}
 \underline{\textbf{Assigned Textbook Readings}}
\begin{itemize}
     \item \textbf{Read} Section 4.2, Appendix A.7 and A.8
     \end{itemize}
    \underline{\textbf{Next Steps}}
    \begin{itemize}
     \item \textbf{Review} tracing signals through D Flip Flops. 
% \begin{itemize}
%     \item Start thinking about the questions in A2
% \end{itemize}
\item \textbf{Attempt} questions in next week's tutorial. 
    \item \textbf{Ask} questions in office hours or the next tutorial.
 \end{itemize}

\end{frame}

\begin{frame}{Additional Slides}
     Remaining slides are additional notes for your information.
 \end{frame}

% for comprehensive
\newpage

\begin{frame}\frametitle{D Flip-Flop Transistor Count}
One D flip-flop uses two D latches and one NOT gate. 
\begin{multicols}{2}
\begin{itemize}
    \item One D latch has the following transistors:
    \begin{itemize}
        \item NOR gate: 4 transistors (2)
        \item AND gate: 6 transistors (2)
        \item NOT gate: 2 transistors (1)
    \end{itemize}
    \item Total transistors in a D Latch: $(4+ 6) \times 2 + 2 = 22$.
\end{itemize}

\columnbreak

\begin{figure}[H]
\centering
	{\includegraphics[scale=0.3]{Figs/d-latch}}
\caption{D latch.}
\end{figure}

\end{multicols}
Therefore, in a D flip-flop, total transistors: $2 \times 22 + 2 = 46$
\end{frame}


% %%%%%%%%%%%%%%%W23

% \begin{frame}{Memory}
% \title{Memory Elements:}
% \begin{itemize}
%     \item ROM
%     \item Registers
%     \item SRAM
%     \item DRAM
% \end{itemize}
    
% \end{frame}


% % \begin{frame}[fragile]
% % \STitle{Implementing Boolean Functions: PLAs}
% % \begin{itemize}
% % 	\item A PLA (Programmable Logic Array) implements a two-level
% % 	function
	
% % %	\Figure{!}{4in}{3in}{PHFigs/B05}
% % 	\Figure{!}{4in}{2in}{PHALL/B05}
% % 	\item Typically a PLA has fixed number of product terms and outputs available
% % \end{itemize}
% % \end{frame}
% % \BNotes\ifnum\Notes=1
% % \begin{frame}[fragile]
% % Instructor's Notes:
% % \begin{itemize}
% % 	\item Mention that both AND and OR arrays are programmable.
% % 	\item Internally, complemented inputs are available
% % 	\item Draw simple example (show AND and OR gates, but not fuses,
% % 		just draw connections) on blackboard
% % \end{itemize}
% % \end{frame}
% % \fi\ENotes


% \begin{frame}[fragile]
% \STitle{Implementing Boolean Functions: ROMs}
% \begin{itemize}
% 	\Figure{!}{1.25in}{.75in}{Figs/ROM}
% 	\item Can think of ROM as table of $2^n$ $m$-bit words
% 	\item Can think of ROM as implementing $m$ one-bit functions
% 	of $n$ variables
% 	\item Internally, consists of a decoder plus an OR gate for
% 	each output
% 	\item Types of ROM: PROM, EPROM, EEPROM
% %	\item ROMs and PLAs closely related
% 	\item PLAs - simplified ROM

% 		Less hardware, but less flexible
% \end{itemize}
% \end{frame}
% \BNotes\ifnum\Notes=1
% \begin{frame}[fragile]
% Instructor's Notes:
% \begin{itemize}
% \item We'll be talking about using ROMs or PLAs in our discussion of
% implementing finite-state machines, but otherwise we do not use them
% much in this course. You can talk about their more general uses if you
% wish. 
% \item Note that the fixed number of product terms in a PLA means that
% it cannot implement all possible collections of $m$ one-bit functions;
% some would require too many product terms. In contrast, a ROM can implement
% any collection.
% \bigskip
% \item {\bf The real point}: there is non-volatile memory in the computer (ie, that doesn't forget when power is turned off).  This memory is read many times and written rarely.  In older techologies, this memory was written once, when it was created, and then installed in the computer.  But newer ones can be written after having been installed in the computer.  Flashing the bios is an example of this.
% \end{itemize}
% \end{frame}
% \fi\ENotes


% \begin{frame}[fragile]
% \STitle{Registers and Register Files}
% \begin{itemize}
% \item Register: an array of flip-flops (64 for a double word register)
% \item Register file: a way of organizing registers

% 	%	\PHFigure{!}{4in}{2in}{PHALL/B18}{Similar to Figure A.8.7}
% 	\Figure{!}{4in}{2in}{Figs/regfile}
% \end{itemize}
% \end{frame}

% \BNotes\ifnum\Notes=1
% \begin{frame}[fragile]
% Instructor's Notes:
% \begin{itemize}
% \item This register file has 32-registers with 64-bits each
% \item Don't forget to point out that the lines on this diagram are not one
% 	bit wide. The read register number lines are log n bits wide, and the read
% 	data lines are 64 bits wide.
% \end{itemize}
% \end{frame}
% \fi\ENotes

% \begin{frame}[fragile]
% \Title{Read/Write Logic for Register File}
% 		\TwoFigure{!}{2.75in}{1.5in}{Figs/regwrite}{Figs/regread}
% \end{frame}

% \BNotes\ifnum\Notes=1
% \begin{frame}[fragile]
% Instructor's Notes:
% \begin{itemize}
% 	\item Not shown in the figure: The clock.
% 	\item Also: the figure on the left is wrong: the register file
% 		should have registers numbered from 0 to $2^n-1$.  The
% 		figure on the right is correct.  The first figure is from
% 		the textbook.  The second one is one we created; the
% 		corresponding figure in the book incorrectly numbers
% 		the outputs of the decoder and the numbers of the
% 		registers.  These figures are wrong in all five editions
% 		of the book.
% 	\item The next slide merges these
% \end{itemize}
% \end{frame}
% \fi\ENotes

% \begin{frame}[fragile]
% \Title{Read/Write Logic for Register File--Merged}
% \ifnum\slides=1
% 		\TwoFigure{!}{2.75in}{2.5in}{Figs/regwrite}{Figs/regread}
% \fi

% 	\Figure{!}{3.5in}{2.5in}{Figs/regRW}
% \BNotes\ifnum\Notes=1
% Instructor's Notes:
% \begin{itemize}
% 	\item This is the merged version of the read/write logic
% 		for the register file.
% \end{itemize}
% \fi\ENotes
% \end{frame}

% \begin{frame}[fragile]
% \STitle{Random Access Memories}
% \begin{itemize}
% \item Static random access memories (SRAM) use D latches

% 	\PHFigure{!}{3in}{1.5in}{PH5Figs/FB-9-1}{Similar to Figure A.9.1}
	
% \item Register file idea won't scale up; decoder and multiplexors too big
% \item Fix multiplexor problem by using three-state buffers
% \item Fix decoder problem by using two-level decoding
% \item This type of memory is {\bf not} clocked
% \end{itemize}
% \end{frame}
% \BNotes\ifnum\Notes=1
% \begin{frame}[fragile]
% Instructor's Notes:
% \begin{itemize}
% 	\item Return to Decode slide and count transistors.

% 	You should get $2^n$ $n$-input ANDs.  That gives $2n$ transistors
% 	for each {\em word} of memory.

% 	\item Return to multiplexor slide and count the transistors.

% 	You should get $2\times 2^n$ for the $2^n$-input OR

% 	And you should have $2^n$ $n+1$-input ANDs.

% 	Thus, each {\em bit} has $2n+2$ transistors in the ANDs (and 2 more for the OR)

% \end{itemize}
% \end{frame}
% \fi\ENotes

% % \begin{frame}[fragile]
% % \Title{Three-state buffer Gate}
% % \begin{itemize}
% % 	\item Has three outputs
% % 		0, 1, and {\em floating} (connected to neither power or ground)
% % 		\Figure{!}{2.5in}{1.5in}{NewTransFigs/tgate}
% % 	\item $C=1$, then 
% % 	\begin{itemize}
% % 		\item NMOS gate passes 0 well 
% % 		\item $\bar{C}=0$ and PMOS gate passes 1 well
% % 	\end{itemize}
% % 	\item $C=0$, then $\bar{C}=1$ and both transistors are off (output is floating).
% % \end{itemize}
% % \end{frame}
% % \BNotes\ifnum\Notes=1
% % \begin{frame}[fragile]
% % Instructor's Notes:
% % \begin{itemize}
% % \item The floating state is like a physical disconnection
% % \item It cannot be propagated through other gates
% % \item One caution with three-state buffers: note that when C is high, then
% % 	the output IS the input.  Ie, with a CMOS NAND, the output is connected
% % 	either to power or ground.  With a CMOS Three-state, when C is high,
% % 	the output is wired to the input.  This means we don't get the power
% % 	boost that is normally associated with passing the signal through a
% % 	gate (i.e., the three-state gate doesn't have a fan-out count, since
% % 	its output is really the output of another gate).
% % \end{itemize}
% % \end{frame}
% % \fi\ENotes


% % \begin{frame}[fragile]
% % \STitle{Using Three-State Buffers}
% % 		\Figure{!}{1.25in}{1in}{Figs/tristate}
% % \begin{itemize}
% % \item High-impedance outputs can be ``tied together'' without problems
% % \item Normally, do not tie output lines together
% % 		\Figure{!}{1.75in}{1in}{Figs/wiredORa}
% % \end{itemize}
% % \end{frame}

% % \BNotes\ifnum\Notes=1
% % \begin{frame}[fragile]
% % Instructor's Notes:
% % \begin{itemize}
% % \item The reason for not tying output lines together is that 1 (high
% % voltage) means a low-resistance path to power, and 0 (low voltage)
% % means a low-resistance path to ground. If, in the crossed-out example
% % above, the output of one AND gate was 1 and of the other 0, this would
% % mean a direct connection between power and ground -- a short-circuit.
% % \item In some older technologies (TTL open collector, for example), you
% % 	could use a Wired-OR.
% % \end{itemize}
% % \end{frame}
% % \fi\ENotes

% % \begin{frame}[fragile]
% % \Title{XOR from Three-State Buffers}
% %  \Figure{!}{2.5in}{1in}{Figs/xorTrans} 

% %         Circuit analysis with tri-stage gates:
% %         \begin{itemize}
% %                 \item Label floating output as '---'
% %                 \item Tied lines better have exactly one non-floating!
% %         \end{itemize}
% %         \begin{center}
% %         \begin{tabular}{cc|cc|cc|c}
% %                 $X$ & $Y$ & $\bar{X}$ & $\bar{Y}$ & $F_0$ & $F_1$ & $F$\\
% %                 \hline
% %                 0 & 0 &&&&&\\
% %                 0 & 1 &&&&&\\
% %                 1 & 0 &&&&&\\
% %                 1 & 1 &&&&&
% %         \end{tabular}
% %         \end{center}                                                            
% % \BNotes\ifnum\Notes=1
% % ~
% % \fi\ENotes
% % \end{frame}

% % \begin{frame}[fragile]
% % \Title{Making Multiplexors from Three-State Buffers}
% % 		\PHFigure{!}{4in}{2in}{PHALL/B22}{Similar to Figure A.9.2}

% % IMPORTANT: Must ensure that at most one select input is 1, or
% % 		short-circuit may result (physical meltdown)\\
% % \BNotes\ifnum\Notes=1
% % Instructor's Notes:
% % This figure scales up in an obvious fashion.
% % \fi\ENotes
% % \end{frame}

% % \begin{frame}[fragile]
% % \Title{Example of SRAM Structure}
% % 		\PHFigure{!}{5in}{2.5in}{PHALL/B23}{Simlar to Figure A.9.3}
 
% % Does this design scale up well?
% % \end{frame}
% % \BNotes\ifnum\Notes=1
% % \begin{frame}[fragile]
% % Instructor's Notes:
% % \begin{itemize}
% % \item Point out the places where output lines are tied together using
% % three-state buffers (Dout[1] and Dout[0] lines).
% % \item  Discuss how this
% % scales up (horizontally for wider words, vertically for more words).
% % \item Point out, though, that if there are 32K words, a 14-to-32K decoder is
% % needed.  This is very large.

% % However, the size isn't so bad if you compare number of transistors per
% % AND gate of decoder to number of transistors in an 8 bit word made of
% % D latches.  If needed, we could reduce using two level decoding, but
% % we won't cover that in class.
% % \end{itemize}
% % \end{frame}
% % \fi\ENotes

% \begin{frame}[fragile]
%   \Title{6 Transistor SRAM Cell}
%   Flipflop takes about 40 transistors.

%   Can implement SRAM cell with 6 transistors:\bigskip
  
% \Figure{!}{5in}{2.in}{6transSRAM/6tranSRAM}  
% \BNotes\ifnum\Notes=1
% Instructor's Notes:
% It's hard to figure out from a black and white, but the next slide
% colours things to make it easier.
% \fi\ENotes
% \end{frame}

% \begin{frame}[fragile]
%   \Title{6 Transistor SRAM Cell--How It Works}
%   Put 1 on Cell Select, and then (for example)

%   put $1$ on Data and $0$ on $\overline{Data}$ \bigskip
  
%   \Figure{!}{5in}{2.in}{6transSRAM/6tranSRAMColor}
% \end{frame}
%   \BNotes\ifnum\Notes=1
%   \begin{frame}[fragile]
%   Instructor's Notes:
%     \begin{itemize}
%       \item When Cell Select is 0, Data is ignored.
%       \item When Cell Select is 1, the Data is passed to the internals.
%         The p and n transistors from Data and !Data are connected to
%         two internal transistors each;
%         Each sets its pair ``opposite'' to the other, but the low
%         resistance connections to power and ground are consistent with
%         the values of Data and !Data.
%         When Cell Select goes back to 0, the internal connections stay
%         the same.
%       \item Unclear how to read data; and once a value is stored, it's
%         unclear how the opposite value gets stored (ie, the Data lines
%         conflict with the internal low resistance paths to power and ground).

%         It's easy to see how to get rewrite to work if you insert two pmos
%         transistors in front of power and ground (with Cell Select as input), although even then it's unclear what happens with Cell Select goes to 0 (you have to assume that the transistors change resistance slowly when connected to neither power or ground);
%         read could work if you listened to Data lines rather than put data
%         on them.
%       \end{itemize}
% \end{frame}
%  \fi\ENotes

% \begin{frame}[fragile]
% \Title{Dynamic RAM}
% \begin{itemize}
% \item Even six transistors is too expensive
% \item Alternative: use a capacitor to store a charge to represent 1
% \item Problem: charge leaks away, must be refreshed
% \end{itemize}
% \BNotes\ifnum\Notes=1
% Instructor's Notes:
% \begin{itemize}
% \item The six-transistor implementation uses two cross-coupled NOT gates implemented using two transistors each, plus two additional transistors controlling access for both read and write.  See previous slides.
% \end{itemize}
% \fi\ENotes
% \end{frame}

% \begin{frame}[fragile]
% \Title{DRAM Cell}
% \Figure{!}{2.5in}{1.in}{NewTransFigs/dramcell}
% \begin{itemize}
% \item To write: place value on bit line, 1 on word line

% 	Change word line to 0 before changing bit line
% \item To read: put half-voltage on bit line, 1 on word line

% Charge in capacitor will slightly increase bit line voltage, \\
% no charge will slightly decrease voltage

% This change detected, amplified, and written back
% \end{itemize}
% \end{frame}
% \BNotes\ifnum\Notes=1
% \begin{frame}[fragile]
% Instructor's Notes:
% \begin{itemize}
% \item Briefly explain what a capacitor does
% \item The capacitor is made out of a transistor
% \item The reading method seems somewhat elaborate, but
% sense amplification is also used in reading the six-transistor SRAM
% cell
% \item Note that read is destructive, hence need for writeback
% \item The term ``word line'' makes no sense here, but it will make
% sense after the following slide.
% \end{itemize}
% \end{frame}
% \fi\ENotes

% \begin{frame}[fragile]
% \Title{Design of 4Mx1 DRAM}
% \PHFigure{!}{3in}{1.5in}{PHALL/B26}{Similar to Figure A.9.6}
% \begin{itemize}
% \item 20-bit address provided 11 bits at a time
% \item Whole row is read at once
% \item Column address selects single bit
% \item Refresh handled a row at a time (external controller)
% \item If capacitors hold charge for 4ms, refresh takes 80ns, fraction of time
% devoted to refresh is about 4\%
% \end{itemize}
% \end{frame}
% \BNotes\ifnum\Notes=1
% \begin{frame}[fragile]
% Instructor's Notes:
% \begin{itemize}
% \item Note that the figure is in error; a different set of bits goes into
% 	the MUX than goes into the decoder.
% \item There are additional signals called RAS (Row Access Strobe) and
% CAS (Column Access Strobe) to signal whether a row or column address
% is provided 
% \item The math on the refresh goes like this: $2048 \times 80$ ns
%   $=163840\times 10^{-9}$ secs, and dividing this by $4\times 10^{-3}$
%   secs gives $0.04096$
% \item In practice, refresh overhead can often be hidden by other
%   necessary operations in the total memory cycle
% \item Note that a ``word line'' now accesses 2048 bits -- not our
%   usual definition of ``word''
% \end{itemize}
% \end{frame}
% \fi\ENotes

% \begin{frame}[fragile]
% \Title{DRAM Complications}
% \begin{itemize}
% \item DRAM is cheaper than SRAM, but slower
% \item Refresh controller must also allow read/write access
% \item Possibility of getting more bits out at a time (e.g. page-mode RAM)
% \item SDRAM: synchronized DRAM
%    \begin{itemize}
%    \item Uses external clock to synchronize with processor
%    \item Useful in memory hierarchies 
%    \end{itemize}
% \end{itemize}
% \end{frame}
% \BNotes\ifnum\Notes=1
% \begin{frame}[fragile]
% Instructor's Notes:
% \begin{itemize}
% \item The textbook discusses page-mode, static-column-mode,
%   nibble-mode, and extended-data-out DRAM; there is also video
%   RAM. Details are not provided because they are complicated and vary
%   from vendor to vendor.
% \item SDRAM is often used in ``burst mode'' to access and transmit a
%   series of bits; thus it is suited for interfacing with caches and
%   paging mechanisms.
% \item We are not providing enough detail here for students to be
%   tested on these concepts; our goal is demystification.
% \end{itemize}
% \end{frame}
% \fi\ENotes
% %%%%%%%%%%%W23 zhk  edited

% % \begin{frame}[fragile]
% % \STitle{Designing Using Finite-State Machines}
% % 	%	\PHFigure{!}{3.5in}{2.75in}{PHALL/B27}{Figure A.10.1}
% % 		\Figure{!}{3.5in}{2in}{Figs/fsmHL}

% % High-level circuit implementation of finite-state machine
% % \end{frame}
% % \BNotes\ifnum\Notes=1
% % \begin{frame}[fragile]
% % Instructor's Notes:
% % \begin{itemize}
% % \item This is a schematic drawing from the text.
% % \item Students sometimes confuse this diagram with the graphical
% % representation of an FSM, introduced below.
% % \item The brown line is for Mealy machines (see below) and isn't there
% % 	for CS 251
% % \end{itemize}
% % \end{frame}
% % \fi\ENotes



% % \begin{frame}[fragile]
% % \Title{Example: Traffic Light}
% % \begin{itemize}
% % \item Output signals: NSlight, EWlight
% % \item Input signals: NScar, EWcar
% % \item State names: NSgreen, EWgreen (no yellow for now)
% % \item Functionality: want light to change only if car is waiting at
% % red light
% % \end{itemize}
% % \end{frame}
% % \BNotes\ifnum\Notes=1
% % \begin{frame}[fragile]
% % Instructor's Notes:
% % \begin{itemize}
% % \item Explain the meanings of these signals, as given in book (Appendix B.10).

% % Outputs:
% % \begin{itemize}
% % 	\item {\em NSlite:} When this signal is asserted, the light on the
% % 		north-south road is green; when it is deasserted, the light
% % 		on the north-south road is red.
% % 	\item {\em EWlite:} When this signal is asserted, the light on the
% % 		east-west road is green; when it is deasserted, the light
% % 		on the east-west road is red.
% % 	\item 
% % \end{itemize}

% % Inputs:
% % \begin{itemize}
% % 	\item {\em NScar:} Indicates that a car is over the detector
% % 		placed in the roadbed in front of the light on the
% % 		north-sout road (going north or south).
% % 	\item {\em EWcar:} Indicates that a car is over the detector
% % 		placed in the roadbed in front of the light on the
% % 		north-sout road (going north or south).
% % 	\item 
% % \end{itemize}

% % States:
% % \begin{itemize}
% % 	\item {\em NSgreen:} The traffic light is green in the north-south
% % 		direction.
% % 	\item {\em EWgreen:} The traffic light is green in the east-west
% % 		direction.
% % \end{itemize}
% % \item Do the
% % next-state function and output functions (listed in book) on the board.
% % \end{itemize}
% % \end{frame}
% % \fi\ENotes

% % \begin{frame}[fragile]
% % \Title{Graphical Representation of Traffic Light Controller}
% % %		\PHFigure{!}{3.5in}{2.5in}{PHALL/B28}{Figure A.10.2}
% % 		\Figure{!}{3.5in}{2in}{Figs/tlc}

% % \begin{itemize}
% % \item Names of states outside ovals
% % \item Output in given state inside oval
% % \item Transition arc labelled with Boolean formula of inputs
% % \end{itemize}
% % \end{frame}
% % \BNotes\ifnum\Notes=1
% % \begin{frame}[fragile]
% % Instructor's Notes:
% % \begin{itemize}
% % \item Clock rate has to be slow to avoid rapid light cycling
% % \item Book suggests clock period (cycle length) of 30 seconds (0.033 Hz)
% % \item Re-emphasize difference between representation here and one used
% % in language recognition
% % \item Point out equivalence of this representation and truth tables
% % \end{itemize}
% % \end{frame}
% % \fi\ENotes

% % \begin{frame}[fragile]
% % \STitle{Variations on Finite-State Machines}
% % \begin{itemize}
% % \item Moore machine: output depends only on state (what we use)
% % \item Mealy machine: output can depend on inputs
% % \item Moore machine may be faster, Mealy machine may be smaller
% % \item Conceptually, computation is infinite (input streams have no
% % beginning or end)
% % \item In practice, need to worry about power-up and power-down (as
% % with all our state devices)
% % \item Different in language-recognition context (e.g. CS 241)
% % \begin{itemize}
% % \item Input is single character at a time, not set of bits
% % \item Because strings have finite length, 
% % computation is finite (start state, final states)
% % \item Mealy machines used (outputs on transition arcs)
% % \end{itemize}
% % \end{itemize}
% % \BNotes\ifnum\Notes=1
% % ~% notes text
% % \fi\ENotes
% % \end{frame}

% % \begin{frame}[fragile]
% % \STitle{Electronic Implementation of Finite-State Controller}
% % \bigskip
% % \PHFigure{!}{4in}{3in}{PHALL/B29}{Figure A.10.3}
% % \end{frame}
% % \BNotes\ifnum\Notes=1
% % \begin{frame}[fragile]
% % Instructor's Notes:
% % \begin{itemize}
% % \item Complete the example:
% % \begin{itemize}
% % 	\item Do the state assignment
% % 	\item Write a formula for the next-state function
% % 	\item Write formulae for the outputs
% % 	\item Draw the circuit
% % \end{itemize}
% % \item Point out that the combinatorial logic is usually implemented via a
% % 	PLA or ROM. 

% % 	This is the scheme we will use to implement the control for our CPUs.
% % \end{itemize}
% % \end{frame}
% % \fi\ENotes

% % \begin{frame}[fragile]
% % \STitle{Extending the Traffic-Light Controller}
% % \begin{itemize}
% % \item Add 4-second yellow light
% % \item Assume 0.25Hz clock
% % \item need to add 28-second timer
% % \begin{itemize}
% % \item Timer input: TimerReset (TR)
% % \item Timer output: TimerSignal (TS)
% % \end{itemize}
% % \item Behaviour of system
% % \begin{itemize}
% % \item Stay green in one direction (red in other direction)
% % until car arrives or 32 seconds
% % elapse, whichever happens last
% % \item Green turns to yellow for 4 seconds; red in other direction stays
% % \item Yellow turns to red, red in other direction turns to green
% % \end{itemize}
% % \end{itemize}
% % \end{frame}
% % \BNotes\ifnum\Notes=1
% % \begin{frame}[fragile]
% % Instructor's Notes:
% %  \emph{This example was removed from 3rd Edition but added back into the 4th edition.}
% % This is the example worked out in exercises B.41 to B.44 of the
% % text.

% % Giving a more detailed example of a FSM implementation at this
% % point will allow students to work examples for themselves, as well as
% % to better understand the controllers developed for architectures.
% % \end{frame}
% % \fi\ENotes

% % \begin{frame}[fragile]
% % \Title{State Diagram of Extended Controller}
% % \begin{itemize}
% % \item Inputs: NScar, EWcar, TS
% % \item Outputs: NSg, NSy, NSr, EWg, EWy, EWr, TR
% % \Figure{!}{5in}{2.5in}{Figs/traffic-fsm}
% % \end{itemize}
% % \end{frame}
% % \BNotes\ifnum\Notes=1
% % \begin{frame}[fragile]
% % Instructor's Notes:
% % NSg is short for north-south green light; similarly, NSy and NSr are
% % the yellow and red lights, respectively, for the north-south direction.
% % \end{frame}
% % \fi\ENotes

% % \begin{frame}[fragile]
% % \Title{Next-State Table for Extended Controller}
% % \begin{center}
% % \small
% % \begin{tabular}{|c|c|c|c|c||c|c|c|c|c|}\hline
% % current & \multicolumn{3}{c|}{inputs} & next & current &
% %  \multicolumn{3}{c|}{inputs} & next \\ 
% % \cline{2-4}\cline{7-9}
% % state & NS- & EW- & & state &
% % state & NS- & EW- & & state \\
% % \cline{1-1}\cline{5-6}\cline{10-10}
% % $S_2S_1S_0$ & car & car & TS & $S_2'S_1'S_0'$  &
% % $S_2S_1S_0$  & car & car & TS &$S_2'S_1'S_0'$  \\ \hline
% % 0~~0~~0 & X & X & 0 & 0~~0~~0 & 
% %  1~~0~~0 & X & X & 0 & 1~~0~~0 \\
% % 0~~0~~0 & X & X & 1 & 0~~0~~1 & 
% %  1~~0~~0 & X & X & 1 & 1~~0~~1  \\
% % 0~~0~~1 & X & 0 & X & 0~~0~~1 &
% %  1~~0~~1 & 0 & X & X & 1~~0~~1  \\
% % 0~~0~~1 & X & 1 & X & 0~~1~~0 &
% %  1~~0~~1 & 1 & X & X & 1~~1~~0  \\
% % 0~~1~~0 & X & X & X & 1~~0~~0 & 
% %  1~~1~~0 & X & X & X & 0~~0~~0  \\
% % 0~~1~~1 & X & X & X & X~X~X &
% %  1~~1~~1 & X & X & X & X~X~X  \\ \hline
% % \end{tabular}
% % \end{center}

% % Note unused states, symmetries\\
% % \BNotes\ifnum\Notes=1
% % Instructor's Notes:\\
% % The symmetries are not surprising, since the state diagram is
% % symmetrical. We can exploit these to simplify the logic. The use of
% % don't cares to compress the table is crucial here, since otherwise
% % there would be 64 entries.
% % \fi\ENotes
% % \end{frame}

% % \begin{frame}[fragile]
% % \Title{Output Table For Extended Controller}
% % \begin{itemize}
% % \item Output table looks like truth table

% % 	Inputs are State, Outputs are Outputs

% % \item	Traffic light outputs: NSg, NSy, NSr, EWg, EWy, EWr, TR
% % \item If output listed in State, then 1 in output table

% % 	If output not listed in State, then 0 in output table
% % \end{itemize}
% % \begin{center}
% % \begin{tabular}{ccc|ccccccc}
% % $S_2$ & $S_1$ & $S_0$ & NSg & NSy & NSr & EWg & EWy & EWr & TR\\\hline
% % 0 & 0 & 0 &\\
% % 0 & 0 & 1 &\\
% % 0 & 1 & 0 &\\
% % 0 & 1 & 1 &\\\hline
% % 1 & 0 & 0 &\\
% % 1 & 0 & 1 &\\
% % 1 & 1 & 0 &\\
% % 1 & 1 & 1 &\\
% % \end{tabular}
% % \end{center}
% % \end{frame}
% % \BNotes\ifnum\Notes=1
% % \begin{frame}[fragile]
% % Instructor's Notes:
% % \begin{itemize}
% % 	\item Put up state diagram to explain 3rd bullet
% % 	\item From state diagram, fill in table.  Use don't cares for
% % 		unused states
% % \end{itemize}
% % \end{frame}
% % \fi\ENotes

% % \begin{frame}[fragile]
% % \Title{Next-State/Output Logic For Extended Controller}

% % \vspace*{.1in}
% % Current state = $S_2 S_1 S_0$, next state = $S'_2 S'_1 S'_0$

% % \vspace*{.1in}
% % $S'_0 = \Bar{S_1}\Bar{S_0}\cdot TS + \Bar{S_2}\Bar{S_1}S_0\cdot \Bar{EWcar} + 
% % S_2\Bar{S_1}S_0\cdot \Bar{NScar}$

% % \vspace*{.1in}
% % $S'_1 = \Bar{S_2}\Bar{S_1}S_0\cdot EWcar + S_2\Bar{S_1}S_0\cdot NScar$

% % \vspace*{.1in}
% % $S'_2 = \Bar{S_2}S_1\Bar{S_0} + S_2\Bar{S_1}$

% % \vspace*{.1in}
% % $NSg = \Bar{S_2}\Bar{S_1}$, $EWg = S_2 \Bar{S_1}$

% % \vspace*{.1in}
% % $NSy = \Bar{S_2}S_1\Bar{S_0}$, $EWy = S_2 S_1\Bar{S_0}$

% % \vspace*{.1in}
% % $NSr = S_2$, $EWr = \Bar{S_2}$

% % \vspace*{.1in}
% % $TR = S_1\Bar{S_0}$

% % \end{frame}
% % \BNotes\ifnum\Notes=1
% % \begin{frame}[fragile]
% % Instructor's Notes:
% % \begin{itemize}
% % \item You can show the students on the board how to simply sum the minterms
% % 	multiplied by the appropriate combination of input signals, but then
% % 	either simplify the resulting expressions via algebra or (better)
% % 	point out how to take advantages of don't cares and symmetries.
% % \item Warn them about don't care's and unused states.  In particular,
% % 	suppose that we some how ended up in an unused state (cosmic ray?
% % 	strange start condition?), and what if both lights green at once!
% % 	Or what if one direction has red, green, and yellow all on at once?
% % 	Or what if none of the lights are on?

% % 	Likewise, if we end up in an unused state, what is the next state?
% % 	We'd like to go to a sensible next state.  In the case of a traffic
% % 	light, any used state is probably fine, but you can imagine situations
% % 	where a sequence of states represents a process and going to the middle
% % 	of the process would be catastrophic (example: robot tooth extraction: 
% % 	Steps: Step 1: give anestesia.  Step 2: remove tooth.  Step 3: whatever.
% % 	If an unused state jumped to Step 2 (skipping step 1), the patient 
% % 	would be unhappy!)
% % \end{itemize}
% % \end{frame}
% % \begin{frame}[fragile]
% % Instructor's Notes Continued:
% % \begin{itemize}
% % \item After finishing the example, go back to the State Diagram and
% % 	ask "what if we want a pedestrian button?  How do we modify the
% % 	state diagram?"  The answer is to add EWPed and NSPed signals,
% % 	and change the arc out of 001 to be EWCar+EWPed and the arc
% % 	looping back to 001 to bEWCar AND bEWPed.

% % 	This extension is important as its the only time we've shown that
% % 	arcs can be labeled with Boolean formulas.  Be sure to point out
% % 	that the set of arcs out of a state must be consistent and complete:
% % 	ie, for all settings of the variables there is exactly one arc
% % 	out of the state.
% % \end{itemize}
% % \end{frame}
% % \fi\ENotes


% % \begin{frame}[fragile]
% % \Title{Outputs}
% % Output based on {\em current state}
% % \begin{center}
% % \begin{tabular}{cc}
% % \ifnum\slides=0
% % \includegraphics[height=0.75in]{Figs/fsmE} & 
% % \else
% % \includegraphics[height=1.75in]{Figs/fsmE} & 
% % \fi
% % \begin{tabular}{cc}
% % Next State & Output \\
% % \begin{tabular}{cc|c}
% % S&A&S'\\
% % \hline
% % 0  &0&1\\
% % 0  &1&0\\
% % 1  &X&0\\
% % \end{tabular}&
% % \begin{tabular}{c|c}
% % S&T\\
% % \hline
% % 0 & 1\\
% % 1 & 0
% % \end{tabular}
% % \end{tabular}
% % \vspace*{0.5in}~\\
% % \ifnum\slides=0
% % \includegraphics[height=0.75in]{Figs/fsmCirc} & 
% % \else
% % \includegraphics[height=1.75in]{Figs/fsmCirc} & 
% % \fi
% % \ifnum\slides=0
% % \includegraphics[height=0.75in]{Figs/fsmC} 
% % \else
% % \includegraphics[height=1.75in]{Figs/fsmC} 
% % \fi
% % \end{tabular}
% % \end{center}
% % \end{frame}
% % \BNotes\ifnum\Notes=1
% % \begin{frame}[fragile]
% % Instructor's Notes:
% % \begin{itemize}
% % 	\item The point of this slide is to emphasize that the output is
% % 		based on the current state.
% %     \item The upper left shows a state diagram, the upper right shows
% % 		its truth table.

% % 		The lower left shows a circuit for this FSM.  The important thing
% % 		here is that the next state ($S'$) isn't used to compute $T$.

% % 		The lower right is the important one: while $S=0$, the output $T$ is 1.
% % \end{itemize}
% % \end{frame}
% % \fi\ENotes

% %%%%%%%%%%%%%%%%%%%%%%%%added by ZHK for W23 updated in S23

%      \begin{frame}[fragile]
% \STitle{Textbook Readings}
% \begin{itemize}
% 	\item Section 4.2
%  \item Appendix A.7
%  \item Appendix A.8
%  \end{itemize} 
%  \end{frame}

%%%%%%%%%%%%%%%%%%%%%%%%added by ZHK for W23, updated in S23

