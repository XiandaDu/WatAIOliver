\setlength{\columnseprule}{1pt}
\def\columnseprulecolor{\color{blue}}


% % \begin{frame}[fragile]
% % \Title{ Single-Cycle Processor Implementation}
% % \begin{itemize}

% % \item How to build datapath, control for specific architecture
% % \item We will implement small subset of ARM operations:
% % \begin{itemize}
% % \item Load ({\tt LDUR}) and store ({\tt STUR})
% % \item Add ({\tt ADD}), subtract ({\tt SUB}), and ({\tt AND}), or ({\tt
% % ORR}).
% % \item Compare and branch on zero ({\tt CBZ}) and branch ({\tt B})
% % \end{itemize}
% % \item These suffice to illustrate fundamental ideas
% % \end{itemize}
% % \BNotes\ifnum\Notes=1
% % \begin{itemize}
% % \item
% % Sections 4.1 and 4.2 are introductory material; section 4.3 discusses building the
% % datapath, and section 4.4 completes the description and discusses
% % control. Section C.2 gives details of implementing control for the MIPS computer (not ARM).  They don't appear to give a similar thing for ARM.

% % \item Students
% % should be careful to read the captions of all figures carefully, since
% % many contain important information that is not discussed in the main
% % text. 

% % \item We don't implement branch for the single cycle architecture in the initial datapaths, but we add it as an instruction after completing an initial implementation.
% % \end{itemize}
% % \fi\ENotes
% % \end{frame}


% \begin{frame}[fragile]
% \STitle{Review of ARM Architecture}
% \begin{itemize}
% \item 32 registers (numbered 0 to 31), each with 64 bits
% \item Register 31 (X31 or XZR) always supplies the value 0
% \item Data Memory has 64-bit words (double-words)
% \item Instruction memory has 32 bits words
% \item Memory is byte-addressable\\
%   (word addresses multiples of 4, double-word addresses multiples of 8)

% % \item Instruction memory is All ARM instructions are 32 bits long
% \end{itemize}

% \begin{tcolorbox}[enhanced,attach boxed title to top center={yshift=-3mm,yshifttext=-1mm},
%   colback=blue!5!white,colframe=blue!75!black,colbacktitle=blue!80!black,
%   title=Think About It,fonttitle=\bfseries,
%   boxed title style={size=small,colframe=red!50!black} ]
% Words have multiple bytes, which byte should be the address of the word?
% \begin{itemize}
%     \item (Double-)Words have the address of their most significant byte\footnote{in ARMv8 this is dynamic, x86 prefers LSB as word address}
% \end{itemize}

%   \end{tcolorbox}

% \BNotes\ifnum\Notes=1
% ~% notes text
% \fi\ENotes
% \end{frame}

% \begin{frame}[fragile]
% \Title{Review of ARM Instructions}
% \begin{itemize}
% \item Load: {\tt LDUR X1, [X2,\#200]}
% \begin{itemize}
% \item Operands are register to be loaded, address in memory
% \item Addressing modes: register, base (displacement), immediate
% \end{itemize}
% \item Add: {\tt ADD X1, X2, X3}
% \begin{itemize}
% \item Operands are destination register, two source registers
% \end{itemize}
% \item Conditional branch: {\tt CBZ X1, \#10}
% \begin{itemize}
% \item Operand is registers to compare to 0, relative jump offset
% \item Addressing: PC relative
% \end{itemize}
% \item Branch: {\tt B 3000}
% \begin{itemize}
% \item Operand is word offset of next instruction
% \item Addressing mode: PC-relative

% 	(need to multiply by 4)
% \end{itemize}
% \end{itemize}
% \BNotes\ifnum\Notes=1
% \begin{itemize}
% \item Store works analogously to load.

% \item SUB, AND, ORR work analogously to add. The add instruction
% here has only one addressing mode (register); other instructions such
% as {\tt ADDI} use other modes.

% \item Syntax of conditional branch: If contents of named register equal to 0,
% PC gets 40 added to it,
% because the ARM architecture is byte-addressable but instructions are
% word-aligned, so offsets (and immediates, as in the jump instruction)
% are word counts, not byte addresses. The only addressing mode here is
% PC-relative. 

% \item The addressing mode of the branch instruction is PC-relative
% \end{itemize}
% \fi\ENotes
% \end{frame}

% % \begin{frame}[fragile]
% % \STitle{High-Level View of ARM Functional Units}
% % \PHFigure{!}{2in}{2in}{ARMFigures/Fig0402-crop}{Figure 4.2}
% % \begin{itemize}
% % \item Multiplexors used to send two signals to one location
% % \item Control unit tells other units what to do
% % \end{itemize}
% % \BNotes\ifnum\Notes=1
% % \begin{itemize}
% % \item The bullets are the interesting part of this slide
% % \end{itemize}
% % \fi\ENotes
% % \end{frame}

% \begin{frame}[fragile]
% \STitle{High-Level View of ARM Functional Units}
% %	\PHFigure{!}{3in}{3in}{ARMFigures/Fig0401-crop}{Figure 4.1}
% 	\Figure{!}{1.5in}{1.75in}{Figs/armHL}
% \begin{itemize}
% \item PC: Program Counter (address of current instruction)
% \item Fetch-execute cycle:
% \begin{itemize}
% \item Fetch instruction (update PC)
% \item Execute instruction
% \begin{itemize}
% \item Fetch register operands
% \item Compute result
% \item Store into registers OR use to index memory
% \end{itemize}
% \end{itemize}
% \end{itemize}
% \BNotes\ifnum\Notes=1
% \begin{itemize}
% \item The simplified explanation of fetch-execute on the slide does not take
% 	branching, conditional or otherwise, into account. This should be
% 	discussed at this point verbally. Note the separate instruction and
% 	data memories; this also is a simplification, to be justified on the
% 	next slide.
% \item Point out that the register file is the one we designed earlier and
% 	that the ALU is the one we design.
% 	Memory is also as we discussed earlier.
% \end{itemize}
% \fi\ENotes
% \end{frame}

% % \begin{frame}[fragile]
% % \STitle{High-Level View of ARM Functional Units}
% % \PHFigure{!}{2in}{2in}{ARMFigures/Fig0402-crop}{Figure 4.2}
% % \begin{itemize}
% % \item Multiplexors used to send two signals to one location
% % \item Control unit tells other units what to do
% % \end{itemize}
% % \BNotes\ifnum\Notes=1
% % \begin{itemize}
% % \item The bullets are the interesting part of this slide
% % \end{itemize}
% % \fi\ENotes
% % \end{frame}

% \input 04-Single-Cycle-Processor-Implementation/mem

% \begin{frame}[fragile]
% \STitle{Revisit: High-Level View of ARM Functional Units}
% %	\PHFigure{!}{3in}{3in}{ARMFigures/Fig0401-crop}{Figure 4.1}
% 	\Figure{!}{1.5in}{1.75in}{Figs/armHL}
% \begin{itemize}
% \item PC: Program Counter (address of current instruction)
% \item Fetch-execute cycle:
% \begin{itemize}
% \item Fetch instruction (update PC)
% \item Execute instruction
% \begin{itemize}
% \item Fetch register operands
% \item Compute result
% \item Store into registers OR use to index memory
% \end{itemize}
% \end{itemize}
% \end{itemize}
% \BNotes\ifnum\Notes=1
% \begin{itemize}
% \item The simplified explanation of fetch-execute on the slide does not take
% 	branching, conditional or otherwise, into account. This should be
% 	discussed at this point verbally. Note the separate instruction and
% 	data memories; this also is a simplification, to be justified on the
% 	next slide.
% \item Point out that the register file is the one we designed earlier and
% 	that the ALU is the one we design.
% 	Memory is also as we discussed earlier.
% \end{itemize}
% \fi\ENotes
% \end{frame}


% \begin{frame}[fragile]
% \STitle{High-Level View of ARM Functional Units and Control}
% \PHFigure{!}{2in}{2in}{ARMFigures/Fig0402-crop}{Figure 4.2}
% \begin{itemize}
% \item Multiplexors used to send two signals to one location
% \item Control unit tells other units what to do
% \end{itemize}
% \BNotes\ifnum\Notes=1
% \begin{itemize}
% \item The bullets are the interesting part of this slide
% \end{itemize}
% \fi\ENotes
% \end{frame}


% \begin{frame}[fragile]
% \STitle{First Implementation: One Cycle Per Instruction}
% \begin{itemize}
% \item Simpler to understand, but not practical
% \item Requires separate instruction and data memories

% \item Clock must be slowed to speed of slowest instruction
% \item Subsequently we look at multicycle implementations
% \end{itemize}
% \BNotes\ifnum\Notes=1
% \begin{itemize}
% \item If we did not have separate instruction and data memories, we could
% 	not complete the fetch-execute of an instruction in one clock cycle,
% 	since two accesses to memory may be necessary. In practice instruction
% 	and data memories are nearly always combined.
% \item Of course, in a real computer, we have separate instruction and data
% 	cache, which is very much like having two separate memories.
% \end{itemize}
% \fi\ENotes
% \end{frame}



% \begin{frame}[fragile]
% \Title{Implementing Fetch Portion of Fetch-Execute}
% 	%\PHFigure{!}{4in}{3in}{PHALL/F0505}{Figure 4.6}
% 	\Figure{!}{3in}{2in}{Figs/fetchExecute}
% {\small \begin{itemize}
% \item State elements here are PC (register) and instruction memory
% \item Adder is combinational
% \end{itemize}}
% \begin{tcolorbox}[enhanced,attach boxed title to top center={yshift=-3mm,yshifttext=-1mm},
%   colback=blue!5!white,colframe=blue!75!black,colbacktitle=blue!80!black,
%   title=Think About It,fonttitle=\bfseries,
%   boxed title style={size=small,colframe=red!50!black} ]
%   Can we read from PC and write to PC in the same clock cycle? 
%   \end{tcolorbox}
%   % \item PC is updated in the next clock cycle

% \BNotes\ifnum\Notes=1
% \begin{itemize}
% % \item Add slashes on the diagram to indicate width; all lines shown are 32
% % bits wide.
% \item output from PC is 64 bits wide
% \item output from Instruction Mem is 32 bits wide
% \item Note that PC+4 is in parallel to reading instruction from memory
% \end{itemize}
% \fi\ENotes
% \end{frame}

% \begin{frame}[fragile]
% \Title{Datapath components for R-format instructions}
% \begin{itemize}
% \item Example: {\tt ADD X1, X2, X3}
% \includegraphics[scale=0.2]{04-Single-Cycle-Processor-Implementation/figures/R-format.png}

% %	\PHFigure{!}{3in}{2in}{PHALL/F0507}{Figure 4.10 (sort of)}
% 	\Figure{!}{1.5in}{1.5in}{Figs/r-format}
%  \begin{tcolorbox}[enhanced,attach boxed title to top center={yshift=-3mm,yshifttext=-1mm},
%   colback=blue!5!white,colframe=blue!75!black,colbacktitle=blue!80!black,
%   title=Think About It,fonttitle=\bfseries,
%   boxed title style={size=small,colframe=red!50!black} ]
%   Can we execute the instruction \texttt{ADD X1, X1, X2}?
% \end{tcolorbox}
% \item Design permits read/write of same register
% \end{itemize}
% \BNotes\ifnum\Notes=1
% \begin{itemize}
% \item
% We designed edge-triggered multiport register files earlier. If an
% instruction like {\tt ADD X1, X1, X2} is executed, the old value
% of register {\tt X1} is used as an addend; the new value is written
% into the register file, but will not be available for reading (will
% not affect the output of the state element) until the next clock
% tick. Thus there is no conflict.
% \item
% Put slashes on the diagram to indicate widths: register names are 5
% bits (number of bits needed to address 32 separate registers), all
% other lines are 64 bits. Note that we have not specified yet where the
% bits come from in the instruction; we discuss precise instruction
% formats after doing some high-level design.
% \end{itemize}
% \fi\ENotes
% \end{frame}

% \begin{frame}[fragile]
% \Title{Datapath components for D-format instructions}
% \begin{itemize}
% \item Example: {\tt LDUR X1, [X2,\#200]}
% \includegraphics[scale=0.2]{04-Single-Cycle-Processor-Implementation/figures/D-format.png}

% 	%\PHFigure{!}{4in}{3in}{PHALL/F0509-64}{Figure 4.9 (sort of)}
% 	\Figure{!}{3in}{2in}{Figs/loadstore}
% \item Sign extend is combinational 
% \item Assume for simplicity that data memory is edge-triggered
% \end{itemize}
% \BNotes\ifnum\Notes=1
% In reality, large memories are not clocked, and the CPU must be
% designed to take into account hold times for inputs and response times
% for outputs.
% \fi\ENotes
% \end{frame}

% \begin{frame}[fragile]
%   \STitle{Combining R-format and D-format Datapath Components}

%   \PHFigure{!}{1.5in}{1.7in}{ARMFigures/Fig0410m-crop.pdf}{Figure 4.10}
%   \begin{itemize}
%   \item R-format and Memory instructions are similar
%   \item MUX in front of second ALU input selects Reg File output OR Sign-extended constant
%   \item MUX ``in front of'' Reg File write data selects ALU output OR Memory Read output
%     \end{itemize}
%   \BNotes\ifnum\Notes=1
%   \begin{itemize}
%   \item The book authors label input to sign extend as ``32''; it should be 9
%     as on this slide
%     \end{itemize}
% \fi\ENotes
% \end{frame}

% \begin{frame}[fragile]
% \Title{Datapath components for CB-format instructions}
% \begin{itemize}
% \item Example: {\tt CBZ X1, \#100}
% \includegraphics[scale=0.2]{04-Single-Cycle-Processor-Implementation/figures/CB-format.png}

% \includegraphics[scale=0.5]{04-Single-Cycle-Processor-Implementation/figures/cb-datapath.png}
% 	% \PHFigure{!}{4.5in}{4in}{PHALL/F0510-64}{Figure 4.9}
% 	% \Figure{!}{2in}{2in}{Figs/branch}
% \item Shift is necessary because offset given is in words
% \item Still need mechanism to control PC loading
% \item Special ALUop to \texttt{Pass-B} -- pass input 'b' to ouput ('a' ignored).
% \end{itemize}
% \BNotes\ifnum\Notes=1
% \fi\ENotes
% \end{frame}


% \begin{frame}[fragile]
% \STitle{Assembled Single-Cycle Datapath with Components for R--, D-- and CB--format Instructions}
% 	\PHFigure{!}{5.05in}{1.9in}{ARMFigures/Fig0411-crop}{Figure 4.11}


% \begin{itemize}
% \item MUX before PC

%   Selects between PC+4 OR CBZ address
%   \item Note the ``32-bit'' input to Sign-extend
% \end{itemize}
% \BNotes\ifnum\Notes=1
% \begin{itemize}
% \item The third MUX chooses what value is loaded into the PC, the
% computed branch target or the incremented current PC value.
% \item This diagram already takes a step towards designing control by
% including a separate ALU subcontrol. We still need a main control to
% generate all signals shown here from the instruction. To do that, we
% need a precise set of instruction formats. 

% \item Note the ``32'' input to the Sign-extend.  The actual input is either
%   9 bits (LDUR) or 19 bits (Memory).  And if we implement branch ({\tt B}) then
%   it could also be 26 bits.

%   The book says choosing between the 9 bits of 19 bits (for sign extension)
%   could be handled with a 2:1 MUX, using some of the OPCODE bits as select
%   lines.  But it mistakenly suggests that all you need is a single 2:1
%   bit to choose which bit to use for sign extension.  You would need at
%   least 10 2:1 MUXes, one for each of the extra 10 bits in the 19-bit
%   memory constant field, since you have to choose between the sign extend
%   bit of the LDUR instruction or the bit of the memory instruction.

%   Alternatively, the
%   Sign Extend unit is just wires, and these wires for the 9-bit input go
%   to a different place than the wires for the 19-bit input.
%   Ie, the Sign-Extend with a
%   single output is just conceptual; you'd have two or three sets of wires
%   leaving the instruction, each getting sign-extended to 64 bits, and then
%   sent to the appropriate input.

%   When branch is added, however, you would need MUXes, since the memory
%   instructions use the same ALU input as the branch instruction.
% \end{itemize}
% \fi\ENotes
% \end{frame}



\begin{frame}[fragile]
\STitle{Instruction Formats}

	%\PHFigure{!}{4.in}{2.75in}{ARMFigures/Fig0414-crop}{Figure 4.14}
	\Figure{!}{2.5in}{1.25in}{Figs/arminstructions}

 R-format: {\tt ADD X1, X2, X3} $\Rightarrow$ {\tt ADD {\bf Rd},Rn,Rm}\smallskip

 I-format: {\tt ADDI X1, X2, \#4} $\Rightarrow$ {\tt ADDI {\bf Rd},Rn,\#4} \smallskip

Load/store: {\tt LDUR X1, [X2,\#200]} $\Rightarrow$ {\tt LDUR {\bf Rt},[Rn,\#200]}\\

\hspace*{1in}{\tt STUR X1, [X2,\#400]} $\Rightarrow$ {\tt STUR { Rt},[Rn,\#400]}\\

\medskip
(bold face register is \textbf{written} to)

\BNotes\ifnum\Notes=1
\begin{itemize}
\item First field is operation code (opcode) but length varies

\item Rn (bits 9:5) are the ``first'' register operand (i.e., the top
  one on the register file).
  
\item Note ``destination register'' field is different for {\tt ADD}
and {\tt LDUR}; this complicates the datapath some more

\item Draw correspondences on slide: in first type, {\tt X1} is
	field {\tt Dd}; in second type, it is {\tt Rt}. The rest should be
	obvious.
\item Load/Store are examples of D-format instructions; {\tt CBZ} is CB-format.
\end{itemize}
\fi\ENotes
\end{frame}
\begin{frame}[fragile]
  
\STitle{Branch Instruction Formats}

	%\PHFigure{!}{4.in}{2.75in}{ARMFigures/Fig0414-crop}{Figure 4.14}
	\Figure{!}{1.9in}{0.75in}{Figs/armbranchinstructions}

Branch: {\tt B \#3000} $\Rightarrow$ {\tt B \#3000}\smallskip

Conditional Branch: {\tt CBZ X1,\#3000} $\Rightarrow$ {\tt CBZ Rt,\#3000}

\medskip

\BNotes\ifnum\Notes=1
\begin{itemize}
\item Load/Store are examples of D-format instructions; {\tt CBZ} is CB-format.
\end{itemize}
\fi\ENotes
\end{frame}

\begin{frame}[fragile]
  \Title{Opcodes}
  \begin{center}
  \begin{tabular}{llc}
    Instruction~~ & Opcode & Format\\
    \hline
    B    & \texttt{0001\,01} {\tiny{($5_{10}$)}} & B-format\\
    ADD  & \texttt{1000\,1011\,000} {\tiny{($1112_{10}$)}}~~ & R-format\\
    ADDI & \texttt{1001\,0001\,00} {\tiny{($580_{10}$)}} & I-format\\
    CBZ  & \texttt{1011\,0100} {\tiny{($180_{10}$)}}& CB-format\\
    CBNZ & \texttt{1011\,0101} {\tiny{($181_{10}$)}}& CB-format\\
    SUB  & \texttt{1100\,1011\,000} {\tiny{($1624_{10}$)}} & R-format\\
    SUBI & \texttt{1101\,0001\,00} {\tiny{($836_{10}$)}} & I-format\\
    STUR & \texttt{1111\,1000\,000} {\tiny{($1984_{10}$)}} & D-format\\
    LDUR & \texttt{1111\,1000\,010} {\tiny{($1986_{10}$)}} & D-format\\
  \end{tabular}
  \begin{itemize}
  \item Opcodes length based on format

    R-format: 11 bits~~~~~~~~~~~~
    I-format: 10 bits

    D-format: 11 bits~~~~~~~~~~~~
    B-format: 6 bits

    CB-format: 8 bits

    % Figure 2.20 gives larger table
    \end{itemize}
  \end{center}
\BNotes\ifnum\Notes=1
\begin{itemize}
\item Students are not expected to memorize these.

	However, they may need them for the assignments.

	If needed for an exam, the Opcodes will be provided.
\end{itemize}
\fi\ENotes
  
\end{frame}

\begin{frame}[fragile]
  \STitle{Datapath with Control}
\PHFigure{!}{4.5in}{2.5in}{ARMFigures/Fig0417-crop}{Figure 4.17}


\goodbreak

\begin{minipage}{\textwidth}
	\vspace*{0.5in}{\tt LDUR X1, [X1,\#200]}

	\PHFigure{!}{5in}{5in}{ARMFigures/Fig0417-crop}{Figure 4.17}
\end{minipage}

\goodbreak

\begin{minipage}{\textwidth}
	\vspace*{0.5in}{\tt CBZ X1, \#100}

	\PHFigure{!}{5in}{5in}{ARMFigures/Fig0417-crop}{Figure 4.17}
\end{minipage}

\BNotes\ifnum\Notes=1
\begin{itemize}
\item Things to note on this slide:
\begin{itemize}
	\item The instruction bits are split off and sent to the
		relevant hardware
	\item The dual control unit
	\item The MUX in front of Read register 2
	\item The branching hardware and the mux that chooses between
		CBZ and PC+4
	\item The MUX to the right of Data memory
\end{itemize}
\item Let's
make sure everyone is familiar with the operation of the
datapath. Trace the units involved in the following operations, using
markers on the overhead slide:
\begin{itemize}
\item {\tt ADD X1, X2, X3}
\item {\tt LDUR X1, [X2,\#200]}
\item {\tt CNZ X1, \#100}
\end{itemize}
	The students have three copies of this diagram in the course notes.
\end{itemize}
\fi\ENotes
\end{frame}

% for comprehensive
\newpage
\begin{frame}[fragile]
\Title{Meaning of Signals in Single-Cycle Datapath}
	\begin{center}
	\begin{tabular}{|l|l|l|}\hline
	Signal & Signal=0 & Signal=1 \\ \hline
	Reg2Loc & Instruction[20--16] & Instruction[4--0] \\
 \hline
	Branch* & no branch & Branch \\
\hline
 MemRead & no effect & memory read \\
\hline	
 MemToReg & reg write from ALU & reg write from memory \\ 
\hline	
ALUOp & discussion to come later\footnote{ALUOp is a two bit output from the main Control unit and is used as input to the ALU Control to set the Operation select lines.}& \\
\hline
 MemWrite & no effect & memory written \\
\hline	
 ALUSrc & ALU B input from reg & immediate from instruction \\
\hline	
 RegWrite & no effect & register written \\ \hline
	\end{tabular}
	\end{center}
\hl{* Branch is ANDed with Zero from ALU to get PCsrc}
 
% (Full version in Figure 4.16 of text.)
\BNotes\ifnum\Notes=1
\begin{itemize}
\item Go over the various control lines, using this slide and the previous
one, and make sure the meaning of each
one is clear, since this table has been abbreviated to make it fit on
the slide.
\item Note that 20--16 and 4--0 refer to bits of the instruction that are
  used as the second register to read from the register file.
\end{itemize}
\fi\ENotes
\end{frame}

% \begin{frame}[fragile]
% \Title{Meaning of Signals in Single-Cycle Datapath}
% 	\begin{center}
% 	\begin{tabular}{|l|l|l|}\hline
% 	Signal & Signal=0 & Signal=1 \\ \hline
% 	Reg2Loc & 20--16 & 4--0 \\
% 	Branch* & no branch & Branch \\
% 	MemRead & no effect & memory read \\
% 	MemToReg & reg write from ALU & reg write from memory \\ 
% 	MemWrite & no effect & memory written \\
% 	ALUSrc & ALU B input from reg & immediate from instruction \\
% 	RegWrite & no effect & register written \\ \hline
% 	\end{tabular}
% 	\end{center}
% * Branch is ANDed with Zero from ALU to get PCsrc

% (Full version in Figure 4.16 of text.)
% \BNotes\ifnum\Notes=1
% \begin{itemize}
% \item Go over the various control lines, using this slide and the previous
% one, and make sure the meaning of each
% one is clear, since this table has been abbreviated to make it fit on
% the slide.
% \item Note that 20--16 and 4--0 refer to bits of the instruction that are
%   used as the second register to read from the register file.
% \end{itemize}
% \fi\ENotes
% \end{frame}

\begin{frame}[fragile]
\STitle{Example - Tracing ARM instructions through Datapath}
    \begin{tcolorbox}[enhanced,attach boxed title to top center={yshift=-3mm,yshifttext=-1mm},
  colback=red!5!white,colframe=red!75!black,colbacktitle=red!80!black,
  title=Try this,fonttitle=\bfseries,
  boxed title style={size=small,colframe=red!50!black} ]
Can you trace the following ARM instructions through the Single Cycle Datapath?
\begin{verbatim}
    116: ADD X1, X2, X3
    400: LDUR X1, [X1, #200]
    120: CBZ X1, #100
\end{verbatim}
  \end{tcolorbox}
\end{frame}

\begin{frame}[fragile]\ifnum\slides=0
\ifnum\Ans=1
\begin{columns}

    \column{0.6\textwidth}
        {\footnotesize \tt 116: {\color{red}ADD} {\color{orange}X1}, {\color{purple}X2}, {\color{blue}X3}
  
  {\color{red}1000 1011 000}{\color{blue}0 0011} 0000 00{\color{purple}00 010}{\color{orange}0 0001}}

{\footnotesize \textbf{0x8B030041}}

  \includegraphics[width=\textwidth]{04-Single-Cycle-Processor-Implementation/figures/r-format.png}

  \PHFigure{!}{4in}{2in}{ARMFigures/Fig0417-crop}{Figure 4.17}
  
    \column{0.4\textwidth}

    \includegraphics[width=\textwidth]{04-Single-Cycle-Processor-Implementation/figures/compressedDecodeopcodes.png}

    \includegraphics[width=\textwidth]{04-Single-Cycle-Processor-Implementation/figures/compressedDecodeSignals.png}

    \vspace{5cm}

\end{columns}
\fi

\ifnum\Ans=0
{\tt 116: ADD X1, X2, X3}
\PHFigure{!}{4in}{2in}{ARMFigures/Fig0417-crop}{Figure 4.17}
\fi
  
  \else
  \STitle{Datapath with Control}
\fi


\BNotes\ifnum\Notes=1
\begin{itemize}
\item Things to note on this slide:
\begin{itemize}
	\item The instruction bits are split off and sent to the
		relevant hardware
	\item The dual control unit
	\item The MUX in front of Read register 2
	\item The branching hardware and the mux that chooses between
		CBZ and PC+4
	\item The MUX to the right of Data memory
\end{itemize}
\item Let's
make sure everyone is familiar with the operation of the
datapath. Trace the units involved in the following operations, using
markers on the overhead slide:
\begin{itemize}
\item {\tt ADD X1, X2, X3}
\item {\tt LDUR X1, [X2,\#200]}
\item {\tt CNZ X1, \#100}
\end{itemize}
	The students have three copies of this diagram in the course notes.
\end{itemize}
\fi\ENotes
\end{frame}

\begin{frame}[fragile]\ifnum\slides=0
\ifnum\Ans=1
\begin{columns}
    \column{0.6\textwidth}
        {\footnotesize 400: {\color{red}LDUR} {\color{orange}X1}, [{\color{purple}X1},{\color{blue}\#200}]
  
  {\color{red}1111 1000 010}{\color{blue}0 1100 1000} 00{\color{purple}00 001}{\color{orange}0 0001}}

{\footnotesize \textbf{0xF84C8021}}

  \includegraphics[width=\textwidth]{04-Single-Cycle-Processor-Implementation/figures/r-format.png}

  \PHFigure{!}{4in}{2in}{ARMFigures/Fig0417-crop}{Figure 4.17}
  
    \column{0.4\textwidth}

    \includegraphics[width=\textwidth]{04-Single-Cycle-Processor-Implementation/figures/compressedDecodeopcodes.png}

    \includegraphics[width=\textwidth]{04-Single-Cycle-Processor-Implementation/figures/compressedDecodeSignals.png}

    \vspace{5cm}

\end{columns}
\fi

\ifnum\Ans=0
{\tt 400: LDUR X1, [X1,\#200]}

\PHFigure{!}{4in}{2in}{ARMFigures/Fig0417-crop}{Figure 4.17}
\fi

  \else
  \STitle{Datapath with Control}
\fi

\BNotes\ifnum\Notes=1
\begin{itemize}
\item Things to note on this slide:
\begin{itemize}
	\item The instruction bits are split off and sent to the
		relevant hardware
	\item The dual control unit
	\item The MUX in front of Read register 2
	\item The branching hardware and the mux that chooses between
		CBZ and PC+4
	\item The MUX to the right of Data memory
\end{itemize}
\item Let's
make sure everyone is familiar with the operation of the
datapath. Trace the units involved in the following operations, using
markers on the overhead slide:
\begin{itemize}
\item {\tt ADD X1, X2, X3}
\item {\tt LDUR X1, [X2,\#200]}
\item {\tt CBZ X1, \#100}
\end{itemize}
	The students have three copies of this diagram in the course notes.
\end{itemize}
\fi\ENotes
\end{frame}

\begin{frame}[fragile]\ifnum\slides=0
\ifnum\Ans=1
\begin{columns}
    \column{0.6\textwidth}
        {\footnotesize 120: {\color{red}CBZ} {\color{orange}X1}, {\color{blue}\#100} 
  
  {\color{red}1011 0100} {\color{blue}0000 0000 0000 1100 100}{\color{orange}0 0001}}

{\footnotesize \textbf{0xB4000C81}}

  \includegraphics[width=\textwidth]{04-Single-Cycle-Processor-Implementation/figures/r-format.png}

  \PHFigure{!}{4in}{2in}{ARMFigures/Fig0417-crop}{Figure 4.17}
  
    \column{0.4\textwidth}

    \includegraphics[width=\textwidth]{04-Single-Cycle-Processor-Implementation/figures/compressedDecodeopcodes.png}

    \includegraphics[width=\textwidth]{04-Single-Cycle-Processor-Implementation/figures/compressedDecodeSignals.png}

    \vspace{5cm}

\end{columns}
\fi

\ifnum\Ans=0
{\tt 120: CBZ X1, \#100}

\PHFigure{!}{4in}{2in}{ARMFigures/Fig0417-crop}{Figure 4.17}
\fi
  \else
  \STitle{Datapath with Control}
\fi

\BNotes\ifnum\Notes=1
\begin{itemize}
\item Things to note on this slide:
\begin{itemize}
	\item The instruction bits are split off and sent to the
		relevant hardware
	\item The dual control unit
	\item The MUX in front of Read register 2
	\item The branching hardware and the mux that chooses between
		CBZ and PC+4
	\item The MUX to the right of Data memory
\end{itemize}
\item Let's
make sure everyone is familiar with the operation of the
datapath. Trace the units involved in the following operations, using
markers on the overhead slide:
\begin{itemize}
\item {\tt ADD X1, X2, X3}
\item {\tt LDUR X1, [X2,\#200]}
\item {\tt CNZ X1, \#100}
\end{itemize}
	The students have three copies of this diagram in the course notes.
\end{itemize}
\fi\ENotes
\end{frame}

% \begin{frame}[fragile]
% \Title{Meaning of Signals in Single-Cycle Datapath}
% 	\begin{center}
% 	\begin{tabular}{|l|l|l|}\hline
% 	Signal & Signal=0 & Signal=1 \\ \hline
% 	Reg2Loc & 20--16 & 4--0 \\
% 	Branch* & no branch & Branch \\
% 	MemRead & no effect & memory read \\
% 	MemToReg & reg write from ALU & reg write from memory \\ 
% 	MemWrite & no effect & memory written \\
% 	ALUSrc & ALU B input from reg & immediate from instruction \\
% 	RegWrite & no effect & register written \\ \hline
% 	\end{tabular}
% 	\end{center}
% * Branch is ANDed with Zero from ALU to get PCsrc

% (Full version in Figure 4.16 of text.)
% \BNotes\ifnum\Notes=1
% \begin{itemize}
% \item Go over the various control lines, using this slide and the previous
% one, and make sure the meaning of each
% one is clear, since this table has been abbreviated to make it fit on
% the slide.
% \item Note that 20--16 and 4--0 refer to bits of the instruction that are
%   used as the second register to read from the register file.
% \end{itemize}
% \fi\ENotes
% \end{frame}
%%%%%%%%%%%end of a lecture 

% for comprehensive
\newpage
\begin{frame}[fragile]
\STitle{Overview of Single-Cycle Control}
\includegraphics[width=0.95\textwidth]{04-Single-Cycle-Processor-Implementation/figures/scp-control-output-opcode-expanded.png}
% \Figure{!}{1in}{1.7in}{/figures/sc-control-output-opcode}
\begin{itemize}
\item Could be done in one level
\item Multiple levels of control are conceptually simpler
\item Smaller control units may also be faster
\end{itemize}
\BNotes\ifnum\Notes=1
\begin{itemize}
\item
Note that each of the ovals is a combinational circuit implementing
some Boolean function. The task is to specify the function and figure
out how to compute it.
\item
The main control takes as input the
opcode and produces the ALUop bits (among others); the ALU control
takes as input the ALUop bits and the function field and produces the
ALU control output, this is also the input for the ALU, called ALU operation. 
This includes Ainvert, Binvert, Operation 1 and Operation 0. 
updated image so that input to ALU Control is not Funct (from MIPS) but instead Opcode from ARM
Note the "11 (3)".  While the authors pass the full 11 opcode bits to the ALUControl, you really only need 3-bits here as we'll see shortly.
\item
The main text of the book presents the tables but does not go into
detail on the implementation; that is done in section C.2, which the
students should read.
\end{itemize}
\fi\ENotes
\end{frame}

\begin{frame}[fragile]
\Title{Designing Single-Cycle ALU Control}
Mapping of ALU Control input to output. 
{\footnotesize
\begin{itemize}
    \item Inputs to ALU Control: Opcode and 2 bit ALUop from Main Control
    \item Output from ALU Control: 4 bit ALU operation, same as ALU Operation for 1-bit ALU {\tiny{(designed earlier) }}
\end{itemize}
}
{\footnotesize
	\begin{center}
	\begin{tabular}{|c|c|c|c|c|}\hline
	Instruction & ALUop & Opcode &  ALU
                       & ALU \\
	  &  &  &  action & Control output\\\hline
	 {\tt LDUR} & 00 & xxxxxx xxxxx & add & 0010 \\
	 {\tt STUR} & 00 & xxxxxx xxxxx & add & 0010 \\
	 {\tt CBZ} & 01 & xxxxxx xxxxx & pass b & 0011 \\
\hline
	 {\tt ADD} & 10 & 100010 11000 & add & 0010 \\
	 {\tt SUB} & 10 & 110010 11000 & subtract & 0110 \\
	 {\tt AND} & 10 & 100010 10000 & AND & 0000 \\
	 {\tt ORR} & 10 & 101010 10000 & OR & 0001 \\\hline
	\end{tabular}
	\end{center}
}
Last four instructions use Opcode bits to determine ALU action
{\footnotesize
	\begin{center}
		\SizeD
		\begin{tabular}{cl}
			ALUOp & Operation\\
			\hline
			00 & Add\\
			01 & pass b\\
			10 & R-format\\
			% {\color{red}11} & {\color{red}Subtract}
		\end{tabular}
	\end{center}}
\BNotes\ifnum\Notes=1
\begin{itemize}
\item

	The ALU table is on page 272 of the text.

	The ALU in the book implements NOR; we don't need NOR for the
	subset of ARM instructions that we've implemented, so the
	left most bit of the ALU ctrl output is always 0.

\item The opcode and function field settings are specified by the
	instruction set designer; the ALU control input settings were
	specified in the design we did earlier. We choose the ALUop
	intermediate code now.

\item The ALUOp table was created to achieve the effects that we want.

	The ALUOp is a signal from the Main Control unit to the ALU Control unit,
		telling the ALU Control unit how to generate it's signal to
		the ALU.  Add, pass b, and subtract are clear; the R-format
		requires the ALU Control unit to look at the opcode and figure
		out what signal to send to the ALU.

	Subtract isn't used in the basic single cycle computer or in the textbook, but we'll assume that it's
		available for this class.  You need it for \texttt{SUBI} for example,
		although adding \texttt{SUBI} would require changing the control unit that we design
		here.
\end{itemize}
\fi\ENotes
\end{frame}


\begin{frame}{Recall the Opcodes}
     \begin{center}
  \begin{tabular}{llc}
    Instruction~~ & Opcode & Format\\
    \hline
    % B    & \texttt{0001\,01} {\tiny{($5_{10}$)}} & B-format\\
    LDUR & \texttt{1111\,1000\,010} {\tiny{($1986_{10}$)}} & D-format\\
     STUR & \texttt{1111\,1000\,000} {\tiny{($1984_{10}$)}} & D-format\\
      CBZ  & \texttt{1011\,0100} {\tiny{($180_{10}$)}}& CB-format\\
    CBNZ & \texttt{1011\,0101} {\tiny{($181_{10}$)}}& CB-format\\
    \hline
    ADD  & \texttt{1000\,1011\,000} {\tiny{($1112_{10}$)}}~~ & R-format\\
    % ADDI & \texttt{1001\,0001\,00} {\tiny{($580_{10}$)}} & I-format\\
    SUB  & \texttt{1100\,1011\,000} {\tiny{($1624_{10}$)}} & R-format\\
    % SUBI & \texttt{1101\,0001\,00} {\tiny{($836_{10}$)}} & I-format\\
   {\tt AND} & \texttt{1000\,1010\,000} & R-Format\\
	 {\tt ORR} & \texttt{1010\,1010\,000} & R-Format\\\hline
  
  \end{tabular}
  \end{center}
\end{frame}

\begin{frame}[fragile]
  \Title{Designing ALU Control}
	\begin{center}
 \resizebox{\textwidth}{!}{
	\begin{tabular}{|c|cc|ccccccccccc|c||cccc|}\hline
	Instruction & \multicolumn{2}{|c|}{\bf ALUop} & \multicolumn{11}{|c|}{\bf Opcode} &  ALU
                          & \multicolumn{4}{|c|}{Operation} \\
	  & 1&0 & x&{\SizeD 30}& {\SizeD 29}&x&x&x& ~x&{\SizeD 24}&x&x&x &  action & 3&2&1&0 \\\hline
	 {\tt LDUR} & 0&0 & x&x&x&x&x&x& ~x&x&x&x&x & add & 0&0&1&0 \\
	 {\tt STUR} & 0&0 & x&x&x&x&x&x& ~x&x&x&x&x & add & 0&0&1&0 \\
	 {\tt CBZ} & 0&1 & x&x&x&x&x&x& ~x&x&x&x&x & pass b & 0&0&1&1 \\
\hline
	 {\tt ADD} & 1&0 & c&0&0&c&c&c& ~c&1&c&c&c & add & 0&0&1&0 \\
	 {\tt SUB} & 1&0 & c&1&0&c&c&c& ~c&1&c&c&c & subtract & 0&1&1&0 \\
	 {\tt AND} & 1&0 & c&0&0&c&c&c& ~c&0&c&c&c & AND & 0&0&0&0 \\
	 {\tt ORR} & 1&0 & c&0&1&c&c&c& ~c&0&c&c&c & OR & 0&0&0&1 \\\hline
	\end{tabular}}
	\end{center}
\begin{itemize}
\item  Remove non-varying Opcode bit positions (c in table) for ALUop1=1
\item Split ALU control output ALU Operation into Operation3,Operation2, Operation1, Operation0.

	\hl{(Operation3=0 for our subset of ARM)}
\end{itemize}

\BNotes\ifnum\Notes=1
\begin{itemize}
\item This is NOT in the text!

	\item Mention where Don't Cares come from (ie, ALUop=11 isn't used;
		many opcode bits are redudent information).
	      \item Show how to do full sum-of-products for Operation0.
              \item For complete language, table is bigger and minimized by
                machine.  For this example, we can do it by hand
	\begin{itemize}
		\item Operation0 = ALUop0 + ALUop1 Opcode29
		\item Operation1 = $\Bar{\mbox{ALUop1}}+$ALUop1 Opcode24
		\item Operation2 = ALUop1 Opcode30
	\end{itemize}
		Note that we can't use the Don't Cares for ALUop=00,01;
		ie, we need to ALUop1 ANDed with the Opcode bits in
		these formulas because the Opcodes for LDUR,STUR,CBZ might
		be 1.
\end{itemize}
\fi\ENotes
\end{frame}


%-----------------------------------------------------
% \begin{frame}[fragile]\frametitle{ALU Control Boolean Equations (Part 1)}
% We can write the Boolean equations for the ALU Operation output bits, which are denoted by: Op3, Op2, Op1, Op0.

% \begin{center}
% \begin{tabular}{c|cc|ccc|c}
% Instruction   &  \multicolumn{5}{|c|}{Input} & \multicolumn{1}{|c}{Output}\\\hline
%               & \multicolumn{2}{|c|}{ALUOp} &  \multicolumn{3}{|c|}{Opcode} & \makecell{ALU \\ Operation}\\
%  & 1 & 0 & 30 & 29 & 24 & \\\hline
% load or store & 0 & 0 &  X & X & X & \texttt{0010}\\
% CBZ           & 0 & 1 &  X & X & X & \texttt{0011}\\
% R-format (ADD)& 1 & 0 &  0 & 0 & 1 & \texttt{0010}\\
% R-format (SUB)& 1 & 0 &  1 & 0 & 1 & \texttt{0110}\\
% R-format (AND)& 1 & 0 &  0 & 0 & 0 & \texttt{0000}\\
% R-format (ORR)& 1 & 0 &  0 & 1 & 0 & \texttt{0001}\\
% \end{tabular}
% \end{center}
% The formulas are found ``by inspection", see what input bits need to be 1 for the output bit to be 1. Software is also available to do this.

% \end{frame}

%-----------------------------------------------------
% \begin{frame}[fragile]\frametitle{ALU Control Boolean Equations (\texttt{Op3})}
% $Op3$ is always 0.
% $$Op3=0\;.$$
% \begin{center}
% \begin{tabular}{c|cc|ccc|c}
% Instruction   &  \multicolumn{5}{|c|}{Input} & \multicolumn{1}{|c}{Output}\\\hline
%               & \multicolumn{2}{|c|}{ALUOp} &  \multicolumn{3}{|c|}{Opcode} & Op[3]\\
%  & 1 & 0 & 30 & 29 & 24 & \\\hline
% load or store & 0 & 0 &  X & X & X & \texttt{0}\\
% CBZ           & 0 & 1 &  X & X & X & \texttt{0}\\
% R-format (ADD)& 1 & 0 &  0 & 0 & 1 & \texttt{0}\\
% R-format (SUB)& 1 & 0 &  1 & 0 & 1 & \texttt{0}\\
% R-format (AND)& 1 & 0 &  0 & 0 & 0 & \texttt{0}\\
% R-format (ORR)& 1 & 0 &  0 & 1 & 0 & \texttt{0}\\
% \end{tabular}
% \end{center}


% \end{frame}
%-----------------------------------------------------
\begin{frame}[fragile]\frametitle{ALU Control Boolean Equations for ALU Control Output - \texttt{Operation2}}
% Look at where $Operation2=1$, there is just one row. We need $ALUOp1=Opcode30=1$. Row 3 shows that $Operation2=1$ does not require $Opcode24=1$.

$$Operation2=ALUOp1 \cdot Opcode30$$
\begin{center}
\begin{tabular}{c|c|cc|ccc|c}
Row&Instruction   &  \multicolumn{5}{|c|}{Input} & \multicolumn{1}{|c}{Output}\\\hline
  &            & \multicolumn{2}{|c|}{ALUOp} &  \multicolumn{3}{|c|}{Opcode} & Operation2\\
& & 1 & 0 & 30 & 29 & 24 & \\\hline
1 &load or store & 0 & 0 &  X & X & X & \texttt{0}\\
2&CBZ           & 0 & 1 &  X & X & X & \texttt{0}\\
3&R-format (ADD)& 1 & 0 &  0 & 0 & 1 & \texttt{0}\footnote{Row 3 shows that $Operation2=1$ does not require $Opcode24=1$.}\\
4&R-format (SUB)& 1 & 0 &  1 & 0 & 1 & \texttt{1}\\
5&R-format (AND)& 1 & 0 &  0 & 0 & 0 & \texttt{0}\\
6&R-format (ORR)& 1 & 0 &  0 & 1 & 0 & \texttt{0}\\
\end{tabular}
\end{center}

\end{frame}
%-----------------------------------------------------
\begin{frame}[fragile]\frametitle{ALU Control Boolean Equations for ALU Control Output - \texttt{Operation1}}
% Look at where $Op[1]=1$. Row 1 and row 2 shows we need $ALUOp[1]=0$. Row 3 and 4 shows we need $Opcode[24]=1$, $ALUOp[1]=1$ is not needed since in row 4 and 5, this bit does not ensure  Op[1]=1.
$$Operation1= \overline{ALUOp1} + ALUOp1 \cdot Opcode24$$
\begin{center}
{\small
\begin{tabular}{c|c|cc|ccc|c}
Row&Instruction   &  \multicolumn{5}{|c|}{Input} & \multicolumn{1}{|c}{Output}\\\hline
     &         & \multicolumn{2}{|c|}{ALUOp} &  \multicolumn{3}{|c|}{Opcode} & Operation1\\
& & 1 & 0 & 30 & 29 & 24 & \\\hline
1&load or store & 0 & 0 &  X & X & X & \texttt{1}\\
2&CBZ           & 0 & 1 &  X & X & X & \texttt{1}\\
3&R-format (ADD)& 1 & 0 &  0 & 0 & 1 & \texttt{1}\\
4&R-format (SUB)& 1 & 0 &  1 & 0 & 1 & \texttt{1}\\
5&R-format (AND)& 1 & 0 &  0 & 0 & 0 & \texttt{0}\\
6&R-format (ORR)& 1 & 0 &  0 & 1 & 0 & \texttt{0}\\
\end{tabular}
}
\end{center}
\begin{tcolorbox}[enhanced,attach boxed title to top center={yshift=-3mm,yshifttext=-1mm},
  colback=blue!5!white,colframe=blue!75!black,colbacktitle=blue!80!black,
  title=Think About It,fonttitle=\bfseries,
  boxed title style={size=small,colframe=red!50!black} ]
  Can we simplify $Operation1= \overline{ALUOp1} + Opcode24$? 

  \ifnum\Ans=1{\color{red}No. ALU Control does not look at opcode bits for other instructions such as LDUR, STUR, CBZ and CBNZ. And using don't cares may lead to incorrect conclusions}\fi
  \end{tcolorbox}

\end{frame}
%-----------------------------------------------------

%%%zhk not using this slide - added information to previous slide
%-----------------------------------------------------
% \ifnum\Ans=1
% % \begin{frame}[fragile]\frametitle{ALU Control Boolean Equations for ALU Control Output - \texttt{Operation1} - Solution}
% % No we will not simplify since ALU Control, only refers to instruction opcode when ALUOp is 10, therefore, ALU Control does not look at opcode bits for other instructions such as LDUR, STUR, CBZ and CBNZ.And using don't cares may lead to incorrect conclusions:
% % % No we cannot simplify to $Operation1 = \overline{ALUOp1} + Opcode24$. Since based on the truth table with the opcode bits for instructions LDUR, STUR, CBZ, CBNZ, we notice that 
% % % $Opcode24 = 1$ for CBNZ instruction, therefore, $Operation1 = \overline{ALUOp1} + Opcode24 = 1$,when $ALUOp1=0$ and $Opcode24=1$ for CBNZ instruction. 
% % \begin{center}
% % \begin{tabular}{c|c|cc|ccc|c}
% % Row&Instruction   &  \multicolumn{5}{|c|}{Input} & \multicolumn{1}{|c}{Output}\\\hline
% %      &         & \multicolumn{2}{|c|}{ALUOp} &  \multicolumn{3}{|c|}{Opcode} & Operation1\\
% % & & 1 & 0 & 30 & 29 & 24 & \\\hline
% % 1&load or store & 0 & 0 &  {\color{red}1} & {\color{red}1} & {\color{red}0} & \texttt{1}\\
% % 2&CBZ/CBNZ           & 0 & 1 &  {\color{red}0} & {\color{red}1} & {\color{red}0/1\footnote{for CBNZ Opcode24 is 1}} & \texttt{1}\\
% % 3&R-format (ADD)& 1 & 0 &  0 & 0 & 1 & \texttt{1}\\
% % 4&R-format (SUB)& 1 & 0 &  1 & 0 & 1 & \texttt{1}\\
% % 5&R-format (AND)& 1 & 0 &  0 & 0 & 0 & \texttt{0}\\
% % 6&R-format (ORR)& 1 & 0 &  0 & 1 & 0 & \texttt{0}\\
% % \end{tabular}
% % \end{center}


% % \end{frame}
% \fi
% %-----------------------------------------------------



\begin{frame}[fragile]\frametitle{ALU Control Boolean Equations for ALU Control Output - \texttt{Operation0}}
% Look at the rows were $Operation0=1$ and see which inputs needs to be 1. In row 2, we need $ALUOp[0]=1$, in row 6 $ALUOp[1]$ and $Opcode[29]$ needs to be 1.
$$Operation0=ALUOp0 + ALUOp1 \cdot Opcode29$$
\begin{center}
\begin{tabular}{c|cc|ccc|c}
Instruction   &  \multicolumn{5}{|c|}{Input} & \multicolumn{1}{|c}{Output}\\\hline
              & \multicolumn{2}{|c|}{ALUOp} &  \multicolumn{3}{|c|}{Opcode} & Operation0\\
 & 1 & 0 & 30 & 29 & 24 & \\\hline
load or store & 0 & 0 &  X & X & X & \texttt{0}\\
CBZ           & 0 & 1 &  X & X & X & \texttt{1}\\
R-format (ADD)& 1 & 0 &  0 & 0 & 1 & \texttt{0}\\
R-format (SUB)& 1 & 0 &  1 & 0 & 1 & \texttt{0}\\
R-format (AND)& 1 & 0 &  0 & 0 & 0 & \texttt{0}\\
R-format (ORR)& 1 & 0 &  0 & 1 & 0 & \texttt{1}\\
\end{tabular}
\end{center}

\end{frame}
%-----------------------------------------------------
\begin{frame}[fragile]\frametitle{ALU Control Circuit}

 \begin{tcolorbox}[enhanced,attach boxed title to top center={yshift=-3mm,yshifttext=-1mm},
  colback=red!5!white,colframe=red!75!black,colbacktitle=red!80!black,
  title=Try this,fonttitle=\bfseries,
  boxed title style={size=small,colframe=red!50!black} ]
Given the following three Boolean equations for the output from the ALU Control:
% Operation3&=0 \\
\begin{align*}
Operation2&=ALUOp1 \cdot Opcode30 \\
Operation1&= \overline{ALUOp1} + ALUOp1 \cdot Opcode24\\
Operation0&=ALUOp0 + ALUOp1 \cdot Opcode29 \\
\end{align*}
Can you implement the circuits?
  \end{tcolorbox}

\end{frame} 

%-----------------------------------------------------

%-----------------------------------------------------
\ifnum\Ans=1
\begin{frame}{Solution - ALU Control Circuit}

A sample solution\footnote{Image by Sherlock Yang}:
\includegraphics[height=2.8in]{04-Single-Cycle-Processor-Implementation/figures/alu-control-circuit.png}

\end{frame}
\fi

\begin{frame}[fragile]
\STitle{Overview of Single-Cycle Control}
\includegraphics[width=0.95\textwidth]{04-Single-Cycle-Processor-Implementation/figures/scp-control-output-opcode-expanded.png}
% \Figure{!}{1in}{1.7in}{/figures/sc-control-output-opcode}
\begin{itemize}
\item Objective to build main Control Unit for Single Cycle Processor
\end{itemize}
\end{frame}

% for comprehensive
\newpage
\begin{frame}{Trace the Main Control for Instructions}
    \PHFigure{!}{1.5in}{1.5in}{ARMFigures/Fig0417-crop}{Figure 4.17}
\begin{center}
{\small
\resizebox{\textwidth}{!}{
	\begin{tabular}{c|c|c|c|c|c|c|c|c|c}
 \textbf{Input} & \multicolumn{9}{c}{\textbf{Output}}\\
 \hline\hline
	Instruction & Reg2 & ALU & Mem & Reg & Mem &
	Mem & Brch & ALU & ALU \\ 
	Type/Name     & Loc & Src & ToReg & Write & Read &
	Write &      & op1       & op0   \\ 
\hline\hline
	R-format   & 0 & 0 & 0 & 1 & 0 & 0 & 0 & 1 & 0 \\ \hline
	{\tt LDUR} & X & 1 & 1 & 1 & 1 & 0 & 0 & 0 & 0 \\ \hline
	{\tt STUR} & 1 & 1 & X & 0 & 0 & 1 & 0 & 0 & 0 \\ \hline
	{\tt CBZ}  & 1 & 0 & X & 0 & 0 & 0 & 1 & 0 & 1 \\ 
	\end{tabular}}}
        \end{center}
    
\end{frame}


\begin{frame}[fragile]
\STitle{Implementing Main Control Output Functions}
\begin{center}
\resizebox{\textwidth}{!}{
	\begin{tabular}{c|c|c|c|c|c|c|c|c|c}
 \textbf{Input} & \multicolumn{9}{c}{\textbf{Output}}\\
 \hline\hline
	Instruction & Reg2 & ALU & Mem & Reg & Mem &
	Mem & Brch & ALU & ALU \\ 
	Type/Name     & Loc & Src & ToReg & Write & Read &
	Write &      & op1       & op0   \\ 
\hline\hline
	R-format   & 0 & 0 & 0 & 1 & 0 & 0 & 0 & 1 & 0 \\ \hline
	{\tt LDUR} & X & 1 & 1 & 1 & 1 & 0 & 0 & 0 & 0 \\ \hline
	{\tt STUR} & 1 & 1 & X & 0 & 0 & 1 & 0 & 0 & 0 \\ \hline
	{\tt CBZ}  & 1 & 0 & X & 0 & 0 & 0 & 1 & 0 & 1 \\ 
	\end{tabular}}
        \smallskip
      	\end{center}  
       \begin{tcolorbox}[enhanced,attach boxed title to top center={yshift=-3mm,yshifttext=-1mm},
  colback=green!5!white,colframe=green!75!black,colbacktitle=green!80!black,
  title=Remember It,fonttitle=\bfseries,
  boxed title style={size=small,colframe=green!50!black} ]
{\footnotesize 
{\tt MemRead} is \textbf{never} a don’t care (\textbf{X}).
\begin{itemize}
\item  If the address is not a multiple of 8 $\rightarrow$ bus error $\rightarrow$ program crashes.
\item Even if it is a multiple of 8, an in-advertant read could cause cache issues. {\tiny{More on this in Module 06.}}
% : e.g. in LRU a memory read can cause an entry to be more recently accessed when it was not meant to be.
\item If the address is not in a range that the program can access$\rightarrow$ segmentation fault.
\item Reading memory is very slow, dont read unnecessarily.
\end{itemize}
}
  \end{tcolorbox}


\BNotes\ifnum\Notes=1
\begin{itemize}
\item It should be clear why these signals are the way they are, by
looking at the diagram of the datapath. Go over each of them using
that diagram.

\item Consider drawing table on the board, and have datapath on slide
and show how to fill in table.  It will also be helpful to have the
table on the board when you go to the next slide.

\item The comment about MemRead never don't care: if the address isn't a multiple of 4, then your computer generates a bus error and your program crashes.  Even if it's a multiple of 4, an inadvertant read could cause cache issues.
  
\item The ``Type'' column in the top table is the input; the remaining
  columns are outputs; to fully specify the truth table, we need the
  opcodes for the types, and thus the bottom table.
  
\item In the bottom table, middle column, CBZ has don't cares because
  it only has a 6-bit opcode. R-format has don't cares to give ADD,
  SUB, AND, ORR a ``single opcode''; this row could instead be
  expanded into four rows, one for each.
  
\item In the bottom table, right column, we have removed columns that
  don't vary.  This will allow us to make the circuit on the next slide,
\end{itemize}
\fi\ENotes
\end{frame}


% \begin{frame}{Recall, Observe the Opcodes}
%      \begin{center}
%      {\small
%   \begin{tabular}{llc}
%     Instruction~~ & Opcode & Format\\
%     \hline
%      ADD  & \texttt{1000\,1011\,000} {\tiny{($1112_{10}$)}}~~ & R-format\\
    
%     SUB  & \texttt{1100\,1011\,000} {\tiny{($1624_{10}$)}} & R-format\\
   
%    {\tt AND} & \texttt{1000\,1010\,000} & R-Format\\
% 	 {\tt ORR} & \texttt{1010\,1010\,000} & R-Format\\
%   \hline
%     LDUR & \texttt{1111\,1000\,010} {\tiny{($1986_{10}$)}} & D-format\\
%      STUR & \texttt{1111\,1000\,000} {\tiny{($1984_{10}$)}} & D-format\\
%       CBZ  & \texttt{1011\,0100} {\tiny{($180_{10}$)}}& CB-format\\
%     \hline
%   \end{tabular}}
%   \end{center}
%   \smallskip
  
%     \definecolor{gray}{rgb}{0.6,0.6,0.6}
%     %     \newcommand\gry[1]{{\color{gray}#1}}
%         \begin{center}
%               \begin{tabular}{ccc}
% 	Type & Binary Opcode & Simplified Opcode\\
% \hline
% % 	R-format &   1xx\,0101\,x000 & \gry{x}x\gry{x}\,0101\,\gry{x}\gry{x}0\gry{x}\\
% % 	{\tt LDUR} & 111\,1100\,0010 & \gry{x}1\gry{x}\,1100\,\gry{x}\gry{x}1\gry{x}\\
% % 	{\tt STUR} & 111\,1100\,0000 & \gry{x}1\gry{x}\,1100\,\gry{x}\gry{x}0\gry{x}\\
% % 	{\tt CBZ}&   101\,1010\,0xxx & \gry{x}0\gry{x}\,1010\,\gry{x}\gry{x}x\gry{x}
% \end{tabular}
% 	\end{center}
% \smallskip

% Simplified opcode bits: 30,28,27,26,25,22

% \bigskip
% Software can generate simplified circuit.
% \end{frame}



\begin{frame}[fragile]
\STitle{Recall, the Opcodes}
\begin{center}
 \begin{center}
     {\small
  \begin{tabular}{llc}
    Instruction~~ & Opcode & Format\\
    \hline
     ADD  & \texttt{1000\,1011\,000} {\tiny{($1112_{10}$)}}~~ & R-format\\
    
    SUB  & \texttt{1100\,1011\,000} {\tiny{($1624_{10}$)}} & R-format\\
   
  AND& \texttt{1000\,1010\,000} & R-Format\\
	 ORR & \texttt{1010\,1010\,000} & R-Format\\
  \hline
    LDUR & \texttt{1111\,1000\,010} {\tiny{($1986_{10}$)}} & D-format\\
     STUR & \texttt{1111\,1000\,000} {\tiny{($1984_{10}$)}} & D-format\\
      CBZ  & \texttt{1011\,0100} {\tiny{($180_{10}$)}}& CB-format\\
    \hline
  \end{tabular}}
  \end{center}

  \bigskip
        \definecolor{gray}{rgb}{0.6,0.6,0.6}
        \newcommand\gry[1]{{\color{gray}#1}}
        \bigskip
\begin{tabular}{ccc}
	Type & Binary Opcode & Simplified Opcode\\
\hline
	R-format &   1xx0\,101x\,000 & \gry{x}x\gry{x}\,0101\,\gry{x}\gry{x}0\gry{x}\\
	{\tt LDUR} & 1111\,1000\,010 & \gry{x}1\gry{x}\,1100\,\gry{x}\gry{x}1\gry{x}\\
	{\tt STUR} & 1111\,1000\,000 & \gry{x}1\gry{x}\,1100\,\gry{x}\gry{x}0\gry{x}\\
	{\tt CBZ}&   1011\,0100\,xxx & \gry{x}0\gry{x}\,1010\,\gry{x}\gry{x}x\gry{x}
\end{tabular}
	\end{center}
\smallskip

Simplified opcode bits: 30,28,27,26,25,22

\bigskip
Software generates simplified circuit

\end{frame}

% \begin{frame}{test}

%      \resizebox{\textwidth}{!}{
     
% 	\begin{tabular}{|c|ccccccccccc|}\\
%  \hline
% 	Instruction & \multicolumn{11}{|c|}{\bf Binary Opcode} \\
% 	  &{\SizeD 30}& {\SizeD 29}&{\SizeD 28}&{\SizeD 27}&{\SizeD 26}&{\SizeD 25}&{\SizeD 24}&{\SizeD 23}&{\SizeD 22}&{\SizeD 21}  \\
%    \hline
% \end{tabular}
% }
   
% \end{frame}


% \begin{frame}\frametitle{Main Control Simplified Inputs}
% The inputs can be further simplified:
% \begin{center}
% \begin{tabular}{c|c|c}
% Instruction &  Input       & Simplified Input\\\hline
%            & 11 bit Opcode & 6 bit Opcode \\ \hline
% R-format   & \texttt{1XX 0101 X000} & \texttt{\_X\_ 0101 \_\_0\_} \\\hline
% load word  & \texttt{111 1100 0010} & \texttt{\_1\_ 1100 \_\_1\_} \\\hline
% store word & \texttt{111 1100 0000} & \texttt{\_1\_ 1100 \_\_0\_} \\\hline
% CBZ        & \texttt{101 1010 0XXX} & \texttt{\_0\_ 1010 \_\_X\_} \\\hline
% \end{tabular}
% \end{center}
% MemRead is never don’t care.
% \begin{itemize}
% \item  If the address is not a multiple of 4, then your computer generates a bus error and your program crashes.
% \item Even if it is a multiple of 4, an inadvertant read could cause cache issues.
% \end{itemize}

% \end{frame}

%-----------------------------------------------------
\begin{frame}\frametitle{Main Control Simplified Truth Table}
The simplified inputs are the opcode bits at index  30, 28, 27, 26, 25, 22.
\hfill\break
The outputs are the control bits:
\begin{itemize}
\item Reg2Loc,ALUSrc,MemtoReg,RegWrite,MemRead,MemWrite,Branch
\item ALUOp1, ALUOp0
\end{itemize}
\begin{center}
\begin{tabular}{c|c|c|c}
Instruction &  Input & \multicolumn{1}{|c}{Output} \\\hline
           & 6 bit Opcode & Control Lines & ALUOp\\\hline
R-format   & \texttt{X01010} & \texttt{0001000} & \texttt{10} \\\hline
load word  & \texttt{111001} & \texttt{X111100} & \texttt{00} \\\hline
store word & \texttt{111000} & \texttt{11X0010} & \texttt{00} \\\hline
CBZ        & \texttt{01010X} & \texttt{10X0001} & \texttt{01} \\\hline
\end{tabular}
\end{center}
\end{frame}

%-----------------------------------------------------
\begin{frame}\frametitle{Main Control Circuit}

The 6 opcode bits are ANDed together, one AND gate for each instruction format. ``Don't Care" terms are \emph{not} ANDed.

\begin{figure}[H]
\centering
	{\includegraphics[scale=0.4]{04-Single-Cycle-Processor-Implementation/figures/main-control-in-and-bits}}
\end{figure}
\end{frame}

%-----------------------------------------------------
% \begin{frame}\frametitle{Main Control Circuit}

% The AND gate outputs, where the control bit is 1, are ORed together to give the output control bit.


% \begin{figure}[H]
% \centering
% 	{\includegraphics[scale=0.3]{figures/main-control-circuit-out}}
% \end{figure}
% \end{frame}


\iftrue
\begin{frame}[fragile]
\STitle{Two Level Main Control Circuit}
The AND gate outputs, where the control bit is 1, are ORed together to give the output control bit.
\Figure{!}{6.5in}{2.5in}{ARMFigures/mainControl}
\BNotes\ifnum\Notes=1
\begin{itemize}
  \item Not in text
\item Normally use circuit simplification, but straightforward to make
  two level circuit for this control unit
\end{itemize}
\fi\ENotes
\end{frame}
\fi


% %%%%%%%%%%%%%%%% beginning of timing analysis of single cycle datapath


% \begin{frame}[fragile]
% \STitle{Performance of Single Cycle Machines}
% \begin{itemize}
% \item Suppose memory units take 200 ps (picoseconds), ALUs 200 ps,
% register files 100 ps, no delay on other units
% \item Branch take 400 ps, CBZ take 500 ps, R-format instructions 600 ps,
% STUR 700 ps, LDUR 800 ps.
% \item Clock period must be increased to 800 ps or more
% \item Even worse when floating-point instructions are implemented
% \item Idea: use multicycle implementation
% and R format
% \end{itemize}
% \BNotes\ifnum\Notes=1
% \begin{itemize}
% \item Ask the students why the different types of instructions take the
%   times listed, or work it out on the board.
% \item See Figure 4.25 of the text.  The figure doesn't list branch, but
%   it's the same entry as CBZ except it doesn't use the register file.
%   \end{itemize}
% \fi\ENotes
% \end{frame}

% \input 04-Single-Cycle-Processor-Implementation/single-cycle-timing-f22

\begin{frame}[fragile]
\STitle{Modifying the datapath}
\begin{tcolorbox}[enhanced,attach boxed title to top center={yshift=-3mm,yshifttext=-1mm},
  colback=red!5!white,colframe=red!75!black,colbacktitle=red!80!black,
  title=Try this,fonttitle=\bfseries,
  boxed title style={size=small,colframe=red!50!black} ]
Modify the Single Cycle Datapath for Branch instructions. 

Assume that the control unit generates a special control signal \texttt{Unconditionalbranch} when it decodes a B-format instruction. 
  \end{tcolorbox}
% \begin{itemize}
	% \item Normally design complete datapath for all instructions together.
	% \item
 Various ways to modify datapath.  The following is one
		approach for adding a new assembly instruction:
	\begin{enumerate}
	\item Determine what datapath is needed for new command
	\item Check if any components in current datapath can be used
	\item Wire in components of new datapath into existing datapath
\begin{itemize}
    \item Probably requires new MUXes
\end{itemize}
		
	\item Add new control signals to Control units
	\item Adjust old control signals to account for new command
	\end{enumerate}
% \end{itemize}
\BNotes\ifnum\Notes=1
The book shows how to modify the data path to execute the JUMP instruction.
Students should read this in the textbook; on the next slide, we will add
{\tt jrel} to the data path.
\fi\ENotes
\end{frame}

\begin{frame}[fragile]
  \STitle{Solution: Adding the \texttt{Branch} command}
\includegraphics[scale=0.4]{04-Single-Cycle-Processor-Implementation/figures/add-uncond-branch-redbox}
{\small
\begin{itemize}
    \item Add the ``unconditional branch signal'', \texttt{Uncondbranch}.
    % OR with the original branch signal from the output of the AND gate.
\item Modify the Sign-extend unit since the branch instruction has a different format.
% We also need to modify the Sign-extend unit since the branch instruction has a different format. This is achieved by allowing the sign-extend controller to use the opcode from the instruction to sign-extend accordingly.
\end{itemize}}
% \PHFigure{!}{6.5in}{2.75in}{ARM04-Single-Cycle-Processor-Implementation/figures/Fig0417-crop}{Figure 4.17}
\BNotes\ifnum\Notes=1
\begin{itemize}
\item The textbook does this example; see page 282.

  Basically, you add a new control signal (Unconditionalbranch),
  and OR it with the PCsrc line (labeled in Figure 4.16 but not in Figure 4.17; basically, it's the control on the MUX whose output goes to the PC).
  That can be drawn on the slide.

  However, what can't be drawn is that the Sign Extend unit needs to be
  modified to handle the 26-bit offset of branch.

  Also, the modified datapath appears on the next slide, for adding jrel
  
\item You should also go over how the main control table will change
  (a new line for branch, with Reg2Loc=X, ALUSrc=1, MemToReg=X, RegWrite=MemRead=MemWrite=Branch=0, ALUop1=0, ALUop2=0.
  
\end{itemize}
\fi\ENotes
\end{frame}

\begin{frame}{Solution: Control Outputs for Unconditional Branch instruction}
\begin{itemize}
    \item Turn off register file: RegWrite=0. 
    \item Avoid any cache issue due to memory read (never a don't care): MemRead=0, MemWrite=0.
    \item Branch=X 
    % =0 to be safe, can be don't care.
    \item We are not reading register files or register 2 so Reg2Loc=X, 
    \item MemToReg=X. ALUSrc=X. ALUOp=XX.
\end{itemize}
    
\end{frame}



% \begin{frame}[fragile]
% \begin{itemize}
% 	\item Add {\tt BREL Rn} as an I-format command that performs

% 	                PC $\leftarrow$ PC + 4*Rn

% \PHFigure{!}{5in}{2.5in}{ARM04-Single-Cycle-Processor-Implementation/figures/Fig0423-crop}{Figure 4.23}

% \end{itemize}
% \BNotes\ifnum\Notes=1
% \begin{itemize}
% 	\item Do an example such as a I-format {\tt BREL Rn} which performs

% 		PC $\leftarrow$ PC +  4*Rn

% 		Assume that the specified register is Rn.

% 	\item Draw a datapath for with hardware to perform just this operation.

% 		Use an adders and do the *4 with a shift in wires
% 	\item Observe that the adder and the shift unit are already
% 		appropriately wired in the datapath - you just need to
% 		provide a new input to the shift unit.

% 	\item You will need a new MUX to select the input to the shift
% 		unit.  You will need a new control line from the Control
% 		unit to select this MUX.
% 	\item Finally, give the control signal table for the new command.
% \end{itemize}
% \fi\ENotes
% \end{frame}

% for comprehensive
\newpage
\begin{frame}[fragile]\frametitle{Example: Modifying the Single Cycle Datapath}

\begin{tcolorbox}[enhanced,attach boxed title to top center={yshift=-3mm,yshifttext=-1mm},
  colback=red!5!white,colframe=red!75!black,colbacktitle=red!80!black,
  title=Try this,fonttitle=\bfseries,
  boxed title style={size=small,colframe=red!50!black} ]
Add the Branch Relative (BREL) instruction to the Single Cycle datapath.

Assume that the control unit generates a control signal \texttt{BREL} when it decodes this {\tt BREL} instruction.
\hfill\break

This instruction is I-format, it branches based on a register:
\begin{verbatim}
BREL Rn
\end{verbatim}
Sets PC to \texttt{PC+4*Rn}. \texttt{Rn} is in bits 9-5.
\begin{figure}[H]
\centering
	{\includegraphics[scale=0.3]{04-Single-Cycle-Processor-Implementation/figures/I-format}}
\end{figure}
\end{tcolorbox}
% ** Try this yourself. The process is explained in the following slides.
\end{frame}

%-----------------------------------------------------
\ifnum\Ans=1{\color{red}
\begin{frame}\frametitle{Solution: Add Branch Relative (BREL) Instruction}
\begin{enumerate}
    % \item Add a control bit BREL. 
    \item Add MUX that selects input to ``Shift left 2" unit either: 
    \begin{itemize}
        \item Read data 1 or
        \item Sign-extend.
    \end{itemize}
    \item  Use control signal BREL as select line for this MUX:
    \begin{itemize}
        \item If BREL=1, register Rn is input tp ``Shift left 2" unit,
        \item otherwise, no change to datapath 
    \end{itemize}
    \item Add BREL to OR gated to select PCSrc.
\end{enumerate}

\begin{figure}[H]
\centering
	{\includegraphics[scale=0.33]{04-Single-Cycle-Processor-Implementation/figures/brel-idea-redbox}}
\end{figure}
\end{frame}

%-----------------------------------------------------
% \begin{frame}\frametitle{Add Branch Relative (BREL) Instruction}
% The BREL signal also goes into the OR gated added previously.
% \begin{figure}[H]
% \centering
% 	{\includegraphics[scale=0.4]{04-Single-Cycle-Processor-Implementation/figures/brel-idea-redbox}}
% \end{figure}
% \end{frame}
%-----------------------------------------------------
\begin{frame}\frametitle{Solution: Control Signals For BREL Instruction}
\begin{itemize}
\item First, BREL=1, Branch=Uncondbranch=0, because BREL is used to signal we are branching instead.
\item RegWrite=0, MemRead=MemWrite=0.
\item Don't cares:
\begin{itemize}
\item Reg2Loc=X, because we are not using "Read Register 2".
\item MemtoReg=X, because are not writing back to the register file.
\item ALUOp=XX, ALUSrc=X, because we are not using the ALU.
\end{itemize}
\end{itemize}
Remember to indicate that BREL=0 for all other instructions.
%\begin{figure}[H]
%\centering
%	{\includegraphics[scale=0.4]{figures/brel-idea-redbox}}
%\end{figure}
\end{frame}
}\fi
% \begin{frame}\frametitle{}
%  \begin{tcolorbox}[enhanced,attach boxed title to top center={yshift=-3mm,yshifttext=-1mm},
%   colback=green!5!white,colframe=green!75!black,colbacktitle=green!80!black,
%   title=Remember It,fonttitle=\bfseries,
%   boxed title style={size=small,colframe=green!50!black} ]
% MemRead problems if it is a don't care:
% \begin{itemize}
% \item If the address is not a multiple of 8, the computer gives a bus error.
% \item If the address is not in a range that our program can access, the computer gives a segmentation fault.
% \item Reading memory is very slow.
% \item Can cause cache consistency issues
% \end{itemize}
% \end{tcolorbox}
% \end{frame}


\begin{frame}[fragile]
\STitle{Conclusion}
 \underline{\textbf{Lecture Summary}}
 \begin{itemize}
 \item Control unit implementation
 \begin{tcolorbox}[enhanced,attach boxed title to top center={yshift=-3mm,yshifttext=-1mm},
  colback=green!5!white,colframe=green!75!black,colbacktitle=green!80!black,
  title=Remember It,fonttitle=\bfseries,
  boxed title style={size=small,colframe=green!50!black} ]
{\footnotesize {\tt MemRead} is \textbf{never} a don't care:}
% {\footnotesize
% \begin{itemize}
% \item If the address is not a multiple of 8, the computer gives a bus error.

% \item Can cause cache consistency issues
% \end{itemize}
% }
\end{tcolorbox}
\end{itemize}

 % \end{itemize}
 \underline{\textbf{Assigned Textbook Readings}}
\begin{itemize}
     \item \textbf{Read} 
     \begin{itemize}
   \item   Appendix C, Section C.2\footnote{caution: the control unit is for MIPS datapath not ARM datapath!}
     \end{itemize}
     \end{itemize}
    \underline{\textbf{Next Steps}}
    \begin{itemize}
     \item \textbf{Review} logic behind the control unit implementation. 
% \begin{itemize}
%     % \item Start thinking about the exercises in A3
%     % \item A3 will be released in two parts, first set of exercises focus on Sequential Circuits and FSM, second set of exercises on Data Representation
% \end{itemize}
\item \textbf{Attempt} questions in next week's tutorial. 
    \item \textbf{Ask} questions in office hours or the next tutorial.
 \end{itemize}

\end{frame}