
\setlength{\columnseprule}{1pt}
\def\columnseprulecolor{\color{blue}}



\begin{frame}[fragile]
\STitle{Using Gates in Logic Design}
\begin{itemize}
	\item Here are symbols for AND, OR, NOT gates, respectively
        \Figure{!}{0.2in}{.3in}{Figs/andornot}
	\item NOT often drawn as ``bubble'' on input or output
	\item AND, OR can be generalized to many inputs (useful)
	\item We can design using AND, OR, NOT, and optimize afterwards
\end{itemize}
% {\footnotesize
\begin{tcolorbox}[enhanced,attach boxed title to top center={yshift=-3mm,yshifttext=-1mm},
  colback=blue!5!white,colframe=blue!75!black,colbacktitle=blue!80!black,
  title=Think About It,fonttitle=\bfseries,
  boxed title style={size=small,colframe=red!50!black} ]
  {\footnotesize
    The inputs for the AND and OR logic gates can be increased arbitrarily. For example, consider the 3-input AND illsutrated below and its truth table, where $F=XYZ$.}
    % \bigskip
    \begin{multicols}{2}
    \input{02-circuit/02-combinational/circuit2}
    \columnbreak
  {\footnotesize
    \begin{tabular}{ccc|c||ccc|c|}
    \hline
        X & Y &Z & $F$ & X & Y &Z & $F$\\
        \hline
        0 & 0 &0 & 0 &1 & 0 &0 & 0 \\
         0 & 0 &1 & 0 &1 & 0 &1 & 0 \\
          0 & 1 &0 & 0 &1 & 1 &0 & 0 \\
 0 & 1 &1 & 0 &1 & 1 &1 & 1 \\
 \hline
     \end{tabular}
     }
    \end{multicols}
\end{tcolorbox}

\BNotes\ifnum\Notes=1
Instructor's Notes:
\begin{itemize}
\item
AND, OR, NOT are useful for us conceptually because they are
intuitive, and because we have the tools of Boolean algebra to help
us.
\item Show 3-input AND, with truth table.
\item Return to previous slide and draw circuit for simplified formula
  with AND and OR.  Keep it handy for next slide...
\end{itemize}
\fi\ENotes
\end{frame}

\begin{frame}[fragile]
    \STitle{Digital Logic Circuit Using Gates}
    Consider the boolean function, $F = XY + \Bar{Y}Z$
    \\
The circuit can be built using the digital AND and OR logic gates:
\bigskip
\input{02-circuit/02-combinational/circuit}
 \begin{tcolorbox}[enhanced,attach boxed title to top center={yshift=-3mm,yshifttext=-1mm},
  colback=green!5!white,colframe=green!75!black,colbacktitle=green!80!black,
  title=Remember It,fonttitle=\bfseries,
  boxed title style={size=small,colframe=green!50!black} ]
For \textbf{any} input $A$, its inverse $\Bar{A}$ is \textbf{always} available to use in a digital logic circuit!
  \end{tcolorbox}
\end{frame}

\begin{frame}[fragile]
\Title{Good Style in Circuit Drawing}
\begin{itemize}
	\item Assume all literals (inputs\footnote{signals} and their negations) are available
	\item Rectilinear wires, dots when wires split
	\item Do not draw spaghetti wires for inputs; instead, write
	each literal as needed

	\TwoFigure{!}{2.25in}{1in}{Figs/badEx}{Figs/goodEx}

	\centerline{\hfill harder to read \hfill preferred \hfill}

\end{itemize}
\BNotes\ifnum\Notes=1
Instructor's Notes:
\begin{itemize}
\item Draw a BAD example with non-rectilinear wires.
\item Points will be deducted for spaghetti wiring on assignments, at the
discretion of the marker. Note that the main goal is communication.
\end{itemize}
\fi\ENotes
\end{frame}

\begin{frame}[fragile]
  \Title{NAND and NOR}
  \begin{itemize}
  \item In practice, \hl{logic minimization} software works with
	\hl{NAND} or NOR gates, or at \hl{transistor} level

  \item Here are symbols for NAND, NOR:
        \Figure{!}{1in}{.5in}{Figs/nandnor}

	\begin{center}
	\begin{tabular}{|c|c|c|c|c|c|c|}\cline{1-3}\cline{5-7}
	\multicolumn{3}{|c|}{NAND}& &\multicolumn{3}{|c|}{NOR} \\\cline{1-3}\cline{5-7}
	A & B & \small$\overline{AB}$ &~~~~~ & A & B & \small$\overline{A+B}$  \\\cline{1-3}\cline{5-7}
	0 & 0 & 1   & & 0 & 0 & 1   \\\cline{1-3}\cline{5-7}
	0 & 1 & 1   & & 0 & 1 & 0   \\\cline{1-3}\cline{5-7}
	1 & 0 & 1   & & 1 & 0 & 0 \\\cline{1-3}\cline{5-7}
	1 & 1 & 0   & & 1 & 1 & 0 \\\cline{1-3}\cline{5-7}
	\end{tabular}
	\end{center}
    % \item Two level NAND circuit
    % \begin{itemize}
    %     \item AND with inverted output is NAND
    %     \item OR with inverted input in NAND
    % \end{itemize}
    \end{itemize}
\end{frame}
\BNotes\ifnum\Notes=1
\begin{frame}[fragile]
Instructor's Notes:
\begin{itemize}
\item
AND, OR, NOT are useful for us conceptually because they are
intuitive, and because we have the tools of Boolean algebra to help
us. But as we will see, because of the way transistors work, it takes 2
transistors to create a NOT gate, 4 to create a NAND gate, and 6 to
create an AND gate (by negating the NAND gate). Since NAND is
universal (any Boolean function can be expressed using only this
operator: exercise), it makes sense to use just NAND gates in designs,
or to go directly to the transistor level.

\item 	Show both types of NANDs (AND with output complemented, OR with
	inputs complemented) and use DeMorgan as a quick proof of this.
\item ...use circuit you drew with AND and OR, and then put bubbles on the
	AND outputs and the OR inputs to make it all NANDs; the bubbles
	cancel, so it's the same circuit..

\end{itemize}
\end{frame}
\fi\ENotes

\begin{frame}[fragile]
    \STitle{NAND Gate and its equivalents}
    \begin{itemize}
        \item NAND is
        \begin{itemize}
        \item AND with inverted output
        \item OR with inverted inputs
        \item Proof: DeMorgan's Dual Rule $\overline{XY}=\bar{X}+\bar{Y}$
    \end{itemize}
     \resizebox{5cm}{3cm}{
    \input{02-circuit/02-combinational/circuit5}
    }
    \item NAND is a universal gate
    \begin{itemize}
        \item \hl{any Boolean function can be implemented using only NAND gate}
    \end{itemize}
    \item Two level NAND circuit
    \begin{itemize}
        \item Function can be defined using sum of product form
        \item Circuit can be drawn using NAND gates
    \end{itemize}
    
    \end{itemize}
   
\end{frame}

\begin{frame}[fragile]
    \STitle{Digital Logic Circuit Using NAND Gates}
    Recall, the circuit for $F = XY + \Bar{Y}Z$ using AND and OR logic gates:
% \bigskip
\input{02-circuit/02-combinational/circuit}
% \bigskip
{\color{red}Circuit for $F$, using NAND gates only:}
\bigskip
\input{02-circuit/02-combinational/circuit3}

\end{frame}

% \begin{frame}[fragile]
% \Title{Good Style in Circuit Drawing}
% \begin{itemize}
% 	\item Assume all literals (inputs\footnote{signals} and their negations) are available
% 	\item Rectilinear wires, dots when wires split
% 	\item Do not draw spaghetti wires for inputs; instead, write
% 	each literal as needed

% 	\TwoFigure{!}{2.25in}{1in}{Figs/badEx}{Figs/goodEx}

% 	\centerline{\hfill harder to read \hfill preferred \hfill}

% \end{itemize}
% \BNotes\ifnum\Notes=1
% Instructor's Notes:
% \begin{itemize}
% \item Draw a BAD example with non-rectilinear wires.
% \item Points will be deducted for spaghetti wiring on assignments, at the
% discretion of the marker. Note that the main goal is communication.
% \end{itemize}
% \fi\ENotes
% \end{frame}


% \begin{frame}{Example}
%     \begin{tcolorbox}[enhanced,attach boxed title to top center={yshift=-3mm,yshifttext=-1mm},
%   colback=red!5!white,colframe=red!75!black,colbacktitle=red!80!black,
%   title=Try this,fonttitle=\bfseries,
%   boxed title style={size=small,colframe=red!50!black} ]
%   Draw the combinational logic circuit that implements $F$ using only NAND gates:  
% \begin{figure}[H]
% 	\begin{center}
% 		\begin{tabular}{ccc|c}
% 		X&Y&Z & F\\
% 		\hline
% 		0&0&0 & 0\\
% 		0&0&1 & 0\\
% 0&1&0 &1\\
% 0&1&1 & 1\\
% \hline
% 1&0&0 & 0\\
% 1&0&1 & 0\\
% 1&1&0 & 1\\
% 1&1&1 & 0\\
% 		\end{tabular}
% 		\end{center}
% \end{figure}
% \end{tcolorbox}
% \end{frame}

% \begin{frame}{Solution: Derive Boolean expression for F}
%     \begin{itemize}
%         \item Derive the Boolean expression for F
%         \begin{itemize}
%             \item Sum of Products Form:\\
%             \small
%             $F = \bar{X}Y\bar{Z} + \bar{X}YZ + XY\bar{Z} $
%             \item Simplify the expression using Laws of Boolean Algebra:\\
%             \small
%             $F = \bar{X}Y\bar{Z} + \bar{X}YZ + XY\bar{Z} +\bar{X}Y\bar{Z}$\\
%             $F = \bar{X}Y(\bar{Z}+Z) + Y\bar{Z}(X + \bar{X})$\\
%             $F = \bar{X}Y+ Y\bar{Z}$\\ 
%             $F = Y(\bar{X}+\bar{Z})$
%         \item Product of Sums Form:\\
%         \small
%         % \begin{align}
%         %   F= (X+Y+Z)(X+Y+\bar{Z})(\bar{X}+Y+Z) 
%         %         (\bar{X}+Y+\bar{Z})(\bar{X}+\bar{Y}+\bar{Z})  
%         % \end{align}
%          $ F= (X+Y+Z)(X+Y+\bar{Z})(\bar{X}+Y+Z) 
%                  (\bar{X}+Y+\bar{Z})(\bar{X}+\bar{Y}+\bar{Z}) $
%         \end{itemize}
%         \item Draw the circuit using ANDs and ORs
%         \item Convert ANDs and ORs to NAND equivalent 
%     \end{itemize}
   
 
% \end{frame}
% \begin{frame}{Solution: Circuit design }
%     \begin{multicols}{2}
%     Design using ANDs and ORs for SOP or POS form of F

%     Note: $\bar{X}$ is represented as $!X$ in the circuit
%     \includegraphics[height=5cm]{02-circuit/02-combinational/figures/col1.png}
%     \columnbreak

%      Design using NAND equivalent or NAND for F

%          Note: $\bar{X}$ is represented as $!X$ in the circuit

%     \includegraphics[height=5cm]{02-circuit/02-combinational/figures/col2.png}
%     \end{multicols}
% \end{frame}

% \begin{frame}{Solution: Circuit design }
%     \begin{multicols}{2}
%     Design using ANDs and ORs for SOP or POS form of F

%     Note: $\bar{X}$ is represented as $!X$ in the circuit
%     \includegraphics[height=5cm]{02-circuit/02-combinational/figures/col1.png}
%     \columnbreak

%      Design using NAND equivalent or NAND for F

%          Note: $\bar{X}$ is represented as $!X$ in the circuit

%     \includegraphics[height=5cm]{02-circuit/02-combinational/figures/col2.png}
%     \end{multicols}
% \end{frame}

% \begin{frame}{test}
%     \resizebox{6cm}{8cm}{
%     \input circuit6
%     }
% \end{frame}

% \begin{frame}\frametitle{XOR Gate}
% The gate symbol and truth table for XOR
% \begin{multicols}{2}
% \includegraphics[width=5cm]{02-circuit/02-combinational/figures/xor.png}
% \columnbreak

% \begin{center}
%     \begin{tabular}{cc|c}
%          X & Y & F \\
%          \hline
%          0 & 0 & 0\\
%          \hline
%          0 & 1 & 1\\
%         \hline
%          1 & 0 & 1\\
%          \hline
%          1 & 1 & 0\\
%          \hline
%     \end{tabular}
% \end{center}

% \end{multicols}

% The XOR gate has the following mathematical symbol:
% $$F = X \oplus Y$$

% \end{frame}

% \begin{frame}\frametitle{XOR and Parity}
% In general, the output of XOR is 1 if the input has an odd number of 1's.

% \begin{align*}
% 1 \oplus 0 \oplus 1 = (1 \oplus 0) \oplus 1 = 1 \oplus 1 =  0 \\
% 1 \oplus 1 \oplus 1 = (1 \oplus 1) \oplus 1 = 0 \oplus 1 =  1 \\
% \end{align*}

% Practical uses of XOR for error detection using odd parity (see the textbook's error correction section page A-64 for details)

% \end{frame}

% \begin{frame}\frametitle{XNOR Gate}
% The complement of the XOR gate is the XNOR Gate:
% \begin{multicols}{2}
% \includegraphics[width=5cm]{02-circuit/02-combinational/figures/xnor.png}
% \columnbreak

% \begin{center}
%     \begin{tabular}{cc|c}
%          X & Y & F \\
%          \hline
%          0 & 0 & 1\\
%          \hline
%          0 & 1 & 0\\
%         \hline
%          1 & 0 & 0\\
%          \hline
%          1 & 1 & 1\\
%          \hline
%     \end{tabular}
% \end{center}
% \end{multicols}
% The XNOR gate has the following mathematical symbol:
% $$F = \overline{X \oplus Y}$$
% \begin{itemize}
%     \item  the output of XNOR is 1 if the input has an even number of 1's
%     \small
%     \begin{itemize}
%         \item 0 is an even number 
%         \item when all inputs are 0, there are 0 number of 1's that is an even number of 1's, which implies, XNOR is asserted (1)
%     \end{itemize}
%     \item Practical uses of XNOR for implementing a full-adder in ALU and the equality  comparator.
% \end{itemize}
% \end{frame}


\begin{frame}[fragile]
\STitle{Example: Deriving Truth Table from Circuit}
\begin{tcolorbox}[enhanced,attach boxed title to top center={yshift=-3mm,yshifttext=-1mm},
  colback=red!5!white,colframe=red!75!black,colbacktitle=red!80!black,
  title=Try this,fonttitle=\bfseries,
  boxed title style={size=small,colframe=red!50!black} ]
  Given the following circuit:
  \begin{center}
      \includegraphics[width=2in]{Figs/xorMagic}
  \end{center}
  
  Complete the following truth table with intermediate outputs.
  \begin{center}\small
	\begin{tabular}{cc|ccc|c}
		X & Y & A & B & C & F\\
		\hline
		0 & 0 &&&&\\
		0 & 1 &&&&\\
		1 & 0 &&&&\\
		1 & 1 &&&&
	\end{tabular}
	\end{center}
\end{tcolorbox}

\end{frame}
\BNotes\ifnum\Notes=1
\begin{frame}[fragile]
Instructor's Notes:
\begin{itemize}
\item Derivation process:

	1. Find gate all of whose inputs are known

	2. Compute its outputs (in truth table)

	3. Repeat
\item This circuit computes the exclusive-OR (XOR) function. 

	Note that the column for C (for example)
	cannot be computed unless the column for A is done first.
\item Draw XOR symbol; draw and discuss XNOR
\end{itemize}
\end{frame}
\fi\ENotes

\ifnum\Ans=1
\begin{frame}[fragile]
\STitle{Solution: Deriving Truth Table from Circuit}
\begin{tcolorbox}[enhanced,attach boxed title to top center={yshift=-3mm,yshifttext=-1mm},
  colback=red!5!white,colframe=red!75!black,colbacktitle=red!80!black,
  title=Try this,fonttitle=\bfseries,
  boxed title style={size=small,colframe=red!50!black} ]
  Given the following circuit:
  \begin{center}
      \includegraphics[width=2in]{Figs/xorMagic}
  \end{center}
  
  Complete the following truth table with intermediate outputs.
  \begin{center}\small
	\begin{tabular}{cc|ccc|c}
		X & Y & A & B & C & F\\
		\hline
		0 & 0 & 1 & 1&1&0\\
		0 & 1 & 1 & 1&0&1\\
		1 & 0 & 1 & 0&1&1\\
		1 & 1 & 0 & 1 &1&0\\
	\end{tabular}
	\end{center}
\end{tcolorbox}

\end{frame}
\BNotes\ifnum\Notes=1
\begin{frame}[fragile]
Instructor's Notes:
\begin{itemize}
\item Derivation process:

	1. Find gate all of whose inputs are known

	2. Compute its outputs (in truth table)

	3. Repeat
\item This circuit computes the exclusive-OR (XOR) function. 

	Note that the column for C (for example)
	cannot be computed unless the column for A is done first.
\item Draw XOR symbol; draw and discuss XNOR
\end{itemize}
\end{frame}
\fi\ENotes
\fi

\begin{frame}[fragile]
\STitle{Useful Components: Decoders}
\begin{itemize}
	\item $n$ inputs, $2^n$ outputs (converts binary to ``unary'')
	\item Example: 3-to-8 (or 3-bit) decoder
		\begin{center}
		\begin{tabular}{ccc|cccccccc}
		$A_2$ & $A_1$ & $A_0$ & $D_7$ & $D_6$ & $D_5$ & $D_4$ & $D_3$ & $D_2$ & $D_1$ & $D_0$\\\hline
		0 & 0 & 0 & 0 & 0 & 0 & 0 & 0 & 0 & 0 &1\\
		0 & 0 & 1 & 0 & 0 & 0 & 0 & 0 & 0 & 1 &0\\
		0 & 1 & 0 & 0 & 0 & 0 & 0 & 0 & 1 & 0 &0\\
		0 & 1 & 1 & 0 & 0 & 0 & 0 & 1 & 0 & 0 &0\\
		1 & 0 & 0 & 0 & 0 & 0 & 1 & 0 & 0 & 0 &0\\
		1 & 0 & 1 & 0 & 0 & 1 & 0 & 0 & 0 & 0 &0\\
		1 & 1 & 0 & 0 & 1 & 0 & 0 & 0 & 0 & 0 &0\\
		1 & 1 & 1 & 1 & 0 & 0 & 0 & 0 & 0 & 0 &0\\
		\end{tabular}
		\end{center}
  \item \hl{SOP form for each output function, is exactly one \textbf{minterm}}
	% \item Circuit has regular structure,
 % \item Practical uses: selecting a unit from a number of units 
\end{itemize}
\end{frame}
\BNotes\ifnum\Notes=1
\begin{frame}[fragile]
Instructor's Notes:
\begin{itemize}
\item
	The book does not give the circuit, but gives the same truth table on
	page C--9. This is NOT spaghetti wiring, since it is a
	diagram with many crossing wires, but it has a regular
	structure. Using a sum-of-product representation with input literals
	would hide this structure.
\item 
	What good are decoders? They are used when one of a large number of
	units needs to be ``activated''. We will use them in register files
	and memory units.

	Use as an example a computer with 512MB of memory, and wanting
	to expand it by using 3 more 512MB memory chips (ie, to 2 GB total).
	Draw a picture on the board of four 512MB memory chips with select 
	lines, and use a decoder to select one of them to "turn on".  There
	will be 29 address lines into each memory chip, with two more 
	address lines going into the decoder, giving 31 address lines in
	total.

	The "turn on" part won't make sense until we talk about three-state
	buffers, but with this type of circuit, the outputs of the memory
	chips are tied directly together.
\end{itemize}
\end{frame}
\fi\ENotes

\begin{frame}{Circuit Design based on SOP form for each output $D_0$-$D_7$}
		\Figure{!}{6in}{2.5in}{Figs/3-8-decoder2}
		\centerline{3-to-8 (or 3-bit) Decoder}
\BNotes\ifnum\Notes=1
~
\fi\ENotes
\end{frame}

\begin{frame}{Practical use of decoders}
    \begin{tcolorbox}[enhanced,attach boxed title to top center={yshift=-3mm,yshifttext=-1mm},
  colback=blue!5!white,colframe=blue!75!black,colbacktitle=blue!80!black,
  title=Think About It,fonttitle=\bfseries,
  boxed title style={size=small,colframe=red!50!black} ]
  \begin{itemize}
    \item Decoding "addresses", selecting a unit from a large number of units
    \item For example: 
    \begin{itemize}
            \item A register from 32 registers
    \item A memory chip from a set of memory chips
    \end{itemize}
  \end{itemize}
\end{tcolorbox}
\end{frame}


\begin{frame}[fragile]
\STitle{Multiplexors\footnote{also Multiplexers}}
\begin{itemize}
	\item Inputs: $2^n$ lines ($D_0,\dots,D_{2^n-1}$)

		$n$ select lines ($S_{n-1},\dots,S_0$)
	\item Output: The value of the $D_S$ line
	\item Example: 4-1 Multiplexer
		\begin{center}
  \begin{columns}
      \column{0.3\textwidth}

      \column{0.3\textwidth}
  
		\begin{tabular}{cc|c}
		$S_1$ & $S_0$ & $Y$\\\hline
		0 & 0 & $D_0$\\
		0 & 1 & $D_1$\\
		1 & 0 & $D_2$\\
		1 & 1 & $D_3$\\
		\end{tabular}
        \column{0.3\textwidth}
        {\footnotesize \color{red} the value at that input is the value at the output}
  \end{columns}
		\end{center}

	\Figure{!}{2in}{1in}{Figs/multiplexer-update}
\end{itemize}
\end{frame}

\BNotes\ifnum\Notes=1
\begin{frame}[fragile]
Instructor's Notes:
\begin{itemize}
\item You can warn students that most people spell it
``multiplexers''.
\item The book only does a 2-1 multiplexor.
\item Note the unusual truth table: $D_i$ are inputs, but appear in the
	output column.  Consider putting up part of standard truth table
	(with both $S_i$ and $D_i$ as inputs and $Y$ having 0,1 values)
	and also noting that this shortened form expresses operation
	better.
\item Book usually puts select lines at top, but occasionally at bottom,
	and sometimes select on left, data in on top, output on
	bottom, and sometimes select omitted.

 \item What good are multiplexors? We use them when we want to combine
several possible sources of data on one line.  They can again be useful
in register files and memory, though alternate methods are often
used (three-state buffers). When we have a functional unit and want to
use it for more than one thing, it is natural to put a multiplexor
before its inputs.

Draw a figure on the board with 4 boxes, 2 inputs each (A and B, eight
	or 32 lines) and one (8 or 32 bit) output.  The boxes should be
	labeled $+,-,*,/$.  Feed all four into a multiplexor, whose
	control lines select which operation we want.  

	This is roughly how the CPU works, although we don't have a box
	for multiplication or division.  Ie, we compute everything we
	might want, and use a multiplexor to pick what we really want.

\end{itemize}
\end{frame}
\fi\ENotes

\begin{frame}{Partial Truth Table for Multiplexer}
Example: 4-to-1 MUX
\begin{multicols}{2}
    \begin{center}
\small
		\begin{tabular}{cc|cccc|c}
		 $S_1$ & $S_0$ & $D_3$ & $D_2$ &$D_1$ & $D_0$ & $Y$\\\hline
		% 0 & 0 & 0 & 0 \\
   0 & 0 & 0 & 0 & 0 & \textcolor{blue}{0} & \textcolor{blue}{0} \\
   0 & 0 & 0 & 0 & 0 & \textcolor{blue}{1} & \textcolor{blue}{1} \\
   0 & 0 & 0 & 0 & 1 & \textcolor{blue}{0} & \textcolor{blue}{0} \\
   0 & 0 & 0 & 0 & 1& \textcolor{blue}{1} & \textcolor{blue}{1} \\
   0 & 0 & 0 & 1 & 0 & \textcolor{blue}{0} & \textcolor{blue}{0} \\
    0 & 0 & 0 & 1 & 0 & \textcolor{blue}{1} & \textcolor{blue}{1} \\
    0 & 0 & 0 & 1 & 1 & \textcolor{blue}{0} & \textcolor{blue}{0} \\
    0 & 0 & 0 & 1 & 1 & \textcolor{blue}{1} & \textcolor{blue}{1} \\
   \hline
   0 & 0 & 1 & 0 & 0 & \textcolor{blue}{0} & \textcolor{blue}{0} \\
   0 & 0 & 1 & 0 & 0 & \textcolor{blue}{1} & \textcolor{blue}{1} \\
   0 & 0 & 1 & 0 & 1 & \textcolor{blue}{0} & \textcolor{blue}{0} \\
   0 & 0 & 1 & 0 & 1& \textcolor{blue}{1} & \textcolor{blue}{1} \\
   0 & 0 & 1 & 1 & 0 & \textcolor{blue}{0} & \textcolor{blue}{0} \\
    0 & 0 & 1 & 1 & 0 & \textcolor{blue}{1} & \textcolor{blue}{1} \\
    0 & 0 & 1 & 1 & 1 & \textcolor{blue}{0} & \textcolor{blue}{0} \\
    0 & 0 & 1 & 1 & 1 & \textcolor{blue}{1} & \textcolor{blue}{1} \\
    \hline
		\end{tabular}
		\end{center}

  \columnbreak

\begin{center}
\small
		\begin{tabular}{cc|cccc|c}
		 $S_1$ & $S_0$ & $D_3$ & $D_2$ &$D_1$ & $D_0$ & $Y$\\\hline
		% 0 & 0 & 0 & 0 \\
   0 & 1 & 0 & 0 & \textcolor{blue}{0} & 0 & \textcolor{blue}{0} \\
   0 & 1 & 0 & 0 & \textcolor{blue}{0} & 1 &\textcolor{blue}{0} \\
   0 & 1 & 0 & 0 & \textcolor{blue}{1} & 0 & \textcolor{blue}{1} \\
   0 & 1 & 0 & 0 & \textcolor{blue}{1} & 1 &\textcolor{blue}{1} \\
   0 & 1 & 0 & 1 &  \textcolor{blue}{0} & 0 & \textcolor{blue}{0} \\
    0 & 1 & 0 & 1 &  \textcolor{blue}{0} & 1& \textcolor{blue}{0} \\
    0 &1 & 0 & 1 &  \textcolor{blue}{1} & 0 & \textcolor{blue}{1} \\
    0 & 1 & 0 & 1 &  \textcolor{blue}{1} & 1& \textcolor{blue}{1} \\
   \hline
  0 & 1 & 1 & 0 & \textcolor{blue}{0} & 0 & \textcolor{blue}{0} \\
   0 & 1 & 1 & 0 & \textcolor{blue}{0} & 1 &\textcolor{blue}{0} \\
   0 & 1 & 1 & 0 & \textcolor{blue}{1} & 0 & \textcolor{blue}{1} \\
   0 & 1 & 1 & 0 & \textcolor{blue}{1} & 1 &\textcolor{blue}{1} \\
   0 & 1 & 1 & 1 &  \textcolor{blue}{0} & 0 & \textcolor{blue}{0} \\
    0 & 1 &1 & 1 &  \textcolor{blue}{0} & 1& \textcolor{blue}{0} \\
    0 &1 & 1 & 1 &  \textcolor{blue}{1} & 0 & \textcolor{blue}{1} \\
    0 & 1 & 1 & 1 &  \textcolor{blue}{1} & 1& \textcolor{blue}{1} \\
    \hline
		\end{tabular}
		\end{center}
  \end{multicols}
\end{frame}


\begin{frame}{Partial Compressed Truth Table for Multiplexer}
Example: 4-to-1 MUX, when $S_1 = 0$ and $S_0 = 0$
\begin{multicols}{2}
    \begin{center}
\small
		\begin{tabular}{cc|cccc|c}
		 $S_1$ & $S_0$ & $D_3$ & $D_2$ &$D_1$ & $D_0$ & $Y$\\\hline
		
   0 & 0 & 0 & 0 & 0 & \textcolor{blue}{0} & \textcolor{blue}{0} \\
   0 & 0 & 0 & 0 & 0 & \textcolor{blue}{1} & \textcolor{blue}{1} \\
   0 & 0 & 0 & 0 & 1 & \textcolor{blue}{0} & \textcolor{blue}{0} \\
   0 & 0 & 0 & 0 & 1& \textcolor{blue}{1} & \textcolor{blue}{1} \\
   0 & 0 & 0 & 1 & 0 & \textcolor{blue}{0} & \textcolor{blue}{0} \\
    0 & 0 & 0 & 1 & 0 & \textcolor{blue}{1} & \textcolor{blue}{1} \\
    0 & 0 & 0 & 1 & 1 & \textcolor{blue}{0} & \textcolor{blue}{0} \\
    0 & 0 & 0 & 1 & 1 & \textcolor{blue}{1} & \textcolor{blue}{1} \\
   \hline
   0 & 0 & 1 & 0 & 0 & \textcolor{blue}{0} & \textcolor{blue}{0} \\
   0 & 0 & 1 & 0 & 0 & \textcolor{blue}{1} & \textcolor{blue}{1} \\
   0 & 0 & 1 & 0 & 1 & \textcolor{blue}{0} & \textcolor{blue}{0} \\
   0 & 0 & 1 & 0 & 1& \textcolor{blue}{1} & \textcolor{blue}{1} \\
   0 & 0 & 1 & 1 & 0 & \textcolor{blue}{0} & \textcolor{blue}{0} \\
    0 & 0 & 1 & 1 & 0 & \textcolor{blue}{1} & \textcolor{blue}{1} \\
    0 & 0 & 1 & 1 & 1 & \textcolor{blue}{0} & \textcolor{blue}{0} \\
    0 & 0 & 1 & 1 & 1 & \textcolor{blue}{1} & \textcolor{blue}{1} \\
    \hline
		\end{tabular}
		\end{center}

  \columnbreak

   \begin{center}
\small
		\begin{tabular}{cc|cccc|c}
		 $S_1$ & $S_0$ & $D_3$ & $D_2$ &$D_1$ & $D_0$ & $Y$\\\hline
		
   0 & 0 & X & X & X & \textcolor{blue}{0} & \textcolor{blue}{0} \\
   0 & 0 & X & X & X & \textcolor{blue}{1} & \textcolor{blue}{1} \\
    \hline
		\end{tabular}
		\end{center}

  
% This implies, that $Y=\bar{S_1}\bar{S_0}{D_0}$. This is also the same as
Rearrange truth table
\begin{center}
\small
		\begin{tabular}{cc|c}
		 $S_1$ & $S_0$ & $Y$\\\hline
\hline		
   0 & 0 & \textcolor{blue}{$D_0$} \\
    \hline
		\end{tabular}
		\end{center}
  
  \end{multicols}
\end{frame}

\begin{frame}{Partial Truth Table for Multiplexer}
Example: 4-to-1 MUX when $S_1 = 0$ and $S_0 = 1$
\begin{multicols}{2}
  \begin{center}
\small
		\begin{tabular}{cc|cccc|c}
		 $S_1$ & $S_0$ & $D_3$ & $D_2$ &$D_1$ & $D_0$ & $Y$\\\hline
		% 0 & 0 & 0 & 0 \\
   0 & 1 & 0 & 0 & \textcolor{blue}{0} & 0 & \textcolor{blue}{0} \\
   0 & 1 & 0 & 0 & \textcolor{blue}{0} & 1 &\textcolor{blue}{0} \\
   0 & 1 & 0 & 0 & \textcolor{blue}{1} & 0 & \textcolor{blue}{1} \\
   0 & 1 & 0 & 0 & \textcolor{blue}{1} & 1 &\textcolor{blue}{1} \\
   0 & 1 & 0 & 1 &  \textcolor{blue}{0} & 0 & \textcolor{blue}{0} \\
    0 & 1 & 0 & 1 &  \textcolor{blue}{0} & 1& \textcolor{blue}{0} \\
    0 &1 & 0 & 1 &  \textcolor{blue}{1} & 0 & \textcolor{blue}{1} \\
    0 & 1 & 0 & 1 &  \textcolor{blue}{1} & 1& \textcolor{blue}{1} \\
   \hline
  0 & 1 & 1 & 0 & \textcolor{blue}{0} & 0 & \textcolor{blue}{0} \\
   0 & 1 & 1 & 0 & \textcolor{blue}{0} & 1 &\textcolor{blue}{0} \\
   0 & 1 & 1 & 0 & \textcolor{blue}{1} & 0 & \textcolor{blue}{1} \\
   0 & 1 & 1 & 0 & \textcolor{blue}{1} & 1 &\textcolor{blue}{1} \\
   0 & 1 & 1 & 1 &  \textcolor{blue}{0} & 0 & \textcolor{blue}{0} \\
    0 & 1 &1 & 1 &  \textcolor{blue}{0} & 1& \textcolor{blue}{0} \\
    0 &1 & 1 & 1 &  \textcolor{blue}{1} & 0 & \textcolor{blue}{1} \\
    0 & 1 & 1 & 1 &  \textcolor{blue}{1} & 1& \textcolor{blue}{1} \\
    \hline
		\end{tabular}
		\end{center}

  \columnbreak

 \begin{center}
\small
		\begin{tabular}{cc|cccc|c}
		 $S_1$ & $S_0$ & $D_3$ & $D_2$ &$D_1$ & $D_0$ & $Y$\\\hline
		
   0 & 1 & X & X & \textcolor{blue}{0} &  X & \textcolor{blue}{0} \\
   0 & 1 & X & X & \textcolor{blue}{1} & X & \textcolor{blue}{1} \\
    \hline
		\end{tabular}
		\end{center}

  
% This implies, that $Y=\bar{S_1}\bar{S_0}{D_0}$. This is also the same as
Rearrange truth table
\begin{center}
\small
		\begin{tabular}{cc|c}
		 $S_1$ & $S_0$ & $Y$\\\hline
\hline		
   0 & 1 & \textcolor{blue}{$D_1$} \\
    \hline
		\end{tabular}
		\end{center}

  \end{multicols}
\end{frame}



\begin{frame}{Circuit design of 4:1 MUX using SOP form for Y}
\begin{multicols}{2}
	\Figure{!}{1.5in}{2in}{Figs/multiplexer3}
	% \centerline{4-1 Multiplexor}
\columnbreak

 Truth Table for 4-1 Multiplexer
		\begin{center}
		\begin{tabular}{cc|c}
		$S_1$ & $S_0$ & $Y$\\\hline
		0 & 0 & $D_0$\\
		0 & 1 & $D_1$\\
		1 & 0 & $D_2$\\
		1 & 1 & $D_3$\\
		\end{tabular}
		\end{center}
  This truth table is more concise and operation of MUX is clear. 
\end{multicols}  

\BNotes\ifnum\Notes=1
~
\fi\ENotes
\end{frame}

\begin{frame}{Practical use of MUX}
    \begin{tcolorbox}[enhanced,attach boxed title to top center={yshift=-3mm,yshifttext=-1mm},
  colback=blue!5!white,colframe=blue!75!black,colbacktitle=blue!80!black,
  title=Think About It,fonttitle=\bfseries,
  boxed title style={size=small,colframe=red!50!black} ]
  \begin{itemize}
    \item Connect multiple sources of data to a single output
    \item At any time, one of the inputs is selected to be passed to the output
    \begin{itemize}
        \item If a functional unit is to be used with multiple inputs, a MUX will allow us to select the input to the functional unit
    \end{itemize}
    \item Perform logic functions
    \item \hl{MUX - a universal combinational logic circuit}
    \begin{itemize}
        \item NAND - a universal logic gate
    \end{itemize}
  \end{itemize}
\end{tcolorbox}

\BNotes\ifnum\Notes=1
all possible logic circuit can be implemented using MUX
all possible logic gates can be implemented using NAND
\fi
\end{frame}

% \begin{frame}{Example 1: Use of MUX to select input to a functional unit}
%     \begin{tcolorbox}[enhanced,attach boxed title to top center={yshift=-3mm,yshifttext=-1mm},
%   colback=blue!5!white,colframe=blue!75!black,colbacktitle=blue!80!black,
%   title=Think About It,fonttitle=\bfseries,
%   boxed title style={size=small,colframe=red!50!black} ]
%   \begin{itemize}
%         \item If a functional unit is to be used with multiple inputs, a MUX will allow us to select the input to the functional unit
%   \item Consider the functional unit ALU to perform ADD. The input can be:
%   \begin{itemize}
%       \item contents of two registers
%   \item  content of a register and an \texttt{Immediate}
%   \end{itemize}
  
%   \end{itemize}
% \end{tcolorbox}
% \end{frame}

% \begin{frame}{Example 2: Use of MUX to perform logic functions}
%     \begin{tcolorbox}[enhanced,attach boxed title to top center={yshift=-3mm,yshifttext=-1mm},
%   colback=blue!5!white,colframe=blue!75!black,colbacktitle=blue!80!black,
%   title=Think About It,fonttitle=\bfseries,
%   boxed title style={size=small,colframe=red!50!black} ]
%   Given the truth table below, use a 2:1 MUX to implement the function F.
%  \begin{center}
% 		\begin{tabular}{cc|c}
% 		$X$ & $Y$ & $F$\\\hline
% 		0 & 0 & 0\\
% 		0 & 1 & 1\\
% 		1 & 0 & 1\\
% 		1 & 1 & 1\\
% 		\end{tabular}
% 		\end{center}
%   \begin{itemize}
%     \item inspect the truth table 
%     \item combine pairs of rows to suppress right most literal
%     \item use right most literal to express output of MUX 
%   \end{itemize}
% \end{tcolorbox}
% \end{frame}

% \begin{frame}{Example 3: MUX - universal logic circuit}
%     \begin{tcolorbox}[enhanced,attach boxed title to top center={yshift=-3mm,yshifttext=-1mm},
%   colback=blue!5!white,colframe=blue!75!black,colbacktitle=blue!80!black,
%   title=Think About It,fonttitle=\bfseries,
%   boxed title style={size=small,colframe=red!50!black} ]
%   Use 4 4:1 MUX to design a circuit to achieve circular left shift, with a 4-bit input $A$ and a 4-bit output $F$.
% \end{tcolorbox}

% \BNotes\ifnum\Notes=1
% haris and haris 5.18 in textbook solution is 5.14 in ebook
% \fi
% \end{frame}




\begin{frame}[fragile]
\STitle{Arrays of Logic Elements}
\begin{itemize}
	\item ``Slash'' notation is used to indicate lines carrying
	multiple bits, and to imply parallel constructions

%	\PHFigure{!}{4.5in}{4in}{PHALL/B08}{Similar to Figure A.3.6}
	\Figure{!}{4.5in}{2in}{Figs/32mux}

64-bit wide, 2:1 multiplextor expands to 64, 1 bit, 2:1 multiplexors
\end{itemize}
\end{frame}
\BNotes\ifnum\Notes=1
\begin{frame}[fragile]
Instructor's Notes:
This sort of construction is used a lot in processor designs later
on. Again, it is a shorthand notation that simplifies diagrams.
\end{frame}
\fi\ENotes
\setlength{\columnseprule}{1pt}
\def\columnseprulecolor{\color{blue}}


\begin{frame}[fragile]
\STitle{Implementing Boolean Functions: PLAs}
\begin{itemize}
	\item A PLA (Programmable Logic Array) implements a two-level
	function
	
%	\Figure{!}{4in}{3in}{PHFigs/B05}
	\Figure{!}{4in}{2in}{PHALL/B05}
	\item Typically a PLA has fixed number of product terms and outputs available
 	% \item Typically a PLA has fixed number of inputs and outputs available
\end{itemize}
\end{frame}
\BNotes\ifnum\Notes=1
\begin{frame}[fragile]
Instructor's Notes:
\begin{itemize}
	\item Mention that both AND and OR arrays are programmable.
	\item Internally, complemented inputs are available
	\item Draw simple example (show AND and OR gates, but not fuses,
		just draw connections) on blackboard
\end{itemize}
\end{frame}
\fi\ENotes

\begin{frame}{Example: 3 input 2 output PLA}
\resizebox{8cm}{8cm}{
    \input 02-circuit/02-combinational/pla1_image
    }
\end{frame}

\begin{frame}{Example: \hl{Programmed} 3 input 2 output PLA}
    \resizebox{12cm}{8cm}{
    % \input 02-circuit/02-combinational/pla2_image
    \includegraphics[width=\textwidth]{02-circuit/02-combinational/pla2_image.png}
    }
\end{frame}

% \begin{frame}[fragile]
% \STitle{Implementing Boolean Functions: ROMs}
% \begin{itemize}
% 	\Figure{!}{1.25in}{.75in}{Figs/ROM}
% 	\item Can think of ROM as table of $2^n$ $m$-bit words
% 	\item Can think of ROM as implementing $m$ one-bit functions
% 	of $n$ variables
% 	\item Internally, consists of a decoder plus an OR gate for
% 	each output
% 	\item Types of ROM: PROM, EPROM, EEPROM
% %	\item ROMs and PLAs closely related
% 	\item PLAs - simplified ROM

% 		Less hardware, but less flexible
% \end{itemize}
% \end{frame}
% \BNotes\ifnum\Notes=1
% \begin{frame}[fragile]
% Instructor's Notes:
% \begin{itemize}
% \item We'll be talking about using ROMs or PLAs in our discussion of
% implementing finite-state machines, but otherwise we do not use them
% much in this course. You can talk about their more general uses if you
% wish. 
% \item Note that the fixed number of product terms in a PLA means that
% it cannot implement all possible collections of $m$ one-bit functions;
% some would require too many product terms. In contrast, a ROM can implement
% any collection.
% \bigskip
% \item {\bf The real point}: there is non-volatile memory in the computer (ie, that doesn't forget when power is turned off).  This memory is read many times and written rarely.  In older technologies, this memory was written once, when it was created, and then installed in the computer.  But newer ones can be written after having been installed in the computer.  Flashing the bios is an example of this.
% \end{itemize}
% \end{frame}
% \fi\ENotes



% \begin{frame}[fragile]
% \STitle{Conclusion}
%  \underline{\textbf{Lecture Summary}}
%  \begin{itemize}
%  \item PLA
%  \item ROM
%  \end{itemize}
%  \underline{\textbf{Assigned Textbook Readings}}
% \begin{itemize}
%      \item \textbf{Read} Sections A.3
%      %--A.12 is not in the details relevant to the scope of the course
%      \end{itemize}
%     \underline{\textbf{Next Steps}}
%     \begin{itemize}
%     \item \textbf{Ask} questions in the next tutorial or office hours.
%  \end{itemize}

% \end{frame}


% \begin{frame}{Additional Slides}
%      Remaining slides are additional notes for your information.
%  \end{frame}

 % \begin{frame}{Example: Using ROM for Logic}
 %     \includegraphics[scale=0.2]{02-circuit/02-combinational/rom-logic.png}
 % \end{frame}
% \input 02-4-components
%%%% add PLA to ROM, and other storage ideas
% \begin{frame}[fragile]
% \STitle{Implementing Boolean Functions: PLAs}
% \begin{itemize}
% 	\item A PLA (Programmable Logic Array) implements a two-level
% 	function
	
% %	\Figure{!}{4in}{3in}{PHFigs/B05}
% 	\Figure{!}{4in}{2in}{PHALL/B05}
% 	\item Typically a PLA has fixed number of product terms and outputs available
% \end{itemize}
% \end{frame}
% \BNotes\ifnum\Notes=1
% \begin{frame}[fragile]
% Instructor's Notes:
% \begin{itemize}
% 	\item Mention that both AND and OR arrays are programmable.
% 	\item Internally, complemented inputs are available
% 	\item Draw simple example (show AND and OR gates, but not fuses,
% 		just draw connections) on blackboard
% \end{itemize}
% \end{frame}
% \fi\ENotes

% \begin{frame}{Example: 3 input 2 output PLA}
% \resizebox{8cm}{8cm}{
%     \input 02-circuit/02-combinational/pla1_image
%     }
% \end{frame}

% \begin{frame}{Example: \hl{Programmed} 3 input 2 output PLA}
%     \resizebox{8cm}{8cm}{
%     \input 02-circuit/02-combinational/pla2_image
%     }
% \end{frame}
% \begin{frame}[fragile]
% \STitle{Implementing Boolean Functions: ROMs}
% \begin{itemize}
% 	\Figure{!}{1.25in}{.75in}{Figs/ROM}
% 	\item Can think of ROM as table of $2^n$ $m$-bit words
% 	\item Can think of ROM as implementing $m$ one-bit functions
% 	of $n$ variables
% 	\item Internally, consists of a decoder plus an OR gate for
% 	each output
% 	\item Types of ROM: PROM, EPROM, EEPROM
% %	\item ROMs and PLAs closely related
% 	\item PLAs - simplified ROM

% 		Less hardware, but less flexible
% \end{itemize}
% \end{frame}
\BNotes\ifnum\Notes=1
\begin{frame}[fragile]
Instructor's Notes:
\begin{itemize}
\item We'll be talking about using ROMs or PLAs in our discussion of
implementing finite-state machines, but otherwise we do not use them
much in this course. You can talk about their more general uses if you
wish. 
\item Note that the fixed number of product terms in a PLA means that
it cannot implement all possible collections of $m$ one-bit functions;
some would require too many product terms. In contrast, a ROM can implement
any collection.
\bigskip
\item {\bf The real point}: there is non-volatile memory in the computer (ie, that doesn't forget when power is turned off).  This memory is read many times and written rarely.  In older technologies, this memory was written once, when it was created, and then installed in the computer.  But newer ones can be written after having been installed in the computer.  Flashing the bios is an example of this.
\end{itemize}
\end{frame}
\fi\ENotes

\begin{frame}[fragile]
\STitle{Conclusion}
 \underline{\textbf{Lecture Summary}}
 \begin{itemize}
 \item Gate logic
 \item Multiplexors and Decoders
 \end{itemize}
 \underline{\textbf{Assigned Textbook Readings}}
\begin{itemize}
     \item \textbf{Read} Section A.3
     \end{itemize}
    \underline{\textbf{Next Steps}}
    \begin{itemize}
    \item \textbf{Ask} questions in the next week's tutorial or office hours.
 \end{itemize}

 \end{frame}


 \begin{frame}{Additional Slides}
     Remaining slides are additional notes for your information.
 \end{frame}

% for comprehensive
\newpage

 \begin{frame}{Example}
    \begin{tcolorbox}[enhanced,attach boxed title to top center={yshift=-3mm,yshifttext=-1mm},
  colback=red!5!white,colframe=red!75!black,colbacktitle=red!80!black,
  title=Try this,fonttitle=\bfseries,
  boxed title style={size=small,colframe=red!50!black} ]
  Draw the combinational logic circuit that implements $F$ using only NAND gates:  
\begin{figure}[H]
	\begin{center}
		\begin{tabular}{ccc|c}
		X&Y&Z & F\\
		\hline
		0&0&0 & 0\\
		0&0&1 & 0\\
0&1&0 &1\\
0&1&1 & 1\\
\hline
1&0&0 & 0\\
1&0&1 & 0\\
1&1&0 & 1\\
1&1&1 & 0\\
		\end{tabular}
		\end{center}
\end{figure}
\end{tcolorbox}
\end{frame}

\ifnum\Ans=1
\begin{frame}{Solution: Derive Boolean expression for F}
    \begin{itemize}
        \item Derive the Boolean expression for F
        \begin{itemize}
            \item Sum of Products Form:\\
            \small
            $F = \bar{X}Y\bar{Z} + \bar{X}YZ + XY\bar{Z} $
            \item Simplify the expression using Laws of Boolean Algebra:\\
            \small
            $F = \bar{X}Y\bar{Z} + \bar{X}YZ + XY\bar{Z} +\bar{X}Y\bar{Z}$\\
            $F = \bar{X}Y(\bar{Z}+Z) + Y\bar{Z}(X + \bar{X})$\\
            $F = \bar{X}Y+ Y\bar{Z}$\\ 
            $F = Y(\bar{X}+\bar{Z})$
        % \item Product of Sums Form:\\
        % \small
        % % \begin{align}
        % %   F= (X+Y+Z)(X+Y+\bar{Z})(\bar{X}+Y+Z) 
        % %         (\bar{X}+Y+\bar{Z})(\bar{X}+\bar{Y}+\bar{Z})  
        % % \end{align}
        %  $ F= (X+Y+Z)(X+Y+\bar{Z})(\bar{X}+Y+Z) 
        %          (\bar{X}+Y+\bar{Z})(\bar{X}+\bar{Y}+\bar{Z}) $
        \end{itemize}
        \item Draw the circuit using ANDs and ORs
        \item Convert ANDs and ORs to NAND equivalent 
    \end{itemize}
   
 
\end{frame}
\fi
% \begin{frame}{Solution: Circuit design }
%     \begin{multicols}{2}
%     Design using ANDs and ORs for SOP for of F %or POS form of F

%     % Note: $\bar{X}$ is represented as $!X$ in the circuit
%     \includegraphics[height=5cm]{02-circuit/02-combinational/figures/col1.png}
%     \columnbreak

%      Design using NAND equivalent or NAND for F

%          Note: $\bar{X}$ is represented as $!X$ in the circuit

%     \includegraphics[height=5cm]{02-circuit/02-combinational/figures/col2.png}
%     \end{multicols}
% \end{frame}

% for comprehensive
\newpage

\ifnum\Ans=1
\begin{frame}{Solution: Circuit design }
    \resizebox{6cm}{8cm}{
    \input{02-circuit/02-combinational/circuit6}
    }
\end{frame}
\fi

\begin{frame}{Universal NAND Gates}
\begin{columns}
    \column{0.3\textwidth}
    \includegraphics[width=\textwidth]{02-circuit/02-combinational/figures/NandNot.png}

    \vspace{0.2cm}

    \begin{tabular}{c|c|c}
        X & !(XX) & !X \\\hline
        0 & 1 & 1 \\
        1 & 0 & 0 \\
    \end{tabular}

    \column{0.3\textwidth}

    \includegraphics[width=\textwidth]{02-circuit/02-combinational/figures/NandOr.png}

    ![(!X)(!Y)] = X + Y by DML

    \column{0.3\textwidth}
    \includegraphics[width=\textwidth]{02-circuit/02-combinational/figures/NandAnd.png}
    
    ![!(XY)] = XY
\end{columns}
\end{frame}

\begin{frame}\frametitle{XOR Gate}
The gate symbol and truth table for XOR
\begin{multicols}{2}
\includegraphics[width=5cm]{02-circuit/02-combinational/figures/xor.png}
\columnbreak

\begin{center}
    \begin{tabular}{cc|c}
         X & Y & F \\
         \hline
         0 & 0 & 0\\
         \hline
         0 & 1 & 1\\
        \hline
         1 & 0 & 1\\
         \hline
         1 & 1 & 0\\
         \hline
    \end{tabular}
\end{center}

\end{multicols}

The XOR gate has the following mathematical symbol:
$$F = X \oplus Y$$

\end{frame}

\begin{frame}\frametitle{XOR and Parity}
In general, the output of XOR is 1 if the input has an odd number of 1's.

\begin{align*}
1 \oplus 0 \oplus 1 = (1 \oplus 0) \oplus 1 = 1 \oplus 1 =  0 \\
1 \oplus 1 \oplus 1 = (1 \oplus 1) \oplus 1 = 0 \oplus 1 =  1 \\
\end{align*}

Practical uses of XOR for error detection using odd parity (see the textbook's error correction section page A-64 for details)

\end{frame}

\begin{frame}\frametitle{XNOR Gate}
The complement of the XOR gate is the XNOR Gate:
\begin{multicols}{2}
\includegraphics[width=5cm]{02-circuit/02-combinational/figures/xnor.png}
\columnbreak

\begin{center}
    \begin{tabular}{cc|c}
         X & Y & F \\
         \hline
         0 & 0 & 1\\
         \hline
         0 & 1 & 0\\
        \hline
         1 & 0 & 0\\
         \hline
         1 & 1 & 1\\
         \hline
    \end{tabular}
\end{center}
\end{multicols}
The XNOR gate has the following mathematical symbol:
$$F = \overline{X \oplus Y}$$
\begin{itemize}
    \item  the output of XNOR is 1 if the input has an even number of 1's
    \small
    \begin{itemize}
        \item 0 is an even number 
        \item when all inputs are 0, there are 0 number of 1's that is an even number of 1's, which implies, XNOR is asserted (1)
    \end{itemize}
    \item Practical uses of XNOR for implementing a full-adder in ALU and the equality  comparator.
\end{itemize}
\end{frame}

\begin{frame}{Example 1: Use of Decoder}
    \begin{tcolorbox}[enhanced,attach boxed title to top center={yshift=-3mm,yshifttext=-1mm},
  colback=blue!5!white,colframe=blue!75!black,colbacktitle=blue!80!black,
  title=Think About It,fonttitle=\bfseries,
  boxed title style={size=small,colframe=red!50!black} ]
  \begin{itemize}
    \item Construct a 1KB Memory from $256 \times 8$ memory chips
        \item Make sure it is a unified address space 
    \item How many address lines to each chip?
    \item How to select a chip?
  \end{itemize}
\end{tcolorbox}
\end{frame}


\begin{frame}{Example 1: Use of MUX to select input to a functional unit}
    \begin{tcolorbox}[enhanced,attach boxed title to top center={yshift=-3mm,yshifttext=-1mm},
  colback=blue!5!white,colframe=blue!75!black,colbacktitle=blue!80!black,
  title=Think About It,fonttitle=\bfseries,
  boxed title style={size=small,colframe=red!50!black} ]
  \begin{itemize}
        \item If a functional unit is to be used with multiple inputs, a MUX will allow us to select the input to the functional unit
  \item Consider the functional unit ALU to perform ADD. The input can be:
  \begin{itemize}
      \item contents of two registers
  \item  content of a register and an \texttt{Immediate}
  \end{itemize}
  
  \end{itemize}
\end{tcolorbox}
\end{frame}

\begin{frame}{Example 2: Use of MUX to perform logic functions}
    \begin{tcolorbox}[enhanced,attach boxed title to top center={yshift=-3mm,yshifttext=-1mm},
  colback=blue!5!white,colframe=blue!75!black,colbacktitle=blue!80!black,
  title=Think About It,fonttitle=\bfseries,
  boxed title style={size=small,colframe=red!50!black} ]
  Given the truth table below, use a 2:1 MUX to implement the function F.
 \begin{center}
		\begin{tabular}{cc|c}
		$X$ & $Y$ & $F$\\\hline
		0 & 0 & 0\\
		0 & 1 & 1\\
		1 & 0 & 1\\
		1 & 1 & 1\\
		\end{tabular}
		\end{center}
  \begin{itemize}
    \item inspect the truth table 
    \item combine pairs of rows to suppress right most literal
    \item use right most literal to express output of MUX 
  \end{itemize}
\end{tcolorbox}
\end{frame}

\begin{frame}{Example 3: MUX - universal logic circuit}
    \begin{tcolorbox}[enhanced,attach boxed title to top center={yshift=-3mm,yshifttext=-1mm},
  colback=blue!5!white,colframe=blue!75!black,colbacktitle=blue!80!black,
  title=Think About It,fonttitle=\bfseries,
  boxed title style={size=small,colframe=red!50!black} ]
  Use 4 4:1 MUX to design a circuit to achieve circular left shift, with a 4-bit input $A$ and a 4-bit output $F$.
\end{tcolorbox}

\BNotes\ifnum\Notes=1
haris and haris 5.18 in textbook solution is 5.14 in ebook
\fi
\end{frame}

% \begin{frame}{Universal NAND Gates}
% \begin{columns}
%     \column{0.3\textwidth}
%     \includegraphics[width=\textwidth]{02-circuit/02-combinational/figures/NandNot.png}

%     \vspace{0.2cm}

%     \begin{tabular}{c|c|c}
%         X & !(XX) & !X \\\hline
%         0 & 1 & 1 \\
%         1 & 0 & 0 \\
%     \end{tabular}

%     \column{0.3\textwidth}

%     \includegraphics[width=\textwidth]{02-circuit/02-combinational/figures/NandOr.png}

%     ![(!X)(!Y)] = X + Y by DML

%     \column{0.3\textwidth}
%     \includegraphics[width=\textwidth]{02-circuit/02-combinational/figures/NandAnd.png}
    
%     ![!(XY)] = XY
% \end{columns}
% \end{frame}

\begin{frame}{Example: Using ROM for Logic}
     \includegraphics[scale=0.2]{02-circuit/02-combinational/rom-logic.png}
 \end{frame}