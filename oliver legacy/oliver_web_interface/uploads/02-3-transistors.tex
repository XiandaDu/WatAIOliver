\setlength{\columnseprule}{1pt}
\def\columnseprulecolor{\color{blue}}


\begin{frame}[fragile]
\Title{Implementing Gates Using Transistors}
\begin{itemize}
\item Transistor: an electrically-controlled switch

\Figure{!}{1in}{0.5in}{NewTransFigs/nswitch}

\item An nMOS transistor (``n-transistor'') and its symbol
\begin{multicols}{2}
\centering
\Figure{!}{2in}{1in}{Figs/nmos-label}

\columnbreak
\includegraphics[height=1in]{NewTransFigs/newNMOS}
 
\end{multicols}

\item The nMOS transistor can behave similar to a switch, allowing an on-off functionality.
%will discuss this point later
% \item Problem: transmits strong 0 but weak 1
\end{itemize}
\end{frame}
\BNotes\ifnum\Notes=1
\begin{frame}[fragile]
Instructor's Notes:
\begin{itemize}
\item The source and drain are two nearby channels deposited in the substrate
\item The gate is isolated from them by a layer of insulating material
\item When the gate has high voltage applied to it, it alters the electrical properties of the configuration, and allows current to flow between source and drain
\item Physically, source and drain are identical; by convention the source is the terminal of lower voltage
\item Because of the electrical properties of the transistor, when drain is high, source cannot go all the way up to high voltage without the transistor turning off; hence 1 is ``weak''.
\item Because transistors have no moving
parts, they can be extremely fast, which is what makes modern
computers possible. Before transistors, computers were built using
electromagnetic relays, in which current flowing into C causes an
electromagnet to close the connection between A and B. The same types
of designs are possible, but they would be much, much slower.

\end{itemize}
\end{frame}
\fi\ENotes


\begin{frame}[fragile]
\Title{\texttt{nMOS} naturally implements \texttt{NOT} logic}
% nMOS implements a circuit where the output F is the inverse of input A.
\begin{tcolorbox}[enhanced,attach boxed title to top center={yshift=-3mm,yshifttext=-1mm},
  colback=blue!5!white,colframe=blue!75!black,colbacktitle=blue!80!black,
  title=Think About It,fonttitle=\bfseries,
  boxed title style={size=small,colframe=red!50!black} ]
  Consider the transistor circuit and its truth table:
  \begin{multicols}{2}
  \Figure{!}{1in}{1.5in}{NewTransFigs/ntrans}
  \columnbreak
  \begin{center}
      \begin{tabular}{c|c}
          A & F \\
          \hline
           0 & \textbf{1}\\
           1 & \textbf{0}\\
           \hline
      \end{tabular}
  \end{center}  
  \end{multicols}
  {\footnotesize
  \begin{itemize}
  \item When $A=1 \rightarrow$ low resistance between	drain and source ($F=0$)
  \item When $A=0 \rightarrow$ very high resistance between drain and source ($F=1$)
  \item \hl{Gate ($A$) controls whether output $F$ is connected to $power$ or $ground$}
  \item \textbf{nMOS passes a {\color{gray}weak 1}}
  % \item nMOS is naturally inverting
  \end{itemize}
  }
\end{tcolorbox}
\end{frame}
\BNotes\ifnum\Notes=1
\begin{frame}[fragile]
Instructor's Notes:
\begin{itemize}
\item If current is allowed to flow between drain and source, there is a direct connection between power and ground
\item This results in a short circuit, though in practice, a resistor is put between drain and power to limit current flow
\item Gates and therefore complete circuits can be built entirely out of NMOS, and historically this was used before the CMOS technology we describe shortly.
\end{itemize}
\end{frame}
\fi\ENotes

%this slide is not really necessary, the text can be easily added to previous slide
% \begin{frame}[fragile]
% \Title{\texttt{nMOS NOT}}

% \begin{tcolorbox}[enhanced,attach boxed title to top center={yshift=-3mm,yshifttext=-1mm},
%   colback=blue!5!white,colframe=blue!75!black,colbacktitle=blue!80!black,
%   title=Think About It,fonttitle=\bfseries,
%   boxed title style={size=small,colframe=red!50!black} ]
%   \textbf{Problem:} When $A=1$ lots of current flows to the ground (GND) $\Rightarrow$ power is wasted
%   \begin{multicols}{2}
%   \Figure{!}{2in}{1in}{NewTransFigs/ntrans}
%   \columnbreak
%   \begin{center}
%       \begin{tabular}{c|c}
%           A & F \\
%           \hline
%            0 & \textbf{1}\\
%            1 & \textbf{0}\\
%            \hline
%       \end{tabular}
%   \end{center}
%   \end{multicols}
 
%  \textbf{Solution:} Add a resistor between \texttt{power} and \texttt{drain} to avoid wasteful flow of current from power to ground.

% \end{tcolorbox}


% \end{frame}

% \begin{frame}[fragile]
% \Title{A pMOS transistor}
% % Commonly used pMOS symbol in CS251 
% % pMOS symbol in CS251 
% % \begin{multicols}{2}
% % this graphic is wrong, PMOS should have source as terminal on the RHS and drain as terminal on the LHS of the gate
% %in PMOS source and drain are opposit to NMOS
% % \Figure{!}{1in}{0.5in}{NewTransFigs/pdefn}

% % \columnbreak
% \begin{center}
% \includegraphics[height=1in]{NewTransFigs/newPMOS}    
% \end{center}


% % no need for this diagram - not relevent to course, we dont add resistor at the top betwee SRC and outpu
% % \columnbreak
% % \includegraphics[height=1in]{NewTransFigs/ptrans}
% % \end{multicols}

% \begin{itemize}
% 	\item Opposite behaviour to nMOS:
%         \begin{itemize}
% 	\item	If $A=1$, high resistance between drain and source
% 	\item	If $A=0$, low resistance between drain and source
%  % this point is made in next slide
%         % \item   Transmits strong 1 but weak 0
% 	\end{itemize}
% \item ``bubble'' on gate (input \texttt{A}) indicates pMOS transistor behavior is opposite to nMOS transistor 
% \end{itemize}
% \BNotes\ifnum\Notes=1
% ~
% \fi\ENotes
% \end{frame}

\begin{frame}[fragile]
\Title{Transistor Summary}

	% \item Two types of transistors: nMOS, pMOS

	\begin{center}
        \footnotesize
	\begin{tabular}{cc}
		\includegraphics[height=1in]{NewTransFigs/newNMOS} &
		\includegraphics[height=1in]{NewTransFigs/newPMOS.png} \\
	nMOS & pMOS\\
	\begin{tabular}{|r||c|c|}
	\hline
	Input& $A=0$ & $A=1$\\
	\hline
		Resistance& High & Low \\\hline
  % F & {\color{gray}1} & \textbf{0}\\
	% \hline
	\end{tabular} &
%let us not talk about F since it is not directly related to the idea of strong or weak ones

	\begin{tabular}{|r||c|c|}
	\hline
	Input& $A=0$ & $A=1$\\
	\hline
		% Resistance& Low  \normalsize W0, S1& High\\
  	Resistance Q & Low  & High\\
   \hline
   % F & {\color{gray}0}& \textbf{1}\\
	% \hline
	\end{tabular}
	\end{tabular}
	\end{center}
 % \item \textbf{nMOS} passes a \textbf{strong 0} but when resistance is low, it passes a {\color{gray}\textbf{weak 1}}. 
 %  \item \textbf{pMOS} passes a \textbf{strong 1} but when resistance is low, it passes a {\color{gray}\textbf{weak 0}}. 
 \begin{itemize}
  \item \textbf{nMOS} passes a \textbf{strong 0} but a {\color{gray}\textbf{weak 1}}. 
  \item pMOS with ``bubble'' on gate (input \texttt{A}) indicates its behavior is opposite to nMOS transistor 
  \item \textbf{pMOS} passes a \textbf{strong 1} but a {\color{gray}\textbf{weak 0}}. 
  \item Avoid weak transmissions: complement nMOS with pMOS $\rightarrow$ CMOS
  	% \item Avoid: Wasteful flow of current from power to ground, assume resistors used appropriately 
\end{itemize}
\end{frame}
\BNotes\ifnum\Notes=1
\begin{frame}[fragile]
Instructor's Notes:
\begin{itemize}
\item ZHK notes W24: consider rephrasing the strong and weak pass statemnets, to "detect" strong 0 at A and detect weak 1 at A, consider drawing F as output before drain and drawing another output X at src to consider detecting weak or strong 
\item alternatively, weak 1 could imply that most power flows to GND and few charges flow to output F, hence weak 1, but if we do that, this contradicsts, that there is path from F to GND and F is pulled low 
\item the nmos not, is difficult to since we say path to power or path to gnd, but in the figure, f has a path to power and gnd, when A=1, this is not a good example. needs further discussion 
	\item CMOS uses both types of transistors; see next few slides
	\item The analysis proceedure is very mechnical and easy.

		What's hard about what comes next is that as people, we
		like to impose meaning on the transistor diagrams, and
		we expect to see data "flowing" through the circuit.

		But that's not how it works: The transitor inputs (gate
		inputs) just connect the output to power or ground.
		Except in tri-state, the input never flows through to the
		output.
	\item You'll want this slide handy when you do the next few examples
\end{itemize}
\end{frame}
\fi\ENotes

\begin{frame}[fragile]
\Title{CMOS}
\begin{itemize}
	\item CMOS uses both nMOS and pMOS transistors to build circuits with ``clean'' paths to exactly one of power or ground.

  \item   To analyze CMOS circuit:\\
  \end{itemize}
  \begin{itemize}
% \begin{itemize}
    \item Make truth table with inputs, resistance of transistors, and output(s) and 
    \item For each row in the truth table:
        \begin{itemize}
            \item evaluate whether transistor's resistance is high (\texttt{H}) or low (\texttt{L})
            \item \hl{Output is \textbf{1}: if there is a "clean" path to power (1),} 
    \item  Output is \textbf{0}: \hl{if there is a "clean" path to ground (0) }

        \end{itemize}
    
    \item \hl{Make sure there is \textbf{no} clean path to power \textbf{AND} ground}
    \begin{itemize}
        \item If there is, the output of the circuit is \textbf{unknown} 
    \end{itemize}
\end{itemize}       
% \end{itemize}
\end{frame}
\BNotes\ifnum\Notes=1
\begin{frame}[fragile]
Instructor's Notes:
\begin{itemize}
\item The important thing here is that whether the input to the inverter is 1 or 0, there is no current flow
\item There is a small amount of current flow during an input transition from 1 to 0 or vice-versa
\item For this reason, CMOS technologies are currently popular

\item CMOS was a huge improvement over TTL, the dominant technology before
	CMOS (in the 1970s).  In particular, a TTL NAND
	gate had 4 resistors, 1 diode, and 5 transistors, compared to
	the 4 transistors of the CMOS NAND.  

	Further, CMOS has broader
	fan-out (i.e., you can use the output of one CMOS gate as the
	input to 30-50 other CMOS gates; for a TTL NAND, you can use
	the output as the input to about 10 other TTL gates).

        (R. Mann, W05).  Main advantage of CMOS over TTL is lower power.  Only
        one transitor is on at a time, so ideally, no DC current flows.  Also,
        by using Field Effect Transistors (FETs) instead of standard (bipolar)
        transistors, inputs require almost zero input power.  This explains
        huge fan out.

\end{itemize}
\end{frame}
\fi\ENotes


\begin{frame}[fragile]
\Title{CMOS Circuit Analysis}
\begin{tcolorbox}[enhanced,attach boxed title to top center={yshift=-3mm,yshifttext=-1mm},
  colback=red!5!white,colframe=red!75!black,colbacktitle=red!80!black,
  title=Try this,fonttitle=\bfseries,
  boxed title style={size=small,colframe=red!50!black} ]
  Complete the truth table for the CMOS circuit below:
  	\Figure{!}{2.25in}{1in}{NewTransFigs/cmosnot}

		\begin{center}
		\begin{tabular}{c|cc|c}
		A & $Q_1$ & $Q_2$ & F\\\hline
		0 & & \\
		1 & & \\
		\end{tabular}
		\end{center}
\end{tcolorbox}

\end{frame}
\BNotes\ifnum\Notes=1
\begin{frame}[fragile]
Instructor's Notes:
\begin{itemize}
\item The important thing here is that whether the input to the inverter is 1 or 0, there is no current flow
\item There is a small amount of current flow during an input transition from 1 to 0 or vice-versa
\item For this reason, CMOS technologies are currently popular

\item CMOS was a huge improvement over TTL, the dominant technology before
	CMOS (in the 1970s).  In particular, a TTL NAND
	gate had 4 resistors, 1 diode, and 5 transistors, compared to
	the 4 transistors of the CMOS NAND.  

	Further, CMOS has broader
	fan-out (i.e., you can use the output of one CMOS gate as the
	input to 30-50 other CMOS gates; for a TTL NAND, you can use
	the output as the input to about 10 other TTL gates).

        (R. Mann, W05).  Main advantage of CMOS over TTL is lower power.  Only
        one transitor is on at a time, so ideally, no DC current flows.  Also,
        by using Field Effect Transistors (FETs) instead of standard (bipolar)
        transistors, inputs require almost zero input power.  This explains
        huge fan out.

\end{itemize}
\end{frame}
\fi\ENotes


\ifnum\Ans=1
\begin{frame}[fragile]
\Title{Solution: CMOS NOT}
  	\Figure{!}{2.25in}{1in}{NewTransFigs/cmosnot}

		\begin{center}
		\begin{tabular}{c|cc|c}
		A & $Q_1$ & $Q_2$ & F\\\hline
		0 & Low & High & 1\\
		1 & High & Low & 0
		\end{tabular}
		\end{center}
  The CMOS circuit implements a \texttt{NOT} gate.

\end{frame}
\BNotes\ifnum\Notes=1
\begin{frame}[fragile]
Instructor's Notes:
\begin{itemize}
\item The important thing here is that whether the input to the inverter is 1 or 0, there is no current flow
\item There is a small amount of current flow during an input transition from 1 to 0 or vice-versa
\item For this reason, CMOS technologies are currently popular

\item CMOS was a huge improvement over TTL, the dominant technology before
	CMOS (in the 1970s).  In particular, a TTL NAND
	gate had 4 resistors, 1 diode, and 5 transistors, compared to
	the 4 transistors of the CMOS NAND.  

	Further, CMOS has broader
	fan-out (i.e., you can use the output of one CMOS gate as the
	input to 30-50 other CMOS gates; for a TTL NAND, you can use
	the output as the input to about 10 other TTL gates).

        (R. Mann, W05).  Main advantage of CMOS over TTL is lower power.  Only
        one transitor is on at a time, so ideally, no DC current flows.  Also,
        by using Field Effect Transistors (FETs) instead of standard (bipolar)
        transistors, inputs require almost zero input power.  This explains
        huge fan out.

\end{itemize}
\end{frame}
\fi\ENotes
\fi

\begin{frame}[fragile]
\Title{CMOS NAND}
\begin{multicols}{2}
    \Figure{!}{4in}{2in}{NewTransFigs/cmosnand}
    
\columnbreak
  \begin{tcolorbox}[enhanced,attach boxed title to top center={yshift=-3mm,yshifttext=-1mm},
  colback=red!5!white,colframe=red!75!black,colbacktitle=red!80!black,
  title=Try this,fonttitle=\bfseries,
  boxed title style={size=small,colframe=red!50!black} ]
  Complete the truth table for the CMOS circuit shown:
  \begin{center}   
  \small
\begin{tabular}{cc|cccc|c}
A	& B	& $Q_1$	& $Q_2$	& $Q_3$	& $Q_4$	& Z\\
\hline
% 0	& 0	& Low	& Low	& High	& High	& 1\\
0	& 0	& 	& 	& 	& 	& \\
0	& 1	& 	&	&	&	&\\
1	& 0	& 	&	&	&	&\\
1	& 1	& 	&	&	&	&
\end{tabular}
   \end{center}
\end{tcolorbox}
\end{multicols}

\end{frame}
\BNotes\ifnum\Notes=1
\begin{frame}[fragile]
Instructor's Notes:
\begin{itemize}
\item A NOR gate can be built in similar fashion (exercise for keen students?)

	The exercise is simple: it looks a lot like the CMOS NAND, except the
	PMOS transistors are in series while the NMOS transistors are in
	parallel.

\item In "Fundamentals of Digital Logic Design", by Brown
	and Vranesic, section 3.2, page 65, it says, "Because of the way the
	transistors operate, an NMOS transistor cannot be used to pull its
	drain terminal completely up to $V_{DD}$. Similarly, a PMOS terminal
	cannot be used to pull its drain terminal completely down to Gnd. We
	discuss the operation of MOSFETs in considerable detail in section
	3.8."

	This is important, since otherwise you could get an AND by exchanging 
	power and ground in the NAND transistor circuit on this slide.

\end{itemize}
\end{frame}
\fi\ENotes

\ifnum\Ans=1
\begin{frame}\frametitle{Solution: CMOS NAND Gate}

\Figure{!}{4in}{2in}{NewTransFigs/cmosnand}
\begin{table}[H]
\begin{center}
    \begin{tabular}{ |p{1cm} |p{1cm} || p{1cm} |p{1cm} |p{1cm} |p{1cm} || p{1cm} |}
    \hline 
A & B & $Q_1$ & $Q_2$ & $Q_3$ & $Q_4$ & $Z$  \\ \hline
0 & 0 & Low  & Low  & High & High & 1  \\ \hline
0 & 1 & Low  & High & High & Low  & 1 \\ \hline
1 & 0 & High & Low  & Low  & High & 1\\ \hline
1 & 1 & High & High & Low  & Low  & 0  \\ \hline
\end{tabular}
\small\caption{CMOS NAND Gate Analysis}
\end{center}
\end{table}
\end{frame}
\fi 

%%%added by ZHK for CMOS NOR example with pmos in series

% \begin{frame}[fragile]
% \Title{CMOS NOR}
% \begin{multicols}{2}
%     \Figure{!}{4in}{2in}{NewTransFigs/cmosnor}
    
% \columnbreak
%   \begin{tcolorbox}[enhanced,attach boxed title to top center={yshift=-3mm,yshifttext=-1mm},
%   colback=red!5!white,colframe=red!75!black,colbacktitle=red!80!black,
%   title=Try this,fonttitle=\bfseries,
%   boxed title style={size=small,colframe=red!50!black} ]
%   Complete the truth table for the CMOS circuit shown:
%   \begin{center}   
%   \small
% \begin{tabular}{cc|cccc|c}
% A	& B	& $Q_1$	& $Q_2$	& $Q_3$	& $Q_4$	& Z\\
% \hline
% % 0	& 0	& Low	& Low	& High	& High	& 1\\
% 0	& 0	& 	& 	& 	& 	& \\
% 0	& 1	& 	&	&	&	&\\
% 1	& 0	& 	&	&	&	&\\
% 1	& 1	& 	&	&	&	&
% \end{tabular}
%    \end{center}
% \end{tcolorbox}
% \end{multicols}

% \end{frame}
% \BNotes\ifnum\Notes=1
% \begin{frame}[fragile]
% Instructor's Notes:
% \begin{itemize}
% \item A NOR gate can be built in similar fashion (exercise for keen students?)

% 	The exercise is simple: it looks a lot like the CMOS NAND, except the
% 	PMOS transistors are in series while the NMOS transistors are in
% 	parallel.

% \item In "Fundamentals of Digital Logic Design", by Brown
% 	and Vranesic, section 3.2, page 65, it says, "Because of the way the
% 	transistors operate, an NMOS transistor cannot be used to pull its
% 	drain terminal completely up to $V_{DD}$. Similarly, a PMOS terminal
% 	cannot be used to pull its drain terminal completely down to Gnd. We
% 	discuss the operation of MOSFETs in considerable detail in section
% 	3.8."

% 	This is important, since otherwise you could get an AND by exchanging 
% 	power and ground in the NAND transistor circuit on this slide.

% \end{itemize}
% \end{frame}
% \fi\ENotes

% \begin{frame}[fragile]
% \Title{Solution: CMOS NOR}

%     \Figure{!}{4in}{2in}{NewTransFigs/cmosnor}
    

%   \begin{table}[H]
% \begin{center}
%     \begin{tabular}{ |p{1cm} |p{1cm} || p{1cm} |p{1cm} |p{1cm} |p{1cm} || p{1cm} |}
%     \hline 
% A & B & $Q_1$ & $Q_2$ & $Q_3$ & $Q_4$ & $Z$  \\ \hline
% 0 & 0 & Low  & Low  & High & High & 1  \\ \hline
% 0 & 1 & Low  & High & High & Low  & 0 \\ \hline
% 1 & 0 & High & Low  & Low  & High & 0\\ \hline
% 1 & 1 & High & High & Low  & Low  & 0  \\ \hline
% \end{tabular}
% \caption{CMOS NOR Gate Analysis}
% \end{center}
% \end{table}
% \end{frame}


\begin{frame}[fragile]
\Title{CMOS AND}
\begin{itemize}
\item To get AND gate: add NOT at the end of NAND
% \item Example: 
	\Figure{!}{4in}{2in}{NewTransFigs/cmos2and}
 \item NAND is preferred to AND in actual circuits: lower transistor count:
 \begin{itemize}
\item The NAND gate needs 4 transistors.
\item The NOT gate needs 2 transistors.
\item The AND gate needs 6 transistors.
\end{itemize}
\end{itemize}
\end{frame}
\BNotes\ifnum\Notes=1
\begin{frame}[fragile]
Instructor's Notes:
\begin{itemize}
\item Note that if we have a two-level circuit (OR of ANDs, or sum of products), we can replace each gate by NAND, and the resulting circuit computes the same function
\item Hence we can continue to use AND-OR design
\end{itemize}
\end{frame}
\fi\ENotes

%-----------------------------------------------------
% \begin{frame}\frametitle{CMOS Strong AND Gate}

% Place a CMOS NOT gate after the CMOS NAND gate to get the AND gate.
% \begin{itemize}
% \item The NAND gate needs 4 transistors.
% \item The NOT gate needs 2 transistors.
% \item The AND gate needs 6 transistors.
% \end{itemize}
% The NAND gate design is better than AND gate.
% \begin{figure}[H]
% \centering
% {\includegraphics[scale=0.15]{Figs/cmos-and}}
% \end{figure}

% \end{frame}


\begin{frame}[fragile]
\Title{CMOS 3 Input NAND}
\begin{itemize}
\item $n$-input NAND: 2 transistors per input

\item Example: 3 Input NAND
	\Figure{!}{5in}{2.5in}{NewTransFigs/cmos3nand}
\end{itemize}
\end{frame}
\BNotes\ifnum\Notes=1
\begin{frame}[fragile]
Instructor's Notes:
\begin{itemize}
\item There is a limit to the number of inputs, but it's pretty high
\end{itemize}
\end{frame}
\fi\ENotes

\begin{frame}[fragile]
\Title{Tri-State Buffer Gate}
\begin{itemize}
	\item Has three outputs
		0, 1, and \hl{floating} {\em (connected to neither power nor ground)}
		\Figure{!}{2.5in}{1.5in}{NewTransFigs/tgate}
	\item $C=1$ then , then $\bar{C}=0$ 
	\begin{itemize}
		\item NMOS gate passes 0 well 
		% \item $\bar{C}=0$ and
  \item PMOS gate passes 1 well
  \item $F$ is strong 0 or strong 1
	\end{itemize}
	\item $C=0$, then $\bar{C}=1$ and both transistors are off (output is floating).
\end{itemize}
\end{frame}
\BNotes\ifnum\Notes=1
\begin{frame}[fragile]
Instructor's Notes:
\begin{itemize}
\item The floating state is like a physical disconnection
\item It cannot be propagated through other gates
\item One caution with three-state buffers: note that when C is high, then
	the output IS the input.  Ie, with a CMOS NAND, the output is connected
	either to power or ground.  With a CMOS Three-state, when C is high,
	the output is wired to the input.  This means we don't get the power
	boost that is normally associated with passing the signal through a
	gate (i.e., the three-state gate doesn't have a fan-out count, since
	its output is really the output of another gate).
\end{itemize}
\end{frame}
\fi\ENotes






\begin{frame}[fragile]
\STitle{Using Tri-State Buffers}
		\Figure{!}{1.25in}{1in}{Figs/tristate}
\begin{itemize}
\item High-impedance outputs can be ``tied together'' without problems
\item Normally, do not tie output lines together
		\Figure{!}{1.75in}{1in}{Figs/wiredORa}
\end{itemize}
\end{frame}

\BNotes\ifnum\Notes=1
\begin{frame}[fragile]
Instructor's Notes:
\begin{itemize}
\item The reason for not tying output lines together is that 1 (high
voltage) means a low-resistance path to power, and 0 (low voltage)
means a low-resistance path to ground. If, in the crossed-out example
above, the output of one AND gate was 1 and of the other 0, this would
mean a direct connection between power and ground -- a short-circuit.
\item In some older technologies (TTL open collector, for example), you
	could use a Wired-OR.
\end{itemize}
\end{frame}
\fi\ENotes

\begin{frame}[fragile]
\Title{Tri-State Buffers Circuit Analysis}
\begin{tcolorbox}[enhanced,attach boxed title to top center={yshift=-3mm,yshifttext=-1mm},
  colback=red!5!white,colframe=red!75!black,colbacktitle=red!80!black,
  title=Try this,fonttitle=\bfseries,
  boxed title style={size=small,colframe=red!50!black} ]
\begin{multicols}{2}
    \Figure{!}{2.5in}{1in}{Figs/xorTrans} 

    \columnbreak
    Complete the truth table for the circuit illustrated.
        \begin{center}
        \begin{tabular}{cc|cc|cc|c}
                $X$ & $Y$ & $\bar{X}$ & $\bar{Y}$ & $F_0$ & $F_1$ & $F$\\
                \hline
                0 & 0 &&&&&\\
                0 & 1 &&&&&\\
                1 & 0 &&&&&\\
                1 & 1 &&&&&
        \end{tabular}
        \end{center}             
\end{multicols}
\end{tcolorbox}
 
        Circuit analysis with tri-state buffers:
        \begin{itemize}
                \item Label floating output as '---' or Z
                \item Tied lines better have exactly one non-floating!
        \end{itemize}
\BNotes\ifnum\Notes=1
~
\fi\ENotes
\end{frame}

%-----------------------------------------------------
\ifnum\Ans=1
\begin{frame}\frametitle{Solution: XOR from Tri-state Buffers}

We implemented XOR gate using Tri-State Buffers
\begin{table}[H]
\begin{center}
    \begin{tabular}{cc| cc| cc| c }
 $X$ & $Y$ & $\overline{X}$ & $\overline{Y}$ & $F_0$ & $F_1$ &  $F$ \\ \hline
  0 & 0 & 1 & 1 & 0     & Z & 0  \\ \hline
  0 & 1 & 1 & 0 & Z  &  1   & 1 \\ \hline
  1 & 0 & 0 & 1 & 1     & Z & 1 \\ \hline
  1 & 1 & 0 & 0 & Z  &  0   &  0\\ \hline
\end{tabular}
%\caption{}
\end{center}
\end{table} 
\begin{itemize}
\item A 2-input XOR using CMOS NAND
\begin{itemize}
\item Needs 4 NAND gates
\item Each NAND with 2 inputs: 4 transistors
\item \hl{TOTAL: $4\times 4 = 16$ transistors.}
\end{itemize}

\item XOR using Tri-state buffer:
\begin{itemize}

\item Needs 2 tri-state buffers
% \item NOT gate: 2 transistors
\item Each tri-state buffer: 2 transistors
\item \hl{TOTAL: $2\times 2 = 4$ transistors.}
\end{itemize}
\end{itemize}

\end{frame}
\fi
%-----------------------------------------------------
%  \begin{frame}\frametitle{Transistor Count for XOR}


% \begin{itemize}
% \item A 2-input XOR using CMOS NAND
% \begin{itemize}
% \item needs 4 NAND gates
% \item Each NAND is a 2 input NAND, which requires 4 transistors
% \item  TOTAL: $4\times 4 = 16$ transistors.
% \end{itemize}

% % % \begin{itemize}
% % % \item NOT gate: 2 transistors for a not gate, 2 not gates
% % % \item $\overline{A}$, $\overline{B}$, $A$, and $B$, are connected to 4 PMOS, 4 NMOS. 
% % % \end{itemize}
% % % TOTAL: $2\times 2 + 8 = 12$.

% % \hfill\break
% \item XOR using Tri-state buffer:
% \begin{itemize}

% \item Three-state buffer: 4 transistors
% % \item NOT gate: 2 transistors
% \item TOTAL: 4 transistors
% \end{itemize}
% \end{itemize}

% \end{frame}
% %-----------------------------------------------------
\begin{frame}\frametitle{Making Multiplexors from Tri-state Buffers}

% We can use the tri-state buffers to make a MUX. Such a MUX does not need a big OR gate. Exactly one select line is 1.

\begin{figure}[H]
\centering
	\subfloat[4-to-1 mux with OR gate.]{\includegraphics[width=0.4\textwidth]{Figs/4-1-mux-OR-gate}}
	\subfloat[Tri-state mux without OR gate.]{\includegraphics[width=0.4\textwidth]{Figs/tri-state-mux}}
%\caption{}
\end{figure}
IMPORTANT: Must ensure that at most one select input is 1, or
		short-circuit may result (physical meltdown)\\

\end{frame}


% \begin{frame}[fragile]
% \Title{Making Multiplexors from Three-State Buffers}
% 		\PHFigure{!}{4in}{2in}{PHALL/B22}{Similar to Figure A.9.2}

% IMPORTANT: Must ensure that at most one select input is 1, or
% 		short-circuit may result (physical meltdown)\\
% \BNotes\ifnum\Notes=1
% Instructor's Notes:
% This figure scales up in an obvious fashion.
% \fi\ENotes
% \end{frame}


%-----------------------------------------------------

\begin{frame}\frametitle{CMOS Summary}
\begin{itemize}
    \item Terminology
    \begin{itemize}
    \item N- and P-type Metal-Oxide-Semiconductor (MOS) and CMOS
    % Metal-Oxide-Semiconductor (NMOS/PMOS)
    % \item Complementary Metal Oxide Semiconductor (CMOS)
    \end{itemize}
\item Motivation
    \begin{itemize}
    \item Instruction Set executed by a Function
     \item Function implemented in Digital Logic Circuits
     \item Digital Logic Circuits are composed of Logic Gates
    \item Logic Gates made from CMOS transistors
    \end{itemize}
\item Key Ideas
\begin{itemize}
    \item A \textbf{short circuit} is a low resistance path between power and ground.
    \item A \textbf{float state} is being disconnected from \textbf{both} power and ground.
\item For strong transmission, always connect nMOS to ground and pMOS to power.
% \item Note that if we have a two-level circuit (OR of ANDs, or sum of products), we can replace each gate by NAND, and the resulting circuit computes the same function.
% \item Therefore, we can continue to use AND-OR design.
\end{itemize}

\end{itemize}
 \end{frame}
\begin{frame}[fragile]
\STitle{Conclusion}
 \underline{\textbf{Lecture Summary}}
 \begin{itemize}
 \item CMOS Circuit Analysis
 \item \hl{Proved}: NAND better than AND
 \item Efficient ways of using transistors to implement logic, e.g. 
 \begin{itemize}
     \item Tri-state buffers for MUX
     \item Tri-state buffers for XOR logic
 \end{itemize}
 \end{itemize}
 \underline{\textbf{Assigned Textbook Readings}}
\begin{itemize}
\item Chapter 1, Section 1.5
     \item Lecture Notes
     \end{itemize}
    \underline{\textbf{Next Steps}}
    \begin{itemize} 
    \item \textbf{Ask} questions in the next tutorial or office hours.
 \end{itemize}
\end{frame}

\begin{frame}{Additional Slides}
     Remaining slides are additional notes for your information.
 \end{frame}

\begin{frame}[fragile]
\Title{CMOS NOR}
\begin{multicols}{2}
    \Figure{!}{4in}{2in}{NewTransFigs/cmosnor}
    
\columnbreak
  \begin{tcolorbox}[enhanced,attach boxed title to top center={yshift=-3mm,yshifttext=-1mm},
  colback=red!5!white,colframe=red!75!black,colbacktitle=red!80!black,
  title=Try this,fonttitle=\bfseries,
  boxed title style={size=small,colframe=red!50!black} ]
  Complete the truth table for the CMOS circuit shown:
  \begin{center}   
  \small
\begin{tabular}{cc|cccc|c}
A	& B	& $Q_1$	& $Q_2$	& $Q_3$	& $Q_4$	& Z\\
\hline
% 0	& 0	& Low	& Low	& High	& High	& 1\\
0	& 0	& 	& 	& 	& 	& \\
0	& 1	& 	&	&	&	&\\
1	& 0	& 	&	&	&	&\\
1	& 1	& 	&	&	&	&
\end{tabular}
   \end{center}
\end{tcolorbox}
\end{multicols}

\end{frame}
\BNotes\ifnum\Notes=1
\begin{frame}[fragile]
Instructor's Notes:
\begin{itemize}
\item A NOR gate can be built in similar fashion (exercise for keen students?)

	The exercise is simple: it looks a lot like the CMOS NAND, except the
	PMOS transistors are in series while the NMOS transistors are in
	parallel.

\item In "Fundamentals of Digital Logic Design", by Brown
	and Vranesic, section 3.2, page 65, it says, "Because of the way the
	transistors operate, an NMOS transistor cannot be used to pull its
	drain terminal completely up to $V_{DD}$. Similarly, a PMOS terminal
	cannot be used to pull its drain terminal completely down to Gnd. We
	discuss the operation of MOSFETs in considerable detail in section
	3.8."

	This is important, since otherwise you could get an AND by exchanging 
	power and ground in the NAND transistor circuit on this slide.

\end{itemize}
\end{frame}
\fi\ENotes

\ifnum\Ans=1
\begin{frame}[fragile]
\Title{Solution: CMOS NOR}

    \Figure{!}{4in}{2in}{NewTransFigs/cmosnor}
    

  \begin{table}[H]
\begin{center}
    \begin{tabular}{ |p{1cm} |p{1cm} || p{1cm} |p{1cm} |p{1cm} |p{1cm} || p{1cm} |}
    \hline 
A & B & $Q_1$ & $Q_2$ & $Q_3$ & $Q_4$ & $Z$  \\ \hline
0 & 0 & Low  & Low  & High & High & 1  \\ \hline
0 & 1 & Low  & High & High & Low  & 0 \\ \hline
1 & 0 & High & Low  & Low  & High & 0\\ \hline
1 & 1 & High & High & Low  & Low  & 0  \\ \hline
\end{tabular}
\caption{CMOS NOR Gate Analysis}
\end{center}
\end{table}
\end{frame}
\fi

% \begin{frame}[fragile]
%     Additional slides for curious minds
% \end{frame}
% %-----------------------------------------------------

\ifnum\Ans=1{
\begin{frame}\frametitle{CMOS Weak AND Gate}

In the NAND gate, if we swap power with ground we get the AND gate. We do \textbf{not} use this design. Example:
\begin{itemize}
\item When $A=0$ and $B=0$, $Q_1$ and $Q_2$ have low resistance. But since $Q_1$ and $Q_2$ are PMOS, they transmit weak 0s between $Z$ and ground.
\end{itemize}
\begin{figure}[H]
\centering
\subfloat[NAND gate]{\includegraphics[width=0.4\textwidth]{Figs/cmos-nand}}
\subfloat[Weak AND gate]{\includegraphics[width=0.4\textwidth]{Figs/weak-and}}
\end{figure}

\end{frame}

%-----------------------------------------------------
\begin{frame}\frametitle{CMOS Weak AND Gate}

In the NAND gate, if we swap power with ground we get the AND gate. We do \textbf{not} use this design. Example:
\begin{itemize}
\item When $A=1$ and $B=1$, $Q_3$ and $Q_4$ have low resistance. But since $Q_3$ and $Q_4$ are NMOS they transmit weak 1s between power and $Z$. 
\end{itemize}
\begin{figure}[H]
\centering
\subfloat[NAND gate]{\includegraphics[width=0.4\textwidth]{Figs/cmos-nand}}
\subfloat[Weak AND gate]{\includegraphics[width=0.4\textwidth]{Figs/weak-and}}
\end{figure}
\end{frame}

}\fi